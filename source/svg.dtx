% \CheckSum{5075}
% \iffalse meta-comment
%<*identify>
%
%  svg and svg-extract -- Handling SVG pictures in LaTeX documents
%
%% ----------------------------------------------------------------------------
%%
%% Copyright (C) Philip Ilten <philten@cern.ch>, 2012-2016  
%% Copyright (C) Falk Hanisch <hanisch.latex@outlook.com>, 2017-
%% 
%% ----------------------------------------------------------------------------
%% 
%% This work may be distributed and/or modified under the conditions
%% of the LaTeX Project Public License, version 1.3c of the license.
%% The latest version of this license is in
%%     http://www.latex-project.org/lppl.txt
%% and version 1.3c or later is part of all distributions of
%% LaTeX 2005/12/01 or later and of this work.
%% 
%% This work has the LPPL maintenance status "author-maintained".
%% 
%% The current maintainer and author of this work is Falk Hanisch.
%%
%% This work consists of the main source file README.md, LICENSE.md, svg.dtx 
%% and the derived documentation svg.pdf as well as the example files
%% svg-example.svg, svg-preamble.tex and root.C (with derived file root.svg).
%% 
%% ----------------------------------------------------------------------------
%%
%</identify>
%<*dtx>
\ifx\documentclass\undefined
\input docstrip.tex
\keepsilent

\declarepreamble\svgbasepreamble

svg -- Include SVG pictures in LaTeX documents using Inkscape

\endpreamble
\declarepreamble\svgextractpreamble

svg-extract -- Extract independent graphic files from SVG pictures

\endpreamble
\askforoverwritefalse
\usedir{tex/latex/svg}
\generate{%
  \usepreamble\svgbasepreamble
  \file{svg.sty}{
    \from{svg.dtx}{package,identify,base}
    \from{svg.dtx}{package,option,base}
    \from{svg.dtx}{package,body,base}
  }
  \usepreamble\svgextractpreamble
  \file{svg-extract.sty}{
    \from{svg.dtx}{package,identify,extract}
    \from{svg.dtx}{package,option,extract}
    \from{svg.dtx}{package,body,extract}
  }
%  \usepreamble\defaultpreamble
%  \file{svg-old.sty}{
%    \from{svg.dtx}{package,old}
%  }
}
\else
\let\endbatchfile\relax
\fi
\endbatchfile
%</dtx>
%
% \fi
%
% \CharacterTable
%  {Upper-case    \A\B\C\D\E\F\G\H\I\J\K\L\M\N\O\P\Q\R\S\T\U\V\W\X\Y\Z
%   Lower-case    \a\b\c\d\e\f\g\h\i\j\k\l\m\n\o\p\q\r\s\t\u\v\w\x\y\z
%   Digits        \0\1\2\3\4\5\6\7\8\9
%   Exclamation   \!     Double quote  \"     Hash (number) \#
%   Dollar        \$     Percent       \%     Ampersand     \&
%   Acute accent  \'     Left paren    \(     Right paren   \)
%   Asterisk      \*     Plus          \+     Comma         \,
%   Minus         \-     Point         \.     Solidus       \/
%   Colon         \:     Semicolon     \;     Less than     \<
%   Equals        \=     Greater than  \>     Question mark \?
%   Commercial at \@     Left bracket  \[     Backslash     \\
%   Right bracket \]     Circumflex    \^     Underscore    \_
%   Grave accent  \`     Left brace    \{     Vertical bar  \|
%   Right brace   \}     Tilde         \~}
%
% \iffalse
%<*dtx>
\ifx\ProvidesFile\undefined\def\ProvidesFile#1[#2]{}\fi
\ProvidesFile{svg.dtx}[%
%</dtx>
%<package&identify>\NeedsTeXFormat{LaTeX2e}
%<package&identify&base>\ProvidesPackage{svg}[%
%<package&identify&extract>\ProvidesPackage{svg-extract}[%
%<*dtx|package&identify>
  2019/09/08 v2.03\space%
%<package&identify&base>  (include SVG pictures)%
%<package&identify&extract>  (extract independent graphics from SVG pictures)%
]
%</dtx|package&identify>
%<*dtx>
\documentclass[english,xindy]{tudscrdoc}[2017/03/27]
\ifpdftex{
  \usepackage[T1]{fontenc}
}{
  \usepackage{fontspec}
}
\usepackage{babel}
\KOMAoptions{parskip=half-}

\usepackage{svg}
\usepackage{svg-extract}
\svgpath{{./examples/}}
\svgsetup{extract=true,extractpreamble=svg-preamble.tex,clean=true}

\usepackage{xpatch}
\xpretocmd{\svg@process}{\MakePercentComment}{}{%
  \GenericError{}{Patching \string\svg@process not successful}{%
    see `\jobname.dtx' on input line \the\numexpr\inputlineno-4\relax%
  }{}%
}
\ifdefined\pdfsuppresswarningpagegroup
  \pdfsuppresswarningpagegroup=1
\fi

\usepackage{listings}
\lstset{%
  inputencoding=utf8,extendedchars=true,%
  literate=%
    {ä}{{\"a}}1 {ö}{{\"o}}1 {ü}{{\"u}}1
    {Ä}{{\"A}}1 {Ö}{{\"O}}1 {Ü}{{\"U}}1
    {~}{{\textasciitilde}}1 {ß}{{\ss}}1
}
\newcommand*\svglstbelowskip{\medskipamount}
\AtBeginDocument{%
  \edef\svglstbelowskip{\the\glueexpr\medskipamount-\parskip\relax}%
}
\lstdefinestyle{svg}{%
%  language=[LaTeX]TeX,%
  basicstyle=\small\ttfamily,%
  columns=flexible,%
  frame=single,%
  backgroundcolor=\color{yellow!10},%
  xleftmargin=\fboxsep,%
  xrightmargin=\fboxsep,%
  gobble=2,%
  escapechar=§,%
  belowskip=\svglstbelowskip,%
}
\lstnewenvironment{svgexample}[1][]{%
  \lstset{style=svg,#1}%
}{}
\usepackage{enumitem}
\newlist{DescribeValues}{description}{1}
\setlist[DescribeValues]{font=\normalfont\val,style=nextline,noitemsep}

\usepackage{subcaption}
\usepackage{relsize}

\usepackage{bookmark}
\usepackage[babel]{microtype}

\newcommand*\default[1]{\val{\textit{#1}}}

%\EnableCrossrefs
\CodelineIndex
\RecordChanges
\GetFileInfo{svg.dtx}
\title{The packages \pkg{svg} and \pkg{svg-extract}}
\author{%
  Philip Ilten (2012--2016)\\%
  Falk Hanisch (2017--)\\%
  \url{https://github.com/mrpiggi/svg}\\%
  \mailto{hanisch.latex@outlook.com}%
}
\date{\fileversion\nobreakspace(\filedate)}

\begin{document}
  \maketitle%
  \begin{abstract}
    \noindent
    The \pkg{svg} package is intended for the automated integration of 
    SVG~graphics into \LaTeX~documents. Therefor the capabilities provided 
    by \app{Inkscape}---or more precisely its command line tool---are used 
    to export the text within a SVG~graphic to a separate file, which is 
    then rendered by \LaTeX{}. The two commands \cs{includesvg} and 
    \cs{includeinkscape} are provided as central user\hbox{-}interface, 
    which are very similar to the \cs{includegraphics} command of the 
    \pkg{graphicx} package.
    
    In addition, the package \pkg{svg-extract} allows the extraction of these 
    graphics into independent files in different graphic formats, exactly as it 
    is rendered within the \LaTeX~document. For the creation of these graphics 
    in the well\hbox{-}known formats PDF, EPS and~PS, \LaTeX{} and possibly 
    conversion tools shipped with the distribution are used. If the graphics 
    are required in other file formats, either \app{ImageMagick} or 
    \app{Ghostscript} can be invoked.
  \end{abstract}
  \tableofcontents
  \ToDo{move multiple \val{@} from internal cs to \val{svg@@...}}[v2.03]
  \ToDo{use \cs{@tempa} instead of \cs{svg@tempa} in FamilyKeys!}[v2.03]
  \ToDo{call makefile from .bat}[v2.03]
  \ToDo{%
    general note to path and file names (umlauts or any other non-ascii) at the 
    very beginning, mention \pkg{grffile}?%
  }[v2.03]
% It's recommended to avoid umlauts or any other non-ascii characters as well  
% as any spaces and/or quotes respectively \cs{dq} both in paths and file~names,
% especially when DVI~output is active.
  \ToDo{Hint for cropped text}[v2.03]
  \DocInput{\filename}
\end{document}
%</dtx>
% \fi
%
% \changes{v1.0}{2004/11/05}{initial version by Philip Ilten}^^A
% \changes{v2.00}{2016/09/23}{new maintainer: Falk Hanisch}^^A
% \changes{v2.00}{2017/03/27}{re-implementation from scratch}^^A
% \changes{v2.00}{2017/03/27}{package \pkg{subfig} not required anymore}^^A
% \changes{v2.00}{2017/03/27}{%
%   support of subfigures stopped due to the huge number of packages which deal 
%   with this topic and the large variety of implementing this functionality; 
%   naming exported graphics after their consecutive numbering can't be ensured
%   for all variants of subfigures, so it's neglected%
% }^^A
% \changes{v2.01}{2017/11/28}{%
%   usage of \cs{input}\marg{tex filename} within \app{Inkscape}~graphics
%   locates files in all declared searched folders
% }^^A
% \changes{v2.02}{2018/09/07}{package \pkg{trimspaces} required}^^A
% \changes{v2.02}{2018/09/07}{multiple dots within file names possible}^^A
% \changes{v2.02}{2018/09/07}{%
%   fixed errors with active double quotes from \pkg{babel} in path arguments%
% }^^A
% \changes{v2.02a}{2018/10/17}{%
%   fix bug for package \pkg{polyglossia} which fakes \pkg{babel} poorly%
% }^^A
% \changes{v2.02b}{2018/11/12}{%
%   fix bug for package \pkg{tikzscale} which changes \cs{includegraphics} 
%   globally%
% }^^A
% \changes{v2.03}{2019/06/28}{%
%   command line format of \app{Inkscape~1.0} and later is supported%
% }^^A
% \changes{v2.03}{2019/06/25}{%
%   files exported with \app{Inkscape} are not moved from root directory but 
%   created directly in the path given with \opt{inkscapepath}%
% }^^A
% \changes{v2.03}{2019/07/11}{%
%   files converted with \app{ImageMagick} or \app{Ghostscript} are not moved 
%   from root directory but created directly in the path given with
%   \opt{extractpath}%
% }^^A
% \changes{v2.03}{2019/07/18}{%
%   conditional invocation of export/extraction/conversion based on file 
%   modification date implemented for Xe\TeX{}%
% }^^A
%
%
%
% \part{User documentation}
% \section{Introduction}
%
% The open source program \app{Inkscape} has provided an excellent resource for 
% the simple and easy creation of images and diagrams using a graphical user
% interface. The work by Johan\,B.\,C.~Engelen has further enhanced the ability
% of \app{Inkscape} to split a SVG~file into a text component that can be
% compiled with \LaTeX{}, and an image component that can be imported as a 
% PDF~file. For further information see the documentation of 
% \hrfn{http://www.ctan.org/pkg/svg-inkscape}{\pkg{svg-inkscape} on CTAN}.
% The procedure described therein is taken up and consistently expanded. Thus, 
% it is now possible to include a SVG~file into a \LaTeX~document where the
% text within the SVG~graphic will be rendered natively by \LaTeX{}.
% 
% Both packages \pkg{svg} and \pkg{svg-extract} rely heavily upon executing 
% commands from the shell using the \cs{ShellEscape} command---or respectively 
% the old known \cs{write18}---for executing a variety of commands directly to 
% the system. So it is necessary to include the flag \file{-{}-shell-escape} 
% when compiling documents using \pkg{svg} and/or \pkg{svg-extract}. The 
% executed commands and the possibilities to adapt their invocation with the 
% appropriate options are described later on in this documentation. All this is 
% done automatically with the \cs{includesvg} command. If you don't want to use
% the \file{-{}-shell-escape} flag, either for security reasons or because
% the export of the SVG~files is done in another way, there's also the command
% \cs{includeinkscape} which includes files already exported by \app{Inkscape}.
%
% An working installation of \app{Inkscape} is required for the automated 
% integration of SVG~graphics, whereby the installation path must be 
% known to the operating system. This can be checked on shell by typing 
% \file{inkscape -V}. Moreover, there are some required packages which are 
% loaded by packages \pkg{svg} and \pkg{svg-extract} to provide the 
% functionallity. These are:
% \begin{description}[parsep=.5\smallskipamount]
% \item[\pkg{scrbase}]
%   for the definition and handling of options in key-value-syntax
% \item[\pkg{ifpdf}, \pkg{ifluatex}, \pkg{ifxetex}]
%   for flow control depending on the used \LaTeX~engine
% \item[\pkg{etexcmds}, \pkg{pdftexcmds}, \pkg{shellesc}]
%   to allocate the same primitives independent of the used \LaTeX~engine
% \item[\pkg{ifplatform}]
%   to control the file~access depending on the operating system
% \item[\pkg{trimspaces}]
%   to remove unwanted spaces in file paths
% \item[\pkg{graphicx}]
%   for including the graphic~files after the \app{Inkscape}~export
% \item[\pkg{xcolor},\pkg{transparent}]
%   are possibly needed by the separate \LaTeX~files created by \app{Inkscape}
% \item[\pkg{xr}]
%   is used by \pkg{svg-extract} in order to include labels within the 
%   independent graphic~files 
% \end{description}
%
% If you want to pass options to package \pkg{graphicx}, you must either load 
% it before package \pkg{svg}
% \begin{svgexample}
%   \usepackage§\oarg{options}§{graphicx}
%   §\dots§
%   \usepackage§\oarg{options}§{svg}
% \end{svgexample}
% or use \cs{PassOptionsToPackage}.
% \begin{svgexample}
%   \PassOptionsToPackage§\marg{options}§{graphicx}
%   §\dots§
%   \documentclass§\oarg{options}\marg{class}§
%   §\dots§
%   \usepackage§\oarg{options}§{svg}
% \end{svgexample}
% The usage of packages \pkg{xcolor} and \pkg{transparent} can be switched off 
% while loading package \pkg{svg}. See the two options \opt{usexcolor} and 
% \opt{usetransparent} below.
%
%
%
% \section{Usage of package \pkg{svg}}\label{sec:inkscape}
%
% The purpose of this package is to include SVG~graphics into a \LaTeX~document.
% The command \cs{includesvg} is defined which does all necessary steps for
% this task. It first launches the export of a SVG~file to a supported 
% file~format with Inkscape, if necessary, and includes the exported 
% graphic~file afterwards. The usage and the syntax is quite similiar to the
% command \cs{includegraphics} from the \pkg{graphicx} package. In fact, the
% inclusion of the exported graphic~file is done with \cs{includegraphics}.
%
% \DescribeOptions{usexcolor,usetransparent,noxcolor,notransparent}^^A
% The packages \pkg{xcolor} and \pkg{transparent} are loaded by default at the 
% end of package \pkg{svg}. The listed options are intended to prevent these 
% packages from loading. They are the only options which have to be given while
% loading the \pkg{svg} package. All supported boolean values
% (\val{true/on/yes/false/off/no}) can be assinged to \opt{usexcolor} and 
% \opt{usetransparent}, while \opt{noxcolor} and \opt{notransparent} don't
% accept any value.
% \begin{svgexample}
%   \usepackage§\oarg{options}§{svg}
% \end{svgexample}
%
%
%
% \subsection{General settings}
%
% \DescribeMacro{\svgsetup}^^A
% All other options described in detail below can also be changed after loading 
% the package either in the preamble or within the document. They don't have to 
% be given as optional argument to \cs{usepackage}\oarg{options}|{svg}| but can 
% be set by using macro \cs{svgsetup}\marg{options} where \marg{options} is a 
% comma separated list of options. Settings with \cs{svgsetup} are done in the
% current scope which means globally or within the current group.
% \begin{svgexample}
%   \svgsetup§\marg{options}§
% \end{svgexample}
% Further, it's possible to reset any setting locally with the optional 
% argument of the commands \cs{includesvg}\oarg{options}\marg{svg filename} or
% \cs{includesvg}\oarg{options}\marg{graphic filename}.
%
% \DescribeMacro{\svgpath}^^A
% Most likely you want to organize your SVG~files in a separate folder either 
% as a subfolder in the working directory or elsewhere in your local folder 
% structure. For this purpose, a list of root paths to SVG~files can be
% specified using the \cs{svgpath} command in the same way as \cs{graphicspath}
% is used. Every path has to be given in a group of braces |{}|---even if there 
% is only one---and terminate with |/| last. For example:
% \begin{svgexample}
%   \svgpath{{svg/}{/usr/local/svg/}}
% \end{svgexample}
% would cause the system to look first in the subdirectory \file{svg/} and 
% afterwards in the absolute path \file{/usr/local/svg/}. Further, if no path 
% was specified with \cs{svgpath} or the desired file wasn't found, all
% directories given with \cs{graphicspath} are searched too. Please keep in 
% mind that the current working directory is browsed first in any case. It's 
% recommended to avoid umlauts or any other non-ascii characters as well as 
% any spaces and/or quotes respectively \cs{dq} both in paths and file~names,
% especially when DVI~output is active.
%
%
%
% \subsection{Options for the invocation of \app{Inkscape}}
%
% \DescribeOption{inkscape}^^A
% This option controls, when the export with \app{Inkscape} is invoked and is 
% \default{true} by default.
% \begin{DescribeValues}^^A
% \item[false/off/no]
%   \app{Inkscape} won't be invoked in any case, no export is done.
% \item[\default{true}/on/yes/newer/onlynewer]
%   The export with \app{Inkscape} will only be done, if the exported 
%   graphic~file either does not exist or the file~modification date of the
%   SVG~file is newer than that of the exported graphic~file. Thus the
%   compilation time of the \LaTeX~document can be reduced to the necessary
%   minimum.
% \item[forced/force/overwrite]
%   The \app{Inkscape}~export will definitely be done, any already existing 
%   exported file will overwritten regardlessly.
% \end{DescribeValues}
% 
% In addition to controlling the export behavior, the option \opt{inkscape} can 
% also be used to make additional settings, which then acts as a wrapper for 
% the options described below.
%
% \begin{DescribeValues}^^A
% \item[pdf/eps/ps/png]
%   see \opt{inkscapeformat}|=|\val{pdf/eps/ps/png}
% \item[latex/nolatex]
%   see \opt{inkscapelatex}|=|\val{true/false}
% \item[drawing/page]
%   see \opt{inkscapearea}|=|\val{drawing/page}
% \item[\meta{integer}dpi]
%   see \opt{inkscapedpi}|=|\val{\meta{integer}}
% \end{DescribeValues}
%
% \DescribeOption{inkscapepath}^^A
% The option \opt{inkscapepath} specifies, where the resulting files of the
% \app{Inkscape}~export should be located. The default setting is 
% \default{basesubdir}, which uses the subfolder \file{./svg-inkscape/} 
% within the current working directory.
% \begin{DescribeValues}^^A
% \item[svgdir/svgpath]
%   The PDF/EPS/PS/PNG~graphic~files as well as the \LaTeX~files generated by 
%   \app{Inkscape} will be located in the same directory as the corresponding
%   SVG~file.
% \item[svgsubdir/svgsubpath]
%   Within the folder of the encountered SVG~file, all exported files will be 
%   located in a subfolder named \file{svg-inkscape/}.
% \item[basedir/basepath/jobdir/jobpath]
%   All exported files will be located in the current working directory.
% \item[\default{basesubdir}/basesubpath/jobsubdir/jobsubpath]
%   A subfolder named \file{svg-inkscape/} within the current working directory 
%   will be used for files generated by \app{Inkscape}.
% \item[/path/to/somewhere/]
%   It is also possible to give a custom path, either relative to the current 
%   working directory (\file{./relative/path/}) or as an absolute path.
% \end{DescribeValues}
% \ToDo{Hint for MacOS}[v2.03]
%
% \DescribeOption{inkscapeexe}^^A
% For including a SVG~file, \app{Inkscape} is used to separate the text and
% image from the SVG~file itself. In order to execute the command line tool 
% from shell, the path where the executable is located has to be known to the
% operating system. You can check this by typing \file{inkscape -V} into the 
% shell. If this check fails, you should add the \app{Inkscape} directory to 
% the environment variable \file{PATH} on your OS.
%
% If this is not possible or you aren't willing to do so, you can use option 
% \opt{inkscapeexe} \emph{within the document preamble} to set the absolute 
% path where the executable of \app{Inkscape} is located. The option is set to
% \opt{inkscapeexe}|=|\val{inkscape} by default.
% \ToDo{Hint for MacOS}[v2.03]
%
% \DescribeOption{inkscapeversion}^^A
% The command line tool for \app{Inkscape} is going to be changed slightly from 
% version \val{0.9x} to \val{1.x} and makes some adaptions necessary. It can be
% set either to \opt{inkscapeversion}|=|\val{0}---which is set by default---or
% to any newer version with a value greater than \val{0}. Additionally, it is
% possible to detect the used \app{Inkscape} version automatically. This is
% activated with \opt{inkscapeversion}|=|\val{detect} and done by calling the
% given \opt{inkscapeexe} with parameter \prm{-V} through a pipe.
%
% \centerline{\emph{This functionallity does currently not work with MiK\TeX!}}
%
% \DescribeOption{inkscapename}^^A
% The file names of the \app{Inkscape}~export are derived from the name of the 
% base SVG~file by default but can be modified with 
% \opt{inkscapename}|=|\file{\meta{filename}}. It's possible to use counters for
% specifying the name of the exported file. Repeatedly specifying the same file 
% name will overwrite previously created files. 
% 
% \DescribeOption{inkscapeformat}^^A
% With this option, the \app{Inkscape}~export~format can be controlled. Valid 
% values are \val{pdf}, \val{eps}, \val{ps} and \val{png}, where a \LaTeX~export
% is not possible for \val{png} and option \opt{inkscapelatex} won't have any 
% effect. By default, \opt{inkscapeformat}|=|\val{pdf} is set unless DVI~output
% was detected. In this case \opt{inkscapeformat}|=|\val{eps} is the default
% setting.
%
% \DescribeOption{inkscapelatex}^^A
% If option \opt{inkscapelatex}|=|\val{true} is set, the output is split into 
% a seperate PDF/EPS/PS~file (see option \opt{inkscapeformat}) and a 
% corresponding \LaTeX~file. This is the default setting. Setting 
% \opt{inkscapelatex}|=|\val{false} will result in a single PDF/EPS/PS~file,
% where any contained text won't be rendered by \LaTeX{}. 
%
% \DescribeOption{inkscapearea}^^A
% This option controls which area of the SVG~file should be exported, 
% \default{drawing} is set by default.
% \begin{DescribeValues}^^A
% \item[\default{drawing}/crop]
%   The area exported corresponds to the bounding box of all objects in a
%   drawing, including any that are not on the page.
% \item[page/nocrop]
%   The area exported will correspond to the defined page~area within the
%   SVG~file.
% \end{DescribeValues}
%
% \DescribeOption{inkscapedpi}^^A
% The resolution used either for PNG~export or for fallback rasterization of 
% filtered objects when exporting to PDF/EPS/PS~file. For PNG~export it is set 
% to 300\,dpi by default, if no value was given. The given value should be a 
% positive integer. The default behaviour can be reversed after a given value
% with \opt{inkscapedpi}|=|\val{\cs{relax}}.
%
% \DescribeOption{inkscapeopt}^^A
% You can use this option to pass additional switches to the \app{Inkscape} 
% command line tool. For further information see the documentation of 
% \hrfn{https://inkscape.org/de/doc/inkscape-man.html}{\app{Inkscape}}.
%
% \DescribeOption{svgextension}^^A
% \changes{v2.01}{2017/11/27}{%
%   option \opt{svgextension} added in order to change the format of files 
%   exported by \app{Inkscape} from \file{svg} to a custom one
% }^^A
% The package assumes SVG~files with \file{.svg} extension as source for the 
% \app{Inkscape} export. This option can be used to change this behaviour. 
% For example, in order to process \file{.dia} files instead of \file{.svg} you 
% could use 
% \begin{svgexample}
%   \includesvg[svgextension=dia,§\meta{additional options}§]§\marg{filename}§
% \end{svgexample}
%
%
%
% \subsection{Options for the graphic inclusion}
%
% \DescribeOptions{width,height,distort,scale}^^A
% \changes{v2.02}{2018/09/07}{%
%   option \opt{distort} (or \opt{keepaspectratio}) added for distortion of
%   included graphics%
% }^^A
% The width of the included graphic~file can be specified via the \opt{width} 
% option and the height by the \opt{height} option. If both the width and
% height are specified, the figure will be scaled such that neither of the
% specified dimensions is exceeded, unless option \opt{distort}|=|\val{true} is 
% given.\footnote{\ignorespaces%
%   to provide compatibility for package \pkg{graphicx}, it's possible to use
%   \prm{keepaspectratio}|=|\val{true} as alias for \opt{distort}|=|\val{false}
%   and the other way round
% } If \opt{width} and/or \opt{height} once have been set, this can be undone
% by setting them to \val{0pt} or \cs{relax}. If neither \opt{width} nor 
% \opt{height} are set, the included graphic~file can also be scaled by setting 
% \opt{scale} to a positive real number.
%
% \DescribeOptions{pretex,apptex}^^A
% Commands prior and post to the inclusion of the graphic~file may be desired, 
% such as font or color commands. The options \opt{pretex} and \opt{apptex} are
% provided where the \LaTeX~code given to \opt{pretex} is included before the 
% graphic~file and \opt{apptex} right afterwards. For example, to change the 
% size of the included text one could use:
% \begin{svgexample}
%   \includesvg[pretex=\tiny,§\meta{additional options}§]§\marg{svg filename}§
% \end{svgexample}
%
% \DescribeOption{draft}^^A
% This option can be used with booelan values and is equal to the identically 
% named option of the \pkg{graphicx} package. If the \opt{draft} option is
% given to \pkg{graphicx}, it's activated for \pkg{svg} as well.
%
% \DescribeOption{lastpage}^^A
% A \hrfn{https://bugs.launchpad.net/ubuntu/+source/inkscape/+bug/1417470}{bug} 
% concerning the \LaTeX~export has been reported for \app{Inkscape}~0.91. It 
% may happen that within the exported \LaTeX~file, it's attempted to include
% more pages of the PDF~graphics than actually exist. The \pkg{svg} package
% attempts to bypass the resulting error.
%
% Consequently, the total number of pages is read and only existing PDF~pages
% are included, if both options \opt{inkscapeformat}|=|\val{pdf} and 
% \opt{lastpage}|=|\val{true} are set. This is the default setting 
% (unless DVI~output is active) and can be switched off with
% \opt{lastpage}|=|\val{false}. It's also possible to set the number of the
% last page included of a PDF~graphic manually as optional parameter for 
% \cs{includesvg} or \cs{includeinkscape}. For details, see the description of
% the respective commands.
%
%
%
% \subsection{Including SVG~files}
%
% \DescribeMacro{\includesvg}^^A
% The command \cs{includesvg} to include a SVG~file is quite similar to the 
% \cs{includegraphics} command provided by the \pkg{graphicx} package.
% \begin{svgexample}
%   \includesvg§\oarg{parameters}\marg{svg filename}§
% \end{svgexample}
%
% \begin{DescribeParameterParent}{Macro}{\includesvg}^^A
% \DescribeParameters{%
%   inkscape,inkscapeformat,inkscapelatex,inkscapearea,inkscapedpi,inkscapeopt,%
%   svgextension,width,height,distort,scale,pretex,apptex,draft,%
% }^^A
% It is used right in the same way but where \marg{svg filename} is the 
% file~name of the SVG~file, where any given file~extension will be replaced
% with \file{.svg} ruthlessly. In order to change the source file format for 
% the \app{Inkscape} export, you have to use parameter \prm{svgextension}.
%
% If the given file is not located in the current working directory but
% elsewhere on your file~system, the command \cs{svgpath} could be used to
% specify this path. It is recommended to avoid umlauts or any other non-ascii 
% characters as well as any spaces and/or quotes respectively \cs{dq} both in
% paths an file~names. Espacially when DVI~output is active using quotes will
% certainly cause an error. 
%
% The command \cs{includesvg} is intended to do an automated export with 
% \app{Inkscape} at first, where the given SVG~file is exported to a
% PDF/EPS/PS/PNG~file (see~\opt{inkscapeformat}) and perhaps a correlating
% \LaTeX~file (see~\opt{inkscapelatex}). The export with \app{Inkscape} is only
% invoked, if the SVG~file is newer than the exported graphic~file or latter 
% doesn't exist at all. Once the export has been done, the graphic~file and
% maybe the \LaTeX~file are included.
% 
% All previously described options can also be used as optional parameters to 
% \cs{includesvg} and do have the same effect as described before. However, the 
% optional parameters specified have an effect only once when \cs{includesvg}
% is executed and remain unchanged afterwards.
%
% \DescribeParameter{lastpage}^^A
% In addition to the use of boolean values, the parameter \prm{lastpage} can 
% also be assigned a specific (integer) page number, which defines the last
% used page of a PDF~graphic. This, just like the identically named option, has 
% an effect only when \prm{inkscapeformat}|=|\val{pdf} is set.
%
% \DescribeParameters{angle,origin}^^A
% Both parameters correlate to the identically named parameters of the  
% \cs{includegraphics} command provided by the \pkg{graphicx} package. However,
% unlike to \cs{includegraphics}, they \prm{angle} and \prm{origin} are
% \emph{always evaluated after} \prm{widht}, \prm{height}, \prm{distort} and
% \prm{scale} by \cs{includesvg}, regardless of the used order of the given
% parameters. This is mainly due to the inclusion of the \LaTeX~files
% corresponding to the graphic~files generated by \app{Inkscape}.
% \end{DescribeParameterParent}
%
%
%
% \subsection{Including already exported SVG~files}
%
% \DescribeMacro{\includeinkscape}^^A
% If you don't want to make use of the automated export with \app{Inkscape} but 
% the user interface provided by the \pkg{svg} package, you can use 
% \cs{includeinkscape} instead of \cs{includesvg}.  
% \begin{svgexample}
%   \includeinkscape§\oarg{parameters}\marg{graphic filename}§
% \end{svgexample}
%
% \begin{DescribeParameterParent}{Macro}{\includeinkscape}^^A
% \DescribeParameters{%
%   inkscapeformat,inkscapelatex,%
%   width,height,distort,scale,pretex,apptex,draft,lastpage,angle,origin,%
% }^^A
% You can use it similar to \cs{includesvg} but \marg{graphic filename} has to 
% be the filename of the already exported graphic file. If a valid 
% file~extension (\file{.pdf/.eps/.ps/.png}) is given, the current setting for 
% \opt{inkscapeformat} is overwritten. It's even possible to specify a 
% file~extension like \file{.pdf\_tex} to activate \opt{inkscapelatex}.
% Furthermore, all optional parameters for \cs{includeinkscape} do have the
% same effect as described before for command \cs{includesvg} once when
% \cs{includeinkscape} is executed and remain unchanged afterwards.
% \end{DescribeParameterParent}
%
%
%
% \section{Usage of package \pkg{svg-extract}}
%
% This package allows the extraction of independent graphic~files out of 
% SVG~files which have been included and rendered with \LaTeX{} by the 
% \pkg{svg} package. This is particularly useful when attempting to provide
% images to journals or collaborators, and one wishes the image to appear
% exactly as it does within the original \LaTeX~document.
%
% In order to extract to PDF, EPS, or PS~files the programs \file{pstoeps}, 
% \file{pstopdf} and \file{pdftops} are used which are usually provided by most 
% of the \LaTeXe~distributions. In additon, the command line tools of 
% \app{ImageMagick} and \app{Ghostscript} can be invoked for converting images 
% in formats like PNG, JPG, TIF or something else. It's also possible to create 
% PDF, EPS or PS~files with one of the two programs. Therefor the desired 
% program---\file{magick} and/or \file{gswin32c}/\file{gswin64c} on Windows 
% respectively \file{convert} and/or \file{gs} on unix-like operating 
% systems---must be installed. By typing \file{\meta{program} -{}-version} on 
% shell, this can be checked.
%
% If you want to extract independent graphic~files from included SVG~files, you 
% only have to load \pkg{svg-extract}. All actions for the extraction process 
% will be done by using \cs{includesvg} or \cs{includeinkscape}. Without any 
% additional settings, the extraction will render the SVG~file to the specified 
% output formats(s) of choice using the same settings as specified within the 
% two commands. Consequently, the scale between the image and text in the
% extracted files will remain identical to the scale within the document from 
% which the SVG~file was extracted.
%
% In contrast to package \pkg{svg}, the console commands for graphic extraction 
% are executed with each LaTeX run by package \pkg{svg-extract} when 
% \file{-{}-shell-escape} mode is activated. This behaviour can be switched of 
% with option \opt{extract}|=|\val{false}.
%
% \minisec{Important changes}
%
% In version~v1.0 of package \pkg{svg} the extracted files were named like the 
% numbering of the current \env{subfig} environment by default. As package 
% \pkg{subfig} sometime causes problems and because of the large amount of
% different \LaTeX~packages which all provide the possibility to include
% subfigures with very different implemetations, this feature can't be provided
% reliably by \pkg{svg-extract}. See option \opt{extractname} for further 
% information.
%
%
%
% \subsection{General settings}
%
% \DescribeOptions{on,auto,off}^^A
% This options have to be given while loading the \pkg{svg-extract} package and 
% are intended to toggle the functionality of this package. As both extracting 
% and converting independent graphic~files needs some time due to several calls 
% of external programs, the option \opt{off} can be given to save compilation
% time, once the creation of all desired images has been done and they don't
% have to be re-generated. This is identical to \opt{extract}|=|\val{false}. 
% With option \opt{auto} only those graphic files are extracted, where the 
% corresponding inkscape export file has changed. This is the same as using 
% \opt{extract}|=|\val{true}. To extract graphic files in any case, the option
% \opt{on} can be used. The same behaviour can also be achieved by using 
% \opt{extract}|=|\val{forced}, which is the default setting. For detailed 
% information, see option \opt{extract}.
%
% \DescribeMacros{\svgsetup,\includesvg,\includeinkscape}^^A
% With package \pkg{svg-extract} the applicable options for 
% \cs{svgsetup}\marg{options} as well as parameters for the already described
% macros \cs{includesvg}\oarg{parameters}\marg{filename} and 
% \cs{includeinkscape}\oarg{parameters}\marg{filename} are extended. They can
% be used to control the process of graphic extraction and converting.
%
% All option described below can be used togehter with \cs{svgsetup} and are
% then valid in the current scope. There also exist identically named 
% parameters for the optional arguments of 
% \begin{svgexample}
%   \includesvg§\oarg{parameters}\marg{svg filename}§
%   \includeinkscape§\oarg{parameters}\marg{graphic filename}§
% \end{svgexample}
% They only have an effect once when the specific command is executed.
%^^A \begingroup\ignorespaces%
%^^A \newcommand*\parameterlist{}\ignorespaces%
%^^A \listadd\parameterlist{extract}\ignorespaces%
%^^A \listadd\parameterlist{extractpreamble}\ignorespaces%
%^^A \listadd\parameterlist{extractformat}\ignorespaces%
%^^A \listadd\parameterlist{extractwidth}\ignorespaces%
%^^A \listadd\parameterlist{extractheight}\ignorespaces%
%^^A \listadd\parameterlist{extractdistort}\ignorespaces%
%^^A \listadd\parameterlist{extractscale}\ignorespaces%
%^^A \listadd\parameterlist{extractangle}\ignorespaces%
%^^A \listadd\parameterlist{extractpretex}\ignorespaces%
%^^A \listadd\parameterlist{extractapptex}\ignorespaces%
%^^A \listadd\parameterlist{extractruns}\ignorespaces%
%^^A \listadd\parameterlist{latexexe}\ignorespaces%
%^^A \listadd\parameterlist{latexopt}\ignorespaces%
%^^A \listadd\parameterlist{latexext}\ignorespaces%
%^^A \listadd\parameterlist{dvipsopt}\ignorespaces%
%^^A \listadd\parameterlist{pstoepsopt}\ignorespaces%
%^^A \listadd\parameterlist{pstopdfopt}\ignorespaces%
%^^A \listadd\parameterlist{pdftoepsopt}\ignorespaces%
%^^A \listadd\parameterlist{pdftopsopt}\ignorespaces%
%^^A \listadd\parameterlist{convert}\ignorespaces%
%^^A \listadd\parameterlist{convertformat}\ignorespaces%
%^^A \listadd\parameterlist{convertdpi}\ignorespaces%
%^^A \listadd\parameterlist{magicksetting}\ignorespaces%
%^^A \listadd\parameterlist{magickoperator}\ignorespaces%
%^^A \listadd\parameterlist{gsopt}\ignorespaces%
%^^A \listadd\parameterlist{gsdevice}\ignorespaces%
%^^A \listadd\parameterlist{clean}\ignorespaces%
%^^A \listadd\parameterlist{exclude}\ignorespaces%
%^^A \def\buffer{}\def\do#1{\buffer\prm{#1}\def\buffer{, }}\ignorespaces%
%^^A These parameters are: \dolistloop{\parameterlist}.
%^^A \renewcommand*\marginpar[2][]{}\ignorespaces%
%^^A \def\do#1{\DescribeParameter{#1}}\ignorespaces%
%^^A \begin{DescribeParameterParent}{Macro}{\includesvg}%
%^^A \dolistloop{\parameterlist}%
%^^A \end{DescribeParameterParent}%
%^^A \begin{DescribeParameterParent}{Macro}{\includeinkscape}%
%^^A \dolistloop{\parameterlist}%
%^^A \end{DescribeParameterParent}%
%^^A \endgroup%
%
%
%
% \subsection{Extract independent grahic files}\label{sec:extract}
% \DescribeOption{extract}^^A
% \changes{v2.03}{2019/07/14}{%
%   option \opt{extract} has new value \val{forced}%
% }^^A
% This option controls the functionality for both extracting and---if option 
% \opt{convert} is active---converting and is \default{forced} by default.
% \begin{DescribeValues}^^A
% \item[false/off/no]
%   Both extracting and converting is deactivated in any case.
% \item[true/on/yes/newer/onlynewer]
%   The extraction/conversion will be done, if the independent graphic~file
%   either does not exist or the file modification date of the graphic~file of
%   the \app{Inkscape}~export is newer than that of the extracted/converted
%   independent graphic~file. Thus the compilation time of the \LaTeX~document
%   can be reduced to the necessary minimum. Keep in mind, that only changing 
%   specific export options or parameters like \opt{extractwidth}, 
%   \opt{extractheight}, \opt{extractdistort}, \opt{extractscale},
%   \opt{extractpretex}, \opt{extractapptex} as well as  \opt{extractpreamble}
%    or \opt{extractpreambleend} will not cause to automatically run a new 
%    extraction/conversion. In order to get reliable extracted files, you 
%    should use \opt{extract}|=|\val{forced} at least once after changing one 
%    or more of these .
% \item[\default{forced}/force/overwrite]
%   All independent graphic files are created, any already existing file will 
%   overwritten regardlessly.
% \end{DescribeValues}
%
% \DescribeOption{extractpath}^^A
% \changes{v2.03}{2019/07/11}{%
%   option \opt{extractpath} provides values \val{inkscapepath} and 
%   \val{inkscapesubpath} for creating extracted/converted graphics in the same 
%   folder as the resulting files of the \app{Inkscape}~export%
% }^^A
% The path where the extracted and converted files are located can be specified 
% with option \opt{extractpath}, whereas \default{basesubdir} is set by default.
% \begin{DescribeValues}^^A
% \item[svgdir/svgpath]
%   The extracted and converted independent graphic~files are located in the 
%   same directory as the corresponding SVG~file.
% \item[svgsubdir/svgsubpath]
%   Within the folder of the encountered SVG~file, all extracted and converted
%   files will be located in a subfolder named \file{svg-extract/}.
% \item[basedir/basepath/jobdir/jobpath]
%   All extracted and converted files will be located in the current working 
%   directory.
% \item[\default{basesubdir}/basesubpath/jobsubdir/jobsubpath]
%   A subfolder named \file{svg-extract/} within the current working directory 
%   will be used for all extracted and converted files.
% \item[inkscapepath/inkscapedir]
%   The extracted/converted graphic~files are stored in the same folder as the
%   resulting files of the \app{Inkscape}~export (option \opt{inkscapepath}).
% \item[inkscapesubpath/inkscapesubdir]
%   An additional subfolder within the \app{Inkscape}~export directory named 
%  \file{svg-extract/} will be used for extracted/converted files.
% \item[/path/to/somewhere/]
%   It is also possible to give a custom path, either relative to the current 
%   working directory (\file{./relative/path/}) or as an absolute path.
% \end{DescribeValues}
%
% \DescribeOption{extractname}^^A
% It's also possible to change the name for extracted and converted files. The 
% default setting is \opt{extractname}|=|\val{filenamenumbered}. The appended 
% file extension is derived from option \opt{extractformat}. 
% \begin{DescribeValues}^^A
% \item[filename/name]
%   The name of the exported \app{Inkscape}~file is used and the suffix
%   \file{-extract} is attached.
% \item[\default{filenamenumbered}/namenumbered/numberedfilename/numberedname]
%   Same as above, but a prefix with the current enumerated count of SVG~files
%   is used instead of the suffix.
% \item[numbered/section/numberedsection/sectionnumbered]
%   The file name is composed by the current enumerated countof SVG~files and 
%   the present outline numbering.
% \item[\file{\meta{filename}}]
%   You can use any file name. It's possible to use counters for specifying the 
%   name of the extracted file. Repeatedly specifying the same file name will
%   overwrite previously created files. 
% \end{DescribeValues}
%
% \DescribeOption{extractformat}^^A
% The included SVG~file can be extracted from the document into an independent
% graphic~file of type PDF, EPS or PS. The option can be used with either a
% single value (\opt{extractformat}|=|\val{pdf}) or a comma separated list. 
% For example,
% \begin{svgexample}
%   \includesvg[extractformat={pdf,eps,ps}]§\marg{svg filename}§
% \end{svgexample}
% will extract the SVG~file to both PDF and EPS~formats and generates two 
% independent graphic~files. By default, \opt{extractformat}|=|\val{pdf} is set 
% unless DVI~output was detected. In this case \opt{extractformat}|=|\val{eps} 
% is the default setting.
%
% \DescribeOptions{%
%   extractwidth,extractheight,extractdistort,extractscale,
%   extractpretex,extractapptex,%
% }^^A
% \changes{v2.02}{2018/09/07}{%
%   option \opt{extractdistort} added for distortion of extracted graphics%
% }^^A
% These options can be used to overwrite the settings given for the appearance 
% of a SVG~file within the document. For example, a SVG~file should cover the 
% entire text width within the document but be extracted to a fixed width. This 
% can be done with:
% \begin{svgexample}
%   \includesvg[width=\textwidth,extractwidth=500pt]§\marg{svg filename}§
% \end{svgexample}
% Assigning the value \val{inherit} to one of these options---which is set by 
% default---leads to the usage of the corresponding option of package \pkg{svg} 
% (\opt{width}/\opt{height}/\opt{scale}/\opt{pretex}/\opt{apptex}), whereas
% \opt{extract\dots}|=|\val{\cs{relax}} can be used to ignore a parent option
% utterly.
%
% \DescribeOptions{extractpreamble,extractpreambleend}^^A
% Within the included and extracted SVG~files any \LaTeX~macro can be used 
% either defined by the user---this should be done in the preamble of the 
% \LaTeX~document in which the SVG~file is to be included---or provided by a 
% package which is loaded. As the extraction process of the SVG~files needs an 
% auxiliary \LaTeX~file all used packages and commands have to be known within 
% this file. Consequently, the preamble of the current \LaTeX~document is used 
% for the extraction of the SVG~file by default.
% 
% However, it is possible to specify a different \emph{preamble~file} with the 
% option \opt{extractpreamble} where the file to use as the preamble is given
% as the argument---including maybe path, but file~name and file~extension in 
% any case. The given preamble file is searched similar to SVG~files meaning, 
% every path given with \cs{svgpath} or \cs{graphicspath} is examined. The 
% default definition of \opt{extractpreamble} is \file{\cs{jobname}.tex}---more 
% precisely the file~extension given by option \opt{latexext} is used---and
% should suffice for most cases. The preamble up to the line defined by the 
% option \opt{extractpreambleend} will be used, which is set to a default with
% \cs{begin}|{document}|. 
%
% \DescribeMacros{\svghidepreamblestart,\svghidepreambleend}^^A
% In case, the preamble of the current \LaTeX~document is used, there are maybe 
% packages included or some parts within the preamble, which should not be used
% within the separate auxiliary \LaTeX~file. These parts can be excluded if
% they are enclosed by \cs{svghidepreamblestart} and \cs{svghidepreambleend}.
%
% For example, your current \LaTeX~document uses package \pkg{showframe} which 
% causes some problems with the extraction of independent graphic~files. So you
% want to get rid of it within the auxiliary \LaTeX~file. This can be done with:
% \begin{svgexample}
%   \documentclass§\marg{documentclassname}§
%   §\dots§
%   \usepackage{svg-extract}
%   §\dots§
%   \svghidepreamblestart
%   \usepackage{showframe}
%   \svghidepreambleend
%   §\dots§
% \end{svgexample}
%
% \DescribeOption{extractruns}^^A
% When extracting independent grahic~files by compiling the generated auxiliary 
% \LaTeX~file, it's maybe necessary to do multiple \LaTeX~iterations on this. 
% The number of iterations is controlled with option \opt{extractruns}. It's
% set to \opt{extractruns}|=|\val{2} by default.
%
% \DescribeOptions{latexexe,latexopt,latexext}^^A
% For the extraction of an independent grahic~file, the \LaTeX~program is used
% which is set by the \opt{latexexe} option. Depending on the \LaTeX~processor 
% used for the current \LaTeX~document, it is set to either \app{pdflatex}, 
% \app{lualatex}, \app{xelatex} or \app{latex} by default. It's also possible
% to specify additional flags or switches for the \LaTeX~iterations, which are
% performed during the extraction process by the \opt{latexopt} option. If you
% are used to utilize a different extension for \LaTeX~files than \file{.tex},
% option \opt{latexext} can be used like \opt{latexext}|=|\val{ltx}.
%
% \DescribeOptions{dvipsopt,pstoepsopt,pstopdfopt,pdftoepsopt,pdftopsopt}^^A
% Depending on the used \LaTeX~processor, the file~type of the extracted 
% graphic differs. In order to create all formats, requested with option
% \opt{extractformat}, several converting tools provided by most of the 
% \LaTeXe~distributions are maybe invoked. These are \file{dvips},
% \file{ps2eps}, \file{ps2pdf} and/or \file{pdftops} and can't be changed. It's
% only possible to specify additional switches for every single tool with
% \opt{dvipsopt}, \opt{pstoepsopt}, \opt{pstopdfopt}, \opt{pdftoepsopt} and
% \opt{pdftopsopt}.
%
% \DescribeOption{clean}^^A
% \changes{v2.03}{2019/07/11}{%
%   option \opt{clean} provides value \val{nolatex} for preserving auxiliary 
%   \LaTeX~files used for extraction to be deleted%
% }^^A
% During the extraction process many files are generated for each SVG~file 
% extraction. So it's oftentimes desirable to automatically remove these 
% temporary files. Using the option \opt{clean}|=|\val{true} will remove any 
% generated files created other than the extracted output format(s) requested.
% If the generated auxiliary \LaTeX~file, which is used to extract the desired 
% graphics, should be preserved during deletion, \opt{clean}|=|\val{nolatex}
% can be used. Setting \opt{clean}|=|\val{false} is useful for debugging and
% set by default.
% Additionally, it's possible to use option \opt{clean} with a list of 
% file~extensions in order to specify auxiliary files generated by package 
% \pkg{svg-extract} to be deleted, for example \opt{clean}|=|\val{\{log,aux\}}.
%
% \DescribeOption{exclude}^^A
% Sometimes it may be necessary to extract and/or convert a SVG~file without 
% including it. If the flag \opt{exclude} is specified, the SVG~file will not 
% be rendered in the current \LaTeX~document, but will be extracted and/or
% converted to the requested output format(s).
%
% \DescribeMacros{\includesvg,\includeinkscape}^^A
% As previously mentioned, for extracting independent graphic files it is
% sufficient to load package \pkg{svg-extract} and afterwards everything 
% necassary is done by just using \cs{includesvg} or \cs{includeinkscape}.
%
% \begin{DescribeParameterParent}{Macro}{\includesvg}^^A
% \DescribeParameter{extractangle}^^A
% \begin{DescribeParameterParent}{Macro}{\includeinkscape}^^A
% \renewcommand*\marginpar[2][]{}^^A
% \DescribeParameter{extractangle}^^A
% \changes{v2.02}{2018/09/07}{%
%   parameter \prm{extractangle} for \cs{includesvg} and \cs{includeinkscape}
%   implemented in order to rotate graphics during extractions%
% }^^A
% With this additional parameter the graphic is rotated during the extraction
% process. The value is not inherited from \prm{angle} if it was given by
% default. This can be achieved by setting:
% \begin{svgexample}
%   \includesvg[angle=§\meta{angle}§,extractangle=inherit]§\marg{filename}§
% \end{svgexample}
% \end{DescribeParameterParent}
% \end{DescribeParameterParent}
%
%
%
% \subsection{Convert extracted grahic files}\label{sec:convert}
% 
% Based on the extraction of independent graphic~files, the \pkg{svg-extract} 
% packages also provides the possibility to convert these extracted graphics in 
% another format than PDF, EPS or PS with either \app{ImageMagick}---which is 
% set by default---or \app{Ghostscript}.
%
% \DescribeOption{convert}^^A
% \changes{v2.03}{2019/07/14}{%
%   option \opt{convert} has new value \val{forced}%
% }^^A
% This option can be used to control the invocation of the conversion process. 
% By default, \opt{convert}|=|\val{false} is set. For Windows, there exist two 
% different versions of \app{Ghostscript}, either 64\,bit or 32\,bit. If 
% it is selected as converting tool the 64\,bit~executable is set by default.
% Please not, that option \opt{extract} has to be activated.
% \begin{DescribeValues}^^A
% \item[\default{false}/off/no]
%   No conversion is done.
% \item[true/on/yes/newer/onlynewer]
%   The conversion will be done with the current chosen converting tool, in the 
%   same manner as described for \opt{extract}|=|\val{true}.
% \item[forced/force/overwrite]
%   The conversion will be done in any case with the current chosen converting
%   tool. Any already existing file will overwritten.
%   regardlessly.
% \item[magick/imagemagick/convert]
%   The conversion is activated and \app{ImageMagick} is selected.
% \item[gs/ghostscript]
%   The conversion is activated and \app{Ghostscript} is selected.
% \item[gs64/ghostscript64]
%   This value activates \app{Ghostscript} as conversion tool and sets
%   \opt{gsexe}|=|\val{gswin64c}. On unix-like operating systems, the value for
%   \opt{gsexe} remains unchanged.
% \item[gs32/ghostscript32]
%   The same as for the latter case applies, only option
%  \opt{gsexe}|=|\val{gswin32c} is set on Windows.
% \end{DescribeValues}
% 
% \DescribeOption{convertformat}^^A
% With this option, the desired output format(s) can be given. Multiple graphic 
% formats can be specified in a list, for example something like 
% \opt{convertformat}|=|\val{\{png,jpg,tif\}}. The value specified in 
% \opt{extractformat} is used as the source format for the conversion. If 
% \opt{extractformat} itself contains a file~list, the first value within this 
% list is considered. If \opt{extractformat} is defined empty, the file
% generated anyway during the extraction is used.
%
% \minisec{Settings for specific converting formats}
%
% Maybe it's desired to apply varying settings for different output formats. 
% Therefor some options described below can either be set for all converted 
% files or for a specific output format. In particular, these are the options
% \opt{convertdpi} as well as \opt{magicksetting}, \opt{magickoperator}, 
% \opt{gsdevice} and \opt{gsopt}. All these mentioned options can be used like 
% either \opt{\meta{option}}|=|\val{\meta{value}} or 
% \opt{\meta{option}}|=|\val{\{\meta{outputformat}=\meta{value}\}} and even
% \opt{\meta{option}}|=|\val{\{\meta{outputformat}+=\meta{value}\}} where the 
% desired output~format is trailed with |+| as inner key.
%
% The first variant is applied to all output formats in general. If one of
% these mentioned options is evaluated and an output format specific value was
% given like in the second variant, the general setting is overwritten. If the 
% general setting should be used and extended by an additional output format
% specific settings, then the third variant is to be used. In this case, no 
% output format specific setting (second variant) must not have been used.
%
% If you want to reverse any setting, you only have to use \cs{relax} as a 
% value, either for a general option (\opt{\meta{option}}|=|\val{\cs{relax}})
% or a specific one 
% (\opt{\meta{option}}|=|\val{\{\meta{outputformat}[+]=\cs{relax}\}}).
%
% \DescribeOption{convertdpi}^^A
% This option controls the used density for all file~formats or a specific one,
% whether \app{ImageMagick} or \app{Ghostscript} is used for the graphic 
% conversion. The desired resolution of the converted file is given in dots per
% inch~(DPI) either as a scalar value (e.g.~\opt{convertdpi}|=|\val{600}) or 
% with different resolutions in x- and y\hbox{-}direction 
% (e.g.~\opt{convertdpi}|=|\val{600x400}).
%
% As described before, it's also possible to declare a specific resolution for
% each desired converting format. For example, you want to set different
% resolution for PNG and JPG~formats and something for all other formats:
% \begin{svgexample}
%   \svgsetup{%
%     convertdpi={png=600},%
%     convertdpi={jpg=150},%
%     convertdpi=300%
%   }%
% \end{svgexample}
% If a setting for a specific output format is given, any unspecific setting is 
% overwritten, when the conversion to this format is done. With 
% \opt{convertdpi}|=|\val{\{\meta{outputformat}=\cs{relax}\}} a specific 
% setting can be reversed.
%
% Please note that not every graphic format support different resolutions in x- 
% and y\hbox{-}direction. So using a value like \opt{convertdpi}|=|\val{600x400}
% may not necessarily lead to the desired result. However, this is then due to
% the used conversion tool and not to the processing of the option.
%
%
% \subsubsection{Settings for the invocation of \app{ImageMagick}}
%
% \DescribeOptions{magickexe,magicksetting,magickoperator}^^A
% The conversion with \app{ImageMagick} via the \file{magick} or \file{convert} 
% command-line~tool can be controlled with these options. The option
% \opt{magickexe} determines the used executable and is set to \file{magick} on
% Windows and otherwise to \file{convert} by default. Additionally, there are
% the two options \opt{magicksetting} and \opt{magickoperator} which can be
% used to define \emph{settings} and \emph{operators} for the conversion
% process. As described before, the two options \opt{magicksetting} and 
% \opt{magickoperator} can be set for all output formats or a \emph{specific}
% one either resetting or extending the general settings. For further 
% information see the documentation of 
% \hrfn{http://www.imagemagick.org/script/command-line-processing.php}%
% {\app{ImageMagick} command-line tool}.
%
%
% \subsubsection{Settings for the invocation of \app{Ghostscript}}
%
% \DescribeOptions{gsexe,gsdevice,gsopt}^^A
% The conversion with \app{Ghostscript} is done with command-line~tool \file{gs}
% on unix-like operating systems and \file{gswin64c} or \file{gswin32c} on 
% Windows. The executable can be changed with option \opt{gsexe}. Because 
% \app{Ghostscript} requires the specification of a device, there are some 
% predefined for the most common output formats. These are:
% \begin{svgexample}
%   \svgsetup{%
%     gsdevice={png=png16m},gsdevice={jpeg=jpeg},gsdevice={jpg=jpeg},%
%     gsdevice={tif=tiff48nc},gsdevice={tiff=tiff48nc},%
%     gsdevice={eps=eps2write},gsdevice={ps=ps2write}%
%   }%
% \end{svgexample}
% Furthermore, with \opt{gsopt} additional switches for \app{Ghostscript} can 
% be set. As described before, both \opt{gsdevice} and \opt{gsopt} can be 
% defined in general or for specific output formats. For further information
% see the documentation of 
% \hrfn{https://ghostscript.com/doc/current/Use.htm}{\app{Ghostscript}}.
%
%
%
% \cleardoublepage
% \section{Example}
%
% As an minimal example\footnote{\ignorespaces%
%   The image used here is a slightly modified version of the image used in the
%   initial documentation on how to include a SVG~file in \LaTeX{} by 
%   Johan\,B.\,C.~Engelen available as package 
%   \href{http://www.ctan.org/pkg/svg-inkscape}{\pkg{svg-inkscape} on CTAN}.
% } take the following lines of code:
% \begin{svgexample}
%   \documentclass{article}
%   \usepackage[T1]{fontenc}
%   \usepackage{svg}
%   \usepackage[off]{svg-extract}
%   \svgsetup{clean=true}
%   %\pdfsuppresswarningpagegroup=1
%   \usepackage{relsize}
%   \usepackage{subcaption}
%   \begin{document}
%   \begin{figure}
%     \begin{minipage}{.5\linewidth}
%       \includesvg[width=\linewidth]{svg-example}%
%       \subcaption{This text is too large!}
%     \end{minipage}%
%     \begin{minipage}{.5\linewidth}
%       \includesvg[width=\linewidth,pretex=\relscale{0.6}]{svg-example}%
%       \subcaption{This text fits better.}
%     \end{minipage}
%   \caption{An example figure with \LaTeX~support}\label{fig:example}
%   \end{figure}
%   \begin{figure}\centering
%     \includesvg[%
%       width=.5\linewidth,inkscapelatex=false,extractformat={pdf,eps}%
%     ]{svg-example}%
%     \caption{The same example figure without \LaTeX~support}
%   \end{figure}
%   \end{document}
% \end{svgexample}
% \begin{figure}[tb]
%   \null\hfill\ignorespaces%
%   \begin{minipage}{.45\linewidth}
%     \includesvg[width=\linewidth]{svg-example}%
%     \subcaption{This text is too large!}\label{fig:example-tex-a}
%   \end{minipage}\hfill\ignorespaces%
%   \begin{minipage}{.45\linewidth}
%     \includesvg[width=\linewidth,pretex=\relscale{0.6}]{svg-example}%
%     \subcaption{This text fits better.}\label{fig:example-tex-b}
%   \end{minipage}
%   \hfill\null\ignorespaces%
%   \caption{An example figure with \LaTeX~support}\label{fig:example-tex}
% \end{figure}
% \begin{figure}[tb]\centering
%   \includesvg[%
%     width=.5\linewidth,inkscapelatex=false,extractformat={pdf,eps}%
%   ]{svg-example}%
%   \caption{The same example figure without \LaTeX~support}\label{fig:example}
% \end{figure}
%
% The output is shown in \autoref{fig:example-tex} and \autoref{fig:example}. 
% Within this example the file \file{svg-example.svg} was included three times 
% using the \cs{includesvg} command.
% If you are willing to compile the example, there are two aspects to consider. 
% First, the included SVG~file \file{svg-example.svg} has to be located in the 
% current folder and is located in \file{\meta{texmf}/doc/latex/svg/}. 
% Second, you have to run the desired \LaTeX~engine with 
% \file{-{}-shell-escape} option enabled. 
% 
% As you can see, \autoref{fig:example-tex-a} is created with default settings, 
% except the width specification. The \app{Inkscape}~export with \LaTeX~support
% is done and the extraction of an independent graphic~file in PDF~format as
% the \pkg{svg-extract} package was loaded. 
%
% However, the text is slightly overrunning the margins of the image, and so 
% \autoref{fig:example-tex-b}---which again uses the same \app{Inkscape}~export 
% results---decreases the font size of the text within the image relative 
% using the \prm{pretex} option together with the \cs{relscale} command 
% provided by the \pkg{relsize} package. 
%
% In \autoref{fig:example} the same SVG~file was used but without the export 
% of a separate \LaTeX~file containing all text elements. 
%
% Feel free to use this given example to try out all the options and 
% possibilities described in \autoref{sec:inkscape} for package \pkg{svg}.
% Especially if you want to use package \pkg{svg-extract} for the automated 
% extraction of independent graphics (\autoref{sec:extract}) and their 
% conversion to different graphic formats with \app{ImageMagick} and/or 
% \app{Ghostscript} (\autoref{sec:convert}), this example can be easily used 
% for the first steps.
%
%
%
% \section{Troubleshooting and reporting issues}
%
% When using the packages \pkg{svg} and \pkg{svg-extract}, the most likely 
% occurring problems will be caused by calling the external programs. For this 
% reason, a short package information is written into the log~file right before 
% each call of an external program via shell. If a file should have been 
% created, both packages check after the external call, whether this file
% exists or not and raise an error or at least a warning, if this file is 
% missing. If you got such a message, please check the log~file for lines like:
% \begin{quote}
% |Package svg Info:| or |Package svg-extract Info:|
% \end{quote}
% Right afterwards, there should appear |runsystem(<command>)...excuted.| which 
% you should try to execute manually from shell in the right directory. In most
% cases, the problem will be an invalid command call. If something goes wrong
% during the extraction/converting process of package \pkg{svg-extract}, it
% would make sense to set option \opt{clean}|=|\val{false} to not delete any 
% auxiliary files that might be needed.
%
% If you are sure that the problem is not caused by the configuration of your
% operating system, you can send an error report either via email or create a
% new issue on GitHub. Both addresses can be found on the title.
%
% \minisec{When using pdf\LaTeX{} there are a lot of warnings}
%
% It may happen that several warnings like
% \begin{quote}\ttfamily\raggedright%
% pdfTeX warning:\,pdflatex.exe(file \meta{filename}.pdf):\,PDF inclusion:
% multiple pdfs with page group included in a single page
% \end{quote}
% occur when including the PDF~graphics exported with \app{Inkscape}. This is 
% related to the handling of transparency effects within PDF~files. Since
% pdf\TeX~version 1.40.15 or later, you can get rid of these messages by using
% \cs{pdfsuppresswarningpagegroup}|=|\val{1}. See also the discussion on 
% \hrfn{http://tex.stackexchange.com/questions/76273/}{LaTeX Stack Exchange} 
% for more information.
%
%
%
% \section{Include SVG files created with \app{ROOT}}
%
% This section was originally written by Philip Ilten. In the hope that since 
% then nothing has changed fundamentally in the described procedure. This 
% passage remains in the documentation, even if it will almost certainly be 
% relevant to experimental particle physicists only, who frequently use the
% analysis package \app{ROOT}.
%
% \app{ROOT} has the ability to export directly to a SVG~file, which means that 
% it is possible to completely by-pass all of \app{ROOT}'s internal text
% rendering machinery, and let \LaTeX{} handle the text natively. This means
% that all of the ugly fonts that are rendered by \app{ROOT} can now be
% completely avoided, with the additional bonus of being able to add references
% within plots. So how does one go about using this package with \app{ROOT}?
% \begin{enumerate}
% \item
%   Create the plot with \app{ROOT} as normal, but turn off all 
%   \LaTeX~interpretation of text strings. This is a bit tricky, but can be
%   accomplished by setting the font in \app{ROOT} to a precision of zero as
%   described in the documentation for
%   \hrfn{http://root.cern.ch/root/html/TAttText.html}{\texttt{TAttFill}}. 
%   Remember that the font is set by using the function
%   |(TAttFill*)->SetTextFont(i)| with
%   \[\texttt{i}=\textrm{(font type)} \times 10 + \textrm{(font precision)}\]
%   In the following lines of code, a |TStyle| is defined which sets the font
%   to type ``Courier New'' with a precision of zero.
%   \begin{svgexample}[backgroundcolor=\color{red!10}]
%     TStyle *style = new TStyle("style","style"); int FONT = 80;
%     style->SetTextFont(FONT);
%     style->SetLabelFont(FONT,"XYZ");
%     style->SetTitleFont(FONT,"XYZ");
%     style->SetTitleFont(FONT,"");
%     gROOT->SetStyle("style");
%     gROOT->ForceStyle();
%   \end{svgexample}
%   Now, you can just use the well\hbox{-}known standard \LaTeX~syntax for 
%   creating labels, etc. Note however, that backslashes have to be escaped due
%   to interpretation of special characters by \app{C++}.
% \item Print the plot as a SVG~file.
%   \begin{svgexample}[backgroundcolor=\color{red!10}]
%     gPad->Print("foo.svg");
%   \end{svgexample}
% \item Include the SVG~file within the document using this package.
%   \begin{svgexample}
%     \usepackage{svg}
%     \usepackage{svg-extract}
%     \svgsetup{clean=true}
%     §\dots§
%     \includesvg[width=\linewidth]{foo}
%   \end{svgexample}
% \end{enumerate}
%
% Consider the following example image produced by \app{ROOT} in 
% \autoref{fig:root}. This figure was generated by the \app{ROOT} macro
% \file{root.C}, provided within \file{\meta{texmf}/doc/latex/svg/},
% which produces the file~\file{root.svg} when run. The code used to produce
% this SVG~file from within \app{ROOT} is
% \lstinputlisting[style=svg,backgroundcolor=\color{red!10}]{examples/root.C} 
% where the text produced within the \app{ROOT} plot is set to a precision of
% zero. 
%
% The plot was then included within this document using the following 
% \LaTeX~code
% \begin{svgexample}
%   \begin{figure}
%     \centering%
%     \includesvg[%
%       inkscapearea=page,height=6cm,pretex=\tiny,convertformat=png%
%     ]{root}%
%     \caption{Rendering of a \app{ROOT} plot---no more \emph{Comic CERNs}}%
%     \label{fig:root}%
%   \end{figure}
% \end{svgexample}
% \begin{figure}
%   \centering%
%     \includesvg[%
%       inkscapearea=page,height=6cm,pretex=\tiny,convertformat=png%
%     ]{root}%
%   \caption{Rendering of a \app{ROOT} plot---no more \emph{Comic CERNs}}%
%   \label{fig:root}%
% \end{figure}
% which includes the graphic as well as the \LaTeX~file exported by 
% \app{Inkscape}, produces the extracted PDF~image~(\file{root.pdf}) and 
% converts this to a PNG~image~(\file{root.png}) by using \app{ImageMagick}.
% Enjoy plots from \app{ROOT} with natively rendered \LaTeX{}!
%
% \StopEventually{%
%   \cleardoublepage\PrintIndex%
%   \cleardoublepage\PrintChanges%
%   \cleardoublepage\PrintToDos%
% }
%
% \iffalse
%<*package&option>
% \fi
%
%
%
% \cleardoublepage
% \appendix
% \part{Implementation}
% \section{Initialization}
% \subsection{Packages}
%
% The package \pkg{svg} requires \pkg{scrbase} for options processing, the
% packages \pkg{ifluatex}, \pkg{ifpdf} and \pkg{ifxetex} for detecting the 
% used \LaTeX~engine, \pkg{etexcmds} $\varepsilon$-\TeX{} primitives and 
% \pkg{pdftexcmds} for pdf\TeX{} primitives independent of the used 
% \LaTeX~engine, \pkg{shellesc} and \pkg{ifplatform} for engine independent
% access to systems commands and files as well as \pkg{graphicx} for the
% inclusion of PDF~files. The usage of the two packages \pkg{xcolor} and 
% \pkg{transparent} can be switched of with the corresponding options. 
% The package \pkg{svg-extract} only needs package \pkg{svg} itself.
%    \begin{macrocode}
%<*base>
\RequirePackage{scrbase}[2016/06/14]
\RequirePackage{ifpdf}[2016/05/14]
\RequirePackage{ifluatex}[2016/05/16]
\RequirePackage{ifxetex}[2010/09/12]
\RequirePackage{etexcmds}[2016/05/16]
\RequirePackage{pdftexcmds}[2016/05/21]
\RequirePackage{shellesc}[2016/06/07]
\RequirePackage{trimspaces}[2009/09/17]
\RequirePackage{graphicx}[1999/02/16]
%</base>
%<*extract>
\RequirePackage{svg}[2017/03/27]
%</extract>
%    \end{macrocode}
%
% \iffalse
%<*base>
% \fi
%
% \subsection{Dealing with catcodes}
%
% The catcodes for |"| and |_| have to be of type \val{other}. They current 
% setting are saved and restored at the end of the package.
% \begin{macro}{\svg@catcodes@restore}
% \changes{v2.03}{2019/08/30}{new}^^A
% \begin{macro}{\svg@catcodes@set}
% \changes{v2.03}{2019/08/30}{new}^^A
% \begin{macro}{\svg@@catcodes@set}
% \changes{v2.03}{2019/08/30}{new}^^A
% All catcodes of relevant special characters are stored and adapted with
% \cs{svg@catcodes@set}. Therefore, \cs{svg@@catcodes@set} is called with a 
% specific special character. First, it's definition is expanded to
% \cs{svg@catcodes@restore} and then set to \val{other}. So it is possible
% to call \cs{svg@catcodes@restore} and \cs{svg@catcodes@set} multiple times
% within the package to swap category codes as needed.
%    \begin{macrocode}
\newcommand*\svg@catcodes@restore{}
\newcommand*\svg@catcodes@set{%
  \ifx\svg@catcodes@restore\@empty%
    \svg@@catcodes@set\"%
    \svg@@catcodes@set\_%
    \l@addto@macro\svg@catcodes@restore{\let\svg@catcodes@restore\@empty}%
  \fi%
}
\newcommand*\svg@@catcodes@set[1]{%
  \edef\svg@catcodes@restore{%
    \etex@unexpanded\expandafter{\svg@catcodes@restore}%
    \catcode`\noexpand#1\the\catcode`#1\relax%
  }%
  \@makeother#1%
}
%    \end{macrocode}
% \end{macro}^^A \svg@@catcodes@set
% \end{macro}^^A \svg@catcodes@set
% \end{macro}^^A \svg@catcodes@restore
% \begin{macro}{\svg@RequirePackage}
% \changes{v2.03}{2019/08/30}{new}^^A
% In order to load other packages within these two and not interferring the 
% catcodes, this macro is defined, where catcodes are restored before and reset
% after loading.
%    \begin{macrocode}
\newcommand*\svg@RequirePackage{}
\def\svg@RequirePackage#1[#2]{%
  \svg@catcodes@restore%
  \RequirePackage{#1}[#2]%
  \svg@catcodes@set%
}
%    \end{macrocode}
% \end{macro}^^A \svg@RequirePackage
%
% \iffalse
%</base>
% \fi
%
% The setup itself right at the beginning of the two packages.
%    \begin{macrocode}
\svg@catcodes@set
\AtEndOfPackage{\svg@catcodes@restore}
%    \end{macrocode}
%
% \iffalse
%<*base>
% \fi
%
% \subsection{General macros}
%
% \begin{macro}{\svg@tempa}
% \begin{macro}{\svg@tempb}
% \begin{macro}{\if@svg@tempswa}
% \begin{macro}{\svg@box}
% Internal temporary macros.
%    \begin{macrocode}
\newcommand*\svg@tempa{}
\newcommand*\svg@tempb{}
\newbox\svg@box
\newif\if@svg@tempswa
%    \end{macrocode}
% \end{macro}^^A \svg@box
% \end{macro}^^A \if@svg@tempswa
% \end{macro}^^A \svg@tempb
% \end{macro}^^A \svg@tempa
% \begin{macro}{\svg@ifvalueisrelax}
% \changes{v2.03}{2019/08/01}{also accepts string \val{relax}}^^A
% For some keys the usage of \val{\cs{relax}} as a value should lead to a 
% special reaction, such as restoring to default behavior or reseting the key.
% Therefore, \cs{svg@ifvalueisrelax} checks, whether \val{\cs{relax}} was used
% as value or not.
%    \begin{macrocode}
\newcommand*\svg@ifvalueisrelax[1]{%
  \begingroup%
    \def\svg@tempa{#1}%
    \def\svg@tempb{\relax}%
    \ifx\svg@tempa\svg@tempb%
      \aftergroup\@firstoftwo%
    \else%
      \def\svg@tempb{relax}%
      \ifx\svg@tempa\svg@tempb%
        \aftergroup\@firstoftwo%
      \else%
        \aftergroup\@secondoftwo%
      \fi%
    \fi%
  \endgroup%
}
%    \end{macrocode}
% \end{macro}^^A \svg@ifvalueisrelax
%
% \iffalse
%</base>
% \fi
%
%
% \section{Including SVG files with package \pkg{svg}}
% \subsection{Options}
%
% All options, which can be set either as package options or with 
% \cs{svgsetup}, as well as the optional parameters for both user commands 
% \cs{includesvg}\oarg{parameters}\marg{svg filename} and
% \cs{includeinkscape}\oarg{parameters}\marg{graphic filename} are defined with
% the interface provided by package \pkg{scrbase}.
%    \begin{macrocode}
\DefineFamily{SVG}
\DefineFamilyMember{SVG}
%    \end{macrocode}
%
% \iffalse
%</package&option>
%<*package&base>
%<*option>
% \fi
%
% \begin{macro}{\svg@deprecated@key}
% With version v2.00 the whole user interface was renewed. For reasons of 
% compatibility, outdated options and parameters from version~v1.0 are also 
% provided. If an old key was given, a warning is issued and the valid key is
% used.
%    \begin{macrocode}
\newcommand*\svg@deprecated@key[3][svg]{%
  \PackageWarning{#1}{%
    The option key `#2' is deprecated. \MessageBreak%
    It's recommended to use `#3'\MessageBreak%
    instead%
  }%
  \FamilyOptions{SVG}{#3}%
}
%    \end{macrocode}
% \end{macro}^^A \svg@deprecated@key
%
% Within the exported \LaTeX~files of \app{Inkscape}, some commands are used 
% out of additional packages. But maybe the user doesn't want to load this 
% packages anyways.
%
% \begin{option}{usexcolor}
% \changes{v2.00}{2017/02/17}{new}^^A
% \begin{option}{noxcolor}
% \changes{v2.00}{2017/02/17}{new}^^A
% \begin{macro}{\if@svg@use@xcolor}
% \begin{option}{usetransparent}
% \changes{v2.00}{2017/02/17}{new}^^A
% \begin{option}{notransparent}
% \changes{v2.00}{2017/02/17}{new}^^A
% \begin{macro}{\if@svg@use@transparent}
% Options for preventing packages \pkg{xcolor} and \pkg{transparent} to be 
% loaded.
%    \begin{macrocode}
\newif\if@svg@use@xcolor
\FamilyBoolKey{SVG}{usexcolor}{@svg@use@xcolor}
\DeclareOption{noxcolor}{\FamilyOptions{SVG}{usexcolor=false}}
\newif\if@svg@use@transparent
\FamilyBoolKey{SVG}{usetransparent}{@svg@use@transparent}
\DeclareOption{notransparent}{\FamilyOptions{SVG}{usetransparent=false}}
%    \end{macrocode}
% They are only available during the loading process of package \pkg{svg}.
%    \begin{macrocode}
\AtEndOfPackage{%
  \RelaxFamilyKey{SVG}{usexcolor}%
  \RelaxFamilyKey{SVG}{usetransparent}%
  \if@svg@use@xcolor%
    \RequirePackage{xcolor}[2016/05/11]%
  \else%
    \AfterPackage*{xcolor}{%
      \PackageWarning{svg}{Package `xcolor' was loaded anyway}%
    }%
  \fi%
  \if@svg@use@transparent%
    \RequirePackage{transparent}[2016/05/16]%
  \else%
    \AfterPackage*{transparent}{%
      \PackageWarning{svg}{Package `transparent' was loaded anyway}%
    }%
  \fi%
%    \end{macrocode}
% There is an issue\footnote{\url{https://github.com/ho-tex/oberdiek/issues/62}}
% with package \pkg{transparent}, which currently implements an \emph{invalid} 
% check relying on internal commands of package \pkg{pgfsys}, whereas these
% have changed in the latest version.
%    \begin{macrocode}
  \AfterPackage*{transparent}{%
    \@ifpackagelater{transparent}{2018/11/19}{}{\RequirePackage{pgfsys}}%
  }%
}
%    \end{macrocode}
% \end{macro}^^A \if@svg@use@transparent
% \end{option}^^A notransparent
% \end{option}^^A usetransparent
% \end{macro}^^A \if@svg@use@xcolor
% \end{option}^^A noxcolor
% \end{option}^^A usexcolor
%
%
%
% \subsubsection{The invocation of \app{Inkscape}}
%
% The Application \app{Inkscape} is used to create includable graphic files in 
% a desired format (PDF/EPS/PS/PNG) out of files in SVG~format, whereas the
% support of \LaTeX{} can optionally be used.
%
% \begin{option}{inkscape}
% \changes{v2.00}{2017/02/17}{changed/extended}^^A
% \changes{v2.02}{2018/09/07}{usage of \cs{svg@sanitize@dq}}^^A
% \changes{v2.01}{2017/11/28}{using \cs{trim@spaces}}^^A
% \changes{v2.03}{2019/06/28}{improved handling of legacy option}^^A
% \begin{macro}{\svg@ink@mode}
% The intension of option \opt{inkscape} is to control the running behaviour of 
% \app{Inkscape}. It can be switched off at all (\opt{inkscape}|=|\val{false}) 
% or invoked only if necessary (\opt{inkscape}|=|\val{true}) or the command
% line call can be forced with every \LaTeX~run (\opt{inkscape}|=|\val{forced}).
% Additionally, option \opt{inkscape} can be used as wrapper for options
% \opt{inkscapeformat}, \opt{inkscapelatex}, \opt{inkscapearea} and
% \opt{inkscapedpi}, which are declared later.
%    \begin{macrocode}
\newcommand*\svg@ink@mode{}
\DefineFamilyKey{SVG}{inkscape}[true]{%
  \svg@sanitize@dq\svg@tempb{#1}%
  \svg@lowercase{\svg@tempb}%
  \FamilySetNumerical{SVG}{inkscape}{svg@tempa}{%
    {false}{0},{off}{0},{no}{0},%
    {true}{1},{on}{1},{yes}{1},{auto}{1},{onlynewer}{1},{newer}{1},%
    {forced}{2},{force}{2},{overwrite}{2},%
    {pdf}{3},{eps}{4},{ps}{5},{png}{6},%
    {drawing}{7},{crop}{7},%
    {page}{8},{nocrop}{8},%
    {tex}{9},{latex}{9},{exportlatex}{9},{latexexport}{9},%
    {nolatex}{10},{notex}{10},{noexportlatex}{10},{nolatexexport}{10},%
    {latexnoexport}{10},{raw}{10},{plain}{10},{simple}{10}%
  }{\svg@tempb}%
  \ifx\FamilyKeyState\FamilyKeyStateProcessed%
%    \end{macrocode}
% Setting the mode for invoking \app{Inkscape}, whereas the conditional 
% invocation is only possible, when \cs{pdf@filemoddate} or \cs{filemoddate}
% is defined\dots
%    \begin{macrocode}
    \ifcase\svg@tempa\relax% false
      \def\svg@ink@mode{0}%
    \or% true
      \ifx\svg@IfFileNewer\@undefined%
        \def\svg@ink@mode{2}%
      \else%
        \def\svg@ink@mode{1}%
      \fi%
    \or% forced
      \def\svg@ink@mode{2}%
%    \end{macrocode}
% \dots and the part as wrapper for different options.
%    \begin{macrocode}
    \or% pdf
      \FamilyOptions{SVG}{inkscapeformat=pdf}%
    \or% eps
      \FamilyOptions{SVG}{inkscapeformat=eps}%
    \or% ps
      \FamilyOptions{SVG}{inkscapeformat=ps}%
    \or% png
      \FamilyOptions{SVG}{inkscapeformat=png}%
    \or% drawing
      \FamilyOptions{SVG}{inkscapearea=drawing}%
    \or% page
      \FamilyOptions{SVG}{inkscapearea=page}%
    \or% tex
      \FamilyOptions{SVG}{inkscapelatex=true}%
    \or% nolatex
      \FamilyOptions{SVG}{inkscapelatex=false}%
    \fi%
%    \end{macrocode}
% It's also possible to set the option \opt{inkscapedpi} by passing a number 
% followed by \val{dpi} like \opt{inkscape}|=|\val{300dpi}.
%    \begin{macrocode}
  \else% dpi
    \def\svg@tempa##1dpi##2\@nil{%
      \ifstr{##2}{dpi}{\FamilyOptions{SVG}{inkscapedpi=##1}}{}%
    }%
    \expandafter\svg@tempa\svg@tempb dpi\@nil%
%    \end{macrocode}
% In version~v1.0 the option \opt{inkscape} was used to set both the executable 
% and options for \app{Inkscape}. This is taken into account here.
%    \begin{macrocode}
    \ifx\FamilyKeyState\FamilyKeyStateProcessed\else% legacy option
%    \end{macrocode}
% Splitting executable from options with delimitted macros. After calling 
% \cs{svg@tempa} with the given value, the part for the executable is stored in 
% \cs{svg@tempa} and the option part---which is recognized by the first |-| 
% character--- in \cs{svg@tempb}.
%    \begin{macrocode}
      \svg@quotes@remove[{#1}]{\svg@tempb}%
      \def\svg@tempa##1-##2\@nil{%
        \IfArgIsEmpty{##2}{\let\svg@tempb\@empty}{%
          \def\svg@tempa####1-\@nil{\def\svg@tempb{-####1}}%
          \svg@tempa##2\@nil%
        }%
        \edef\svg@tempa{\trim@spaces{##1}}%
      }%
      \expandafter\svg@tempa\svg@tempb-\@nil%
      \ifx\svg@tempa\@empty\else%
        \if@svg@quotes@found%
          \edef\svg@tempa{"\svg@tempa"}%
        \fi%
      \fi%
      \@svg@tempswafalse%
      \ifx\svg@tempa\@empty\else\@svg@tempswatrue\fi%
      \ifx\svg@tempb\@empty\else\@svg@tempswatrue\fi%
      \if@svg@tempswa%
        \PackageWarning{svg}{%
          Setting%
          \ifx\svg@tempa\@empty\else%
            \space the executable \ifx\svg@tempb\@empty\else and \fi%
          \fi%
          \ifx\svg@tempb\@empty\else%
            \space associated options%
          \fi%
          \MessageBreak%
          for Inkscape should be done with%
          \ifx\svg@tempa\@empty\else%
            \MessageBreak%
            `inkscapeexe=\svg@tempa'%
          \fi%
          \ifx\svg@tempb\@empty\else%
            \MessageBreak%
            \ifx\svg@tempa\@empty\else and \fi`inkscapeopt={\svg@tempb}'%
          \fi%
          \MessageBreak%
          Nevertheless, this was done by now anyway%
        }%
        \edef\svg@tempa{%
          \ifx\svg@tempa\@empty\else%
            \noexpand\FamilyOptions{SVG}{inkscapeexe=\svg@tempa}%
          \fi%
          \ifx\svg@tempb\@empty\else%
            \noexpand\FamilyOptions{SVG}{inkscapeopt=\svg@tempb}%
          \fi%
        }%
        \svg@tempa%
      \fi%
    \fi%
  \fi%
}
%    \end{macrocode}
% \end{macro}^^A \svg@ink@mode
% \end{option}^^A inkscape
% \begin{option}{on}
% \changes{v2.00}{2017/02/24}{new}^^A
% \begin{option}{forced}
% \changes{v2.03}{2019/07/25}{new}^^A
% \begin{option}{auto}
% \changes{v2.03}{2019/07/25}{new}^^A
% \begin{option}{off}
% \changes{v2.00}{2017/02/24}{new}^^A
% Package options which can be used to switch functionality on or off during 
% the loading of package \pkg{svg}.
%    \begin{macrocode}
\DeclareOption{on}{\AtEndOfPackage{\FamilyOptions{SVG}{inkscape=forced}}}
\DeclareOption{forced}{\AtEndOfPackage{\FamilyOptions{SVG}{inkscape=forced}}}
\DeclareOption{auto}{\AtEndOfPackage{\FamilyOptions{SVG}{inkscape=true}}}
\DeclareOption{off}{\AtEndOfPackage{\FamilyOptions{SVG}{inkscape=false}}}
%    \end{macrocode}
% \end{option}^^A off
% \end{option}^^A auto
% \end{option}^^A forced
% \end{option}^^A on
% \begin{option}{inkscapeversion}
% \changes{v2.03}{2019/08/02}{new}^^A
% \begin{macro}{\svg@ink@ver}
% \changes{v2.03}{2019/06/28}{new}^^A
% \begin{option}{inkscapeexe}
% \changes{v2.00}{2017/02/17}{new}^^A
% \changes{v2.03}{2019/06/28}{only usable in preamble}%
% \begin{macro}{\svg@ink@exe}
% \begin{option}{inkscapeopt}
% \changes{v2.00}{2017/02/17}{new}^^A
% \begin{macro}{\svg@ink@opt}
% With these options, the terminal command for invoking \app{Inkscape} as well 
% as additional options can be defined.
%    \begin{macrocode}
\newcommand*\svg@ink@ver{\m@ne}
\DefineFamilyKey{SVG}{inkscapeversion}[true]{%
  \FamilySetNumerical{SVG}{inkscape}{svg@tempa}{%
    {true}{0},{on}{0},{yes}{0},{auto}{0},{detect}{0}%
  }{#1}%
  \ifx\FamilyKeyState\FamilyKeyStateProcessed%
    \renewcommand*\svg@ink@ver{\m@ne}%
  \else%
    \def\svg@tempa##1.##2\@nil{%
      \svg@ifnumber{##1}{%
        \renewcommand*\svg@ink@ver{##1}%
        \FamilyKeyStateProcessed%
      }{}%
    }%
    \svg@tempa#1.\@nil%
  \fi%
}
\newcommand*\svg@ink@exe{inkscape}
\DefineFamilyKey{SVG}{inkscapeexe}{%
  \svg@sanitize@dq\svg@ink@exe{#1}%
  \FamilyKeyStateProcessed%
}
\newcommand*\svg@ink@opt{}
\DefineFamilyKey{SVG}{inkscapeopt}{%
  \renewcommand*\svg@ink@opt{#1}%
  \FamilyKeyStateProcessed%
}
%    \end{macrocode}
% The two options \opt{inkscapeversion} and \opt{inkscapeexe} can only be used 
% within the preamble.
%    \begin{macrocode}
\def\svg@tempa#1{%
  \AtBeginDocument{%
    \DefineFamilyKey[.svg.sty]{SVG}{#1}[]{%
      \PackageError{svg}{Option `#1' too late}{%
        Option `#1' can only be set within\MessageBreak%
        the preamble but you have tried to set it up later.%
      }%
      \FamilyKeyStateProcessed%
    }%
  }%
}
\svg@tempa{inkscapeexe}
\svg@tempa{inkscapeversion}
%    \end{macrocode}
% \end{macro}^^A \svg@ink@opt
% \end{option}^^A inkscapeopt
% \end{macro}^^A \svg@ink@exe
% \end{option}^^A inkscapeexe
% \end{macro}^^A \svg@ink@ver
% \end{option}^^A inkscapeversion
% \begin{option}{inkscapeformat}
% \changes{v2.00}{2017/02/17}{new}^^A
% \begin{macro}{\svg@ink@format}
% With option \opt{inkscapeformat} the output format of the \app{Inkscape} 
% export function, which is called via \cs{ShellEscape}, can be configured.
% It is set to \val{pdf} or, if dvi output could be detected, to \val{eps} 
% during initialization.
%    \begin{macrocode}
\newcommand*\svg@ink@format{pdf}
\ifxetex\else\ifpdf\else
  \renewcommand*\svg@ink@format{eps}
\fi\fi
\DefineFamilyKey{SVG}{inkscapeformat}{%
  \svg@lowercase[{#1}]{\svg@tempa}%
  \FamilySetNumerical{SVG}{inkscapeformat}{svg@tempa}{%
    {pdf}{0},{eps}{1},{ps}{2},{png}{3}%
  }{\svg@tempa}%
  \ifx\FamilyKeyState\FamilyKeyStateProcessed%
    \ifcase\svg@tempa\relax% latex
      \renewcommand*\svg@ink@format{pdf}%
    \or% eps
      \renewcommand*\svg@ink@format{eps}%
    \or% ps
      \renewcommand*\svg@ink@format{ps}%
    \or% png
      \renewcommand*\svg@ink@format{png}%
    \fi%
  \fi%
}
%    \end{macrocode}
% \end{macro}^^A \svg@ink@format
% \end{option}^^A inkscapeformat
% \begin{option}{inkscapelatex}
% \changes{v2.00}{2017/02/17}{new}^^A
% \begin{option}{latex}
% \changes{v2.00b}{2017/02/28}{new, alternative key for \opt{inkscapelatex}}^^A
% \begin{option}{tex}
% \changes{v2.00b}{2017/03/28}{new, alternative key for \opt{inkscapelatex}}^^A
% \begin{macro}{\svg@ink@latex}
% This option controls whether the \app{Inkscape} export will be invoked with 
% or without the generation of a seperate \LaTeX~file.
%    \begin{macrocode}
\newif\if@svg@ink@latex
\FamilyBoolKey{SVG}{inkscapelatex}{@svg@ink@latex}
\FamilyBoolKey{SVG}{latex}{@svg@ink@latex}
\FamilyBoolKey{SVG}{tex}{@svg@ink@latex}
%    \end{macrocode}
% \end{macro}^^A \svg@ink@latex
% \end{option}^^A tex
% \end{option}^^A latex
% \end{option}^^A inkscapelatex
% \begin{option}{inkscapearea}
% \changes{v2.00}{2017/02/17}{new}^^A
% \begin{macro}{\svg@ink@area}
% The exported area for an \app{Inkscape} graphic can be set with this option.
%    \begin{macrocode}
\newcommand*\svg@ink@area{}
\DefineFamilyKey{SVG}{inkscapearea}{%
  \FamilySetNumerical{SVG}{inkscapearea}{svg@tempa}{%
    {drawing}{0},{crop}{0},%
    {page}{1},{nocrop}{1}%
  }{#1}%
  \ifx\FamilyKeyState\FamilyKeyStateProcessed%
    \ifcase\svg@tempa\relax% drawing
      \renewcommand*\svg@ink@area{-D}%
    \else% page
      \renewcommand*\svg@ink@area{-C}%
    \fi%
  \fi%
}
%    \end{macrocode}
% \end{macro}^^A \svg@ink@area
% \end{option}^^A inkscapearea
% \begin{option}{inkscapedpi}
% \changes{v2.00}{2017/02/17}{new}^^A
% \begin{option}{inkscapedensity}
% \begin{macro}{\svg@ink@dpi}
% A density can be chosen, which is used during export with \app{Inkscape} for 
% bitmaps and rasterization of filters.
%    \begin{macrocode}
\newcommand*\svg@ink@dpi{}
\let\svg@ink@dpi\relax
\DefineFamilyKey{SVG}{inkscapedpi}{%
  \FamilyKeyStateUnknownValue%
  \svg@ifvalueisrelax{#1}{%
    \let\svg@ink@dpi\relax%
    \FamilyKeyStateProcessed%
  }{%
    \svg@lowercase[{#1}]{\svg@tempb}%
    \def\svg@tempa##1dpi##2\@nil{\def\svg@tempa{##1}}%
    \expandafter\svg@tempa\svg@tempb dpi\@nil%
    \svg@ifnumber{\svg@tempa}{%
      \edef\svg@ink@dpi{\svg@tempa}%
      \FamilyKeyStateProcessed%
    }{}%
  }%
}
\DefineFamilyKey{SVG}{inkscapedensity}{\FamilyOptions{SVG}{inkscapedpi=#1}}
%    \end{macrocode}
% \end{macro}^^A \svg@ink@dpi
% \end{option}^^A inkscapedensity
% \end{option}^^A inkscapedpi
%
%
%
% \subsubsection{Setting input folder and file}
%
% \begin{option}{svgpath}
% \changes{v2.00}{2017/02/12}{deprecated}^^A
% In version~v1.0 setting the path to SVG~files was done via option. So 
% this method is provided as well.
%    \begin{macrocode}
\DefineFamilyKey{SVG}{svgpath}{%
  \PackageWarning{svg}{%
    The key `svgpath' is deprecated. It's recommended\MessageBreak%
    to use `\string\svgpath' instead%
  }%
  \ifx\svgpath\@undefined%
    \AtEndOfPackage{\svgpath{{#1}}}%
  \else%
    \svgpath{{#1}}%
  \fi%
  \FamilyKeyStateProcessed%
}
%    \end{macrocode}
% \end{option}^^A svgpath
% \begin{option}{svgextension}
% \changes{v2.01}{2017/11/27}{new due to user request}^^A
% \changes{v2.02}{2018/09/07}{usage of \cs{svg@quotes@remove}}^^A
% \changes{v2.02}{2018/09/07}{usage of \cs{svg@remove@leadingchar}}^^A
% \begin{option}{svgformat}
% \begin{macro}{\svg@file@ext}
% This option modifies the expected extension for the input file which is 
% exported with \app{Inkscape}. It is set to \file{svg} by default.
%    \begin{macrocode}
\newcommand*\svg@file@ext{svg}
\DefineFamilyKey{SVG}{svgextension}{%
%    \end{macrocode}
% Remove quotes and leading dots from the extension, which should be in 
% lower case letters.
%    \begin{macrocode}
  \svg@quotes@remove*[{#1}]{\svg@file@ext}%
  \svg@lowercase{\svg@file@ext}%
  \svg@remove@leadingchar.\svg@file@ext%
  \FamilyKeyStateProcessed%
}
\DefineFamilyKey{SVG}{svgformat}{\FamilyOptions{SVG}{svgextension=#1}}
%    \end{macrocode}
% \end{macro}^^A \svg@file@ext
% \end{option}^^A svgformat
% \end{option}^^A svgextension
%
%
%
% \subsubsection{Setting output folder}
%
% \begin{option}{inkscapepath}
% \changes{v2.00}{2017/02/17}{new}^^A
% \changes{v2.02}{2018/09/07}{usage of \cs{svg@sanitize@dq}}^^A
% \begin{macro}{\svg@out@path}
% The option \opt{inkscapepath} controls, in which folder the results of the
% \app{Inkscape} export will be located.
%    \begin{macrocode}
\newcommand*\svg@out@path{}
\DefineFamilyKey{SVG}{inkscapepath}{%
  \svg@sanitize@dq\svg@tempb{#1}%
  \FamilySetNumerical{SVG}{inkscapepath}{svg@tempa}{%
    {svgpath}{0},{svgdir}{0},%
    {svgsubpath}{1},{svgsubdir}{1},%
    {basepath}{2},{basedir}{2},{jobpath}{2},{jobdir}{2},%
    {basesubpath}{3},{basesubdir}{3},{jobsubpath}{3},{jobsubdir}{3}%
  }{\svg@tempb}%
  \ifx\FamilyKeyState\FamilyKeyStateProcessed%
    \ifcase\svg@tempa\relax% svgpath
      \renewcommand*\svg@out@path{\svg@file@path}%
    \or% svgsubpath
      \renewcommand*\svg@out@path{\svg@file@path svg-inkscape/}%
    \or% basepath
      \renewcommand*\svg@out@path{\@currdir}%
    \or% basesubpath
      \renewcommand*\svg@out@path{\@currdir svg-inkscape/}%
    \fi%
  \else%
    \edef\svg@out@path{\svg@tempb}%
    \svg@normalize@path{\svg@out@path}%
    \FamilyKeyStateProcessed%
  \fi%
}
%    \end{macrocode}
% \end{macro}^^A \svg@out@path
% \end{option}^^A inkscapepath
% \begin{option}{inkscapename}
% \changes{v2.00}{2017/02/17}{new}^^A
% \changes{v2.03}{2019/07/25}{\opt{inkscapename} permits value \cs{relax}}^^A
% \begin{macro}{\svg@out@name}
% With option \opt{inkscapename} the name of the exported file can be changed.
% \ToDo{\opt{inkscapename}\val{=hash/checksum/md5}}[v2.03]
%    \begin{macrocode}
\newcommand*\svg@out@name{%
  \svg@file@name\string_\svg@file@ext\if@svg@ink@latex-tex\else-raw\fi%
}
\DefineFamilyKey{SVG}{inkscapename}{%
  \svg@ifvalueisrelax{#1}{%
    \renewcommand*\svg@out@name{%
      \svg@file@name\string_\svg@file@ext\if@svg@ink@latex-tex\else-raw\fi%
    }%
  }{%
    \renewcommand*\svg@out@name{#1}%
  }%
  \FamilyKeyStateProcessed%
}
%    \end{macrocode}
% \end{macro}^^A \svg@out@name
% \end{option}^^A inkscapename
%
%
%
% \subsubsection{Options for the inclusion of graphics}
%
% After the graphic export with \app{Inkscape}, the inclusion of those graphics 
% can be controlled with the following options.
%
% \begin{option}{width}
% \begin{macro}{\svg@param@width}
% \begin{option}{height}
% \changes{v2.00}{2017/02/17}{new}^^A
% \begin{macro}{\svg@param@width}
% \begin{option}{distort}
% \changes{v2.02}{2018/09/07}{new}^^A
% \begin{option}{keepaspectratio}
% \changes{v2.02}{2018/09/07}{new}^^A
% \begin{macro}{\if@svg@param@distort}
% \begin{option}{scale}
% \changes{v2.00}{2017/02/17}{new}^^A
% \begin{macro}{\svg@param@scale}
% These options determine the size of the included graphics. The usage of
% \val{\cs{relax}} as value resets the respective option to the default 
% behavior.
% \ToDo{%
%   implement key \opt{size} in order to use it as master key like%
%   \opt{size=1.5} (scale), \opt{size=5cm} (width) or even \opt{size=5cm:6cm}
%   (width and height) \opt{size=:6cm} (only height) and {size=5cm!6cm} 
%   (distort) as well as \opt{size=5cm:*}, \opt{size=*:6cm}, \opt{size=5cm!*},
%   \opt{size=*!6cm} for inheritance (check if given).
% }[v2.03]
%    \begin{macrocode}
\newcommand*\svg@param@width{\z@}
\DefineFamilyKey{SVG}{width}{%
  \svg@ifvalueisrelax{#1}{%
    \renewcommand*\svg@param@width{\z@}%
    \FamilyKeyStateProcessed%
  }{%
    \FamilySetUseLengthMacro{SVG}{width}{\svg@param@width}{#1}%
    \ifx\FamilyKeyState\FamilyKeyStateProcessed%
      \ifdim\svg@param@width<\z@\relax%
        \FamilyKeyStateUnknownValue%
      \fi%
    \fi%
  }%
}
\newcommand*\svg@param@height{\z@}
\DefineFamilyKey{SVG}{height}{%
  \svg@ifvalueisrelax{#1}{%
    \renewcommand*\svg@param@height{\z@}%
    \FamilyKeyStateProcessed%
  }{%
    \FamilySetUseLengthMacro{SVG}{height}{\svg@param@height}{#1}%
    \ifx\FamilyKeyState\FamilyKeyStateProcessed%
      \ifdim\svg@param@height<\z@\relax%
        \FamilyKeyStateUnknownValue%
      \fi%
    \fi%
  }%
}
\newif\if@svg@param@distort
\FamilyBoolKey{SVG}{distort}{@svg@param@distort}
\DefineFamilyKey{SVG}{keepaspectratio}[true]{%
  \FamilySetBool{SVG}{keepaspectratio}{@svg@tempswa}{#1}%
  \ifx\FamilyKeyState\FamilyKeyStateProcessed%
    \if@svg@tempswa%
      \FamilyExecuteOptions[.svg.sty]{SVG}{distort=false}%
    \else%
      \FamilyExecuteOptions[.svg.sty]{SVG}{distort=true}%
    \fi%
  \fi%
}
\newcommand*\svg@param@scale{1}
\DefineFamilyKey{SVG}{scale}{%
  \FamilyKeyStateUnknownValue%
  \svg@ifvalueisrelax{#1}{%
    \renewcommand*\svg@param@scale{1}%
    \FamilyKeyStateProcessed%
  }{%
    \ifisdimension{#1\p@}{%
      \ifdim\dimexpr#1\p@\relax>\z@\relax%
        \renewcommand*\svg@param@scale{#1}%
        \FamilyKeyStateProcessed%
      \fi%
    }{}%
  }%
}
%    \end{macrocode}
% \end{macro}^^A \svg@param@scale
% \end{option}^^A scale
% \end{macro}^^A \if@svg@param@distort
% \end{option}^^A keepaspectratio
% \end{option}^^A distort
% \end{macro}^^A \svg@param@height
% \end{option}^^A height
% \end{macro}^^A \svg@param@width
% \end{option}^^A width
% \begin{option}{pretex}
% \begin{macro}{\svg@param@pretex}
% \begin{option}{apptex}
% \begin{macro}{\svg@param@apptex}
% \begin{option}{postex}
% \changes{v2.00}{2017/03/02}{deprecated}^^A
% For executing code right before or after the graphic inclusion, two hooks are 
% defined.
%    \begin{macrocode}
\newcommand*\svg@param@pretex{}
\let\svg@param@pretex\relax
\DefineFamilyKey{SVG}{pretex}{%
  \svg@ifvalueisrelax{#1}{%
    \let\svg@param@pretex\relax%
  }{%
    \def\svg@param@pretex{#1}%
  }%
  \FamilyKeyStateProcessed%
}
\newcommand*\svg@param@apptex{}
\let\svg@param@apptex\relax
\DefineFamilyKey{SVG}{apptex}{%
  \svg@ifvalueisrelax{#1}{%
    \let\svg@param@apptex\relax%
  }{%
    \def\svg@param@apptex{#1}%
  }%
  \FamilyKeyStateProcessed%
}
\DefineFamilyKey{SVG}{postex}{%
  \svg@deprecated@key{postex=#1}{apptex=#1}%
}
%    \end{macrocode}
% \end{option}^^A postex
% \end{macro}^^A \svg@param@apptex
% \end{option}^^A apptex
% \end{macro}^^A \svg@param@pretex
% \end{option}^^A pretex
% \begin{option}{lastpage}
% \changes{v2.00}{2017/02/20}{new}^^A
% \begin{counter}{svg@param@lastpage}
% For \app{Inkscape}~0.91 a bug concerning the \LaTeX~export has been reported
% (\url{https://bugs.launchpad.net/ubuntu/+source/inkscape/+bug/1417470}). 
% Sometimes the \LaTeX~file created by \app{Inkscape} tries to include more
% pages than actually are present in the PDF~file. To work around this 
% problem, a patch is provided. For this purpose, the total page number is read 
% from the PDF~file. 
%    \begin{macrocode}
\newcounter{svg@param@lastpage}
\DefineFamilyKey{SVG}{lastpage}[true]{%
  \FamilySetNumerical{SVG}{lastpage}{svg@tempa}{%
    {false}{0},{off}{0},{no}{0},{ignore}{0},%
    {true}{1},{on}{1},{yes}{1},{auto}{1}%
  }{#1}%
  \ifx\FamilyKeyState\FamilyKeyStateProcessed%
    \ifcase\svg@tempa\relax% false
      \FamilySetCounter{SVG}{lastpage}{svg@param@lastpage}{\m@ne}%
    \or% true
      \FamilySetCounter{SVG}{lastpage}{svg@param@lastpage}{\z@}%
    \fi%
  \fi%
}
%    \end{macrocode}
% \end{counter}^^A svg@param@lastpage
% \end{option}^^A lastpage
% \begin{option}{draft}
% \changes{v2.00}{2017/02/17}{new}^^A
% \begin{macro}{\if@svg@draft}
% The option \opt{draft} has the same effect as the eponymous option of package
% \pkg{graphicx}.
%    \begin{macrocode}
\newif\if@svg@draft
\FamilyBoolKey{SVG}{draft}{@svg@draft}
\AfterPackage*{graphicx}{\ifGin@draft\@svg@drafttrue\fi}
%    \end{macrocode}
% \end{macro}^^A \if@svg@draft
% \end{option}^^A draft
%
%
%
% \subsection{Handling path information}
%
% Both packages \pkg{svg} and \pkg{svg-extract} should be able to handle 
% user-defined input and output paths. As there is the possibility for users to 
% provide paths with or without quotes to \LaTeX{}, this is taken into account.
%
% \begin{macro}{\svg@deactivate@dq}
% \changes{v2.02}{2018/09/07}{new}^^A
% \changes{v2.02a}{2018/10/17}{bug fix for \pkg{polyglossia}}^^A
% In order to avoid errors concerning file names with package \pkg{babel} and 
% it's active double quotes, this command is defined. 
%    \begin{macrocode}
\newcommand*\svg@deactivate@dq{}
\AfterPackage+{babel}{%
  \renewcommand*\svg@deactivate@dq{\bbl@deactivate{"}}%
  \providecommand*\bbl@deactivate[1]{}%
}
%    \end{macrocode}
% \end{macro}^^A \svg@deactivate@dq
% \begin{macro}{\svg@sanitize@dq}
% \changes{v2.02}{2018/09/07}{new}^^A
% Save expansion of the second argument in the macro from the first argument 
% with deactivated double quotes.
%    \begin{macrocode}
\newcommand*\svg@sanitize@dq[2]{%
  \begingroup%
    \svg@deactivate@dq%
    \edef\svg@tempa{\endgroup\def\noexpand#1{#2}}%
  \svg@tempa%
}
%    \end{macrocode}
% \end{macro}^^A \svg@sanitize@dq
% \begin{macro}{\svg@quotes@check}
% \begin{macro}{\svg@@quotes@check}
% \begin{macro}{\if@svg@quotes@found}
% During the treatment of paths, it may be necessary to temporarily remove 
% quotes and, if required, add them again later. For this purpose, the switch 
% \cs{if@svg@quotes@found} as well as the commands \cs{svg@quotes@check} and
% \cs{svg@@quotes@check}, which controls the switch, are defined. As before, 
% the string is passed in a macro to \cs{svg@quotes@check}.
%    \begin{macrocode}
\newif\if@svg@quotes@found
\newcommand*\svg@quotes@check[1]{%
  \expandafter\svg@@quotes@check#1"\@nil%
}
\newcommand*\svg@@quotes@check{}
\def\svg@@quotes@check#1"#2\@nil{%
  \IfArgIsEmpty{#2}{\@svg@quotes@foundfalse}{\@svg@quotes@foundtrue}%
}
%    \end{macrocode}
% \end{macro}^^A \if@svg@quotes@found
% \end{macro}^^A \svg@@quotes@check
% \end{macro}^^A \svg@quotes@check
% \begin{macro}{\svg@quotes@remove}
% \changes{v2.02}{2018/09/07}{usage of \cs{svg@sanitize@dq}}^^A
% \changes{v2.02}{2018/09/07}{calling \cs{svg@quotes@check}}^^A
% \changes{v2.03}{2018/08/28}{star version for trimming spaces}^^A
% \begin{macro}{\svg@@quotes@remove}
% \begin{macro}{\svg@@@quotes@remove}
% \begin{macro}{\svg@@@@quotes@remove}
% These commands are used to remove all occurring quotes within a string. The
% mandatory argument passed to \cs{svg@quotes@remove} is not the string itself
% but a macro in which a string is stored. The optional argument can be used to
% pass a different string which should manipulated and stored in the given
% macro. The star version trims spaces after removing the quotes.
%    \begin{macrocode}
\newcommand\svg@quotes@remove{%
  \kernel@ifstar%
    {\svg@@quotes@remove{\trim@spaces}}%
    {\svg@@quotes@remove{\@firstofone}}%
}
\newcommand*\svg@@quotes@remove[1]{%
  \kernel@ifnextchar[%]
    {\svg@@@quotes@remove{#1}}%
    {\svg@@@quotes@remove{#1}[\@nil]}%
}
\newcommand*\svg@@@quotes@remove{}
\def\svg@@@quotes@remove#1[#2]#3{%
  \begingroup%
%    \end{macrocode}
% As the optional argument could be used with an empty string, an undefined
% macro as token is used to detect an ungiven optional argument.
%    \begin{macrocode}
    \ifstr{\detokenize{#2}}{\detokenize{\@nil}}{%
      \svg@sanitize@dq\svg@tempa{#3}%
    }{%
      \svg@sanitize@dq\svg@tempa{#2}%
    }%
    \expandafter\svg@quotes@check\expandafter{\svg@tempa}%
    \edef\svg@tempa{\expandafter#1\expandafter{\svg@tempa}}%
    \expandafter\svg@@@@quotes@remove\svg@tempa"\@nil%
    \edef\svg@tempa{\expandafter#1\expandafter{\svg@tempa}}%
    \edef\svg@tempb{%
      \endgroup%
      \def\noexpand#3{\expandafter\detokenize\expandafter{\svg@tempa}}%
      \if@svg@quotes@found%
        \noexpand\@svg@quotes@foundtrue%
      \else%
        \noexpand\@svg@quotes@foundfalse%
      \fi%
    }%
  \svg@tempb%
}
%    \end{macrocode}
% Recursivly calling \cs{svg@@@@quotes@remove} removes every occuring quote.
%    \begin{macrocode}
\newcommand*\svg@@@@quotes@remove{}
\def\svg@@@@quotes@remove#1"#2\@nil{%
  \IfArgIsEmpty{#2}{%
    \edef\svg@tempa{#1}%
  }{%
    \svg@@@@quotes@remove#1#2\@nil%
  }%
}
%    \end{macrocode}
% \end{macro}^^A \svg@@@@quotes@remove
% \end{macro}^^A \svg@@@quotes@remove
% \end{macro}^^A \svg@@quotes@remove
% \end{macro}^^A \svg@quotes@remove
% \begin{macro}{\svg@quotes@normalize}
% \changes{v2.03}{2019/08/29}{new}^^A
% \begin{macro}{\svg@@quotes@normalize}
% \changes{v2.03}{2019/08/29}{new}^^A
% \begin{macro}{\svg@@@quotes@normalize}
% \changes{v2.03}{2019/08/29}{new}^^A
% Removes all quotes found and if any was existent, encloses the non-empty 
% string in the same. The star version trims spaces from string after removing 
% quotes with \cs{svg@quotes@remove*}.
%    \begin{macrocode}
\newcommand*\svg@quotes@normalize{%
  \kernel@ifstar%
    {\svg@@quotes@normalize{*}}%
    {\svg@@quotes@normalize{}}%
}
\newcommand*\svg@@quotes@normalize[1]{%
  \kernel@ifnextchar[%]
    {\svg@@@quotes@normalize{#1}}%
    {\svg@@@quotes@normalize{#1}[\@nil]}%
}
\newcommand*\svg@@@quotes@normalize{}
\def\svg@@@quotes@normalize#1[#2]#3{%
  \ifstr{\detokenize{#2}}{\detokenize{\@nil}}{%
    \svg@quotes@remove#1{#3}%
  }{%
    \svg@quotes@remove#1[#2]{#3}%
  }%
  \if@svg@quotes@found%
    \ifx#3\@empty\else%
      \edef#3{"#3"}%
    \fi%
  \fi%
}
%    \end{macrocode}
% \end{macro}^^A \svg@@@quotes@normalize
% \end{macro}^^A \svg@@quotes@normalize
% \end{macro}^^A \svg@quotes@normalize
% \begin{macro}{\svg@set@input@path}
% \changes{v2.01}{2017/11/28}{new}^^A
% \changes{v2.02}{2018/09/07}{usage of \cs{svg@deactivate@dq}}^^A
% \begin{macro}{\svg@append@input@path}
% \changes{v2.01}{2017/11/28}{new}^^A
% \changes{v2.02}{2018/09/07}{avoid duplicates in \cs{input@path}}^^A
% In order to import SVG~files from different folders, \cs{svg@set@input@path}
% evaluates several macros, which are supposed to be used for holding different 
% search folders. Any given path will be handled by \cs{svg@normalize@path}.
% The optional argument can be used to append an additional search path.
%    \begin{macrocode}
\newcommand*\svg@set@input@path[1][]{%
  \begingroup%
    \svg@deactivate@dq%
%    \end{macrocode}
% If a path was already found and stored within \cs{svg@file@path}, it is
% searched first and wrapped in curly braces. This is necessary for using
% commands like \cs{input}\marg{tex filename} within SVG~files.
%    \begin{macrocode}
    \ifx\svg@file@path\@currdir%
      \let\svg@file@path\@empty%
    \fi%
    \ifx\svg@file@path\@empty\else%
      \svg@normalize@path{\svg@file@path}%
      \edef\svg@file@path{{\svg@file@path}}%
    \fi%
%    \end{macrocode}
% Afterwards, several search paths are appended. If \cs{svgpath} was used, it 
% is searched next. If nothing was found, \cs{graphicspath} is considered if 
% defined followed by a path given in the third argument. If nothing was found
% yet, the standard \cs{input@path} is searched last.
%    \begin{macrocode}
    \svg@append@input@path{\svg@file@path}{\svg@input@path}%
    \svg@append@input@path{\svg@file@path}{\Ginput@path}%
    \IfArgIsEmpty{#1}{}{%
      \edef\svg@tempa{#1}%
      \svg@append@input@path{\svg@file@path}{{\svg@tempa}}%
    }%
    \svg@append@input@path{\svg@file@path}{\input@path}%
%    \end{macrocode}
% Finally, \cs{input@path} is set.
%    \begin{macrocode}
    \edef\svg@tempa{%
      \endgroup%
      \ifx\svg@file@path\@empty\else%
        \def\noexpand\input@path{\svg@file@path}%
      \fi%
    }%
  \svg@tempa%
}
%    \end{macrocode}
% Only, if a certain search path is defined, it is added. The paths given in 
% the first argument are compared to each path in the second argument and only 
% new ones are added.
%    \begin{macrocode}
\newcommand*\svg@append@input@path[2]{%
  \ifx#2\@undefined\else%
    \begingroup%
      \edef\svg@tempb{#2}%
      \expandafter\@tfor\expandafter\svg@tempa\expandafter:\expandafter=%
          \svg@tempb\do{%
%    \end{macrocode}
% Passing each new path to \cs{svg@normalize@path}. If a path already exists, 
% switch \cs{if@svg@tempswa} is set to \val{false}.
%    \begin{macrocode}
        \svg@normalize@path{\svg@tempa}%
        \ifx\svg@tempa\@currdir\else%
          \@svg@tempswatrue%
          \expandafter\@tfor\expandafter\svg@tempb\expandafter:\expandafter=%
              #1\do{%
            \ifx\svg@tempa\svg@tempb%
              \@svg@tempswafalse%
              \@break@tfor%
            \fi%
          }%
          \if@svg@tempswa%
            \edef#1{#1{\svg@tempa}}%
          \fi%
        \fi%
      }%
      \edef\svg@tempb{%
        \endgroup%
        \def\noexpand#1{#1}%
      }%
    \svg@tempb%
  \fi%
}
%    \end{macrocode}
% \end{macro}^^A \svg@append@input@path
% \end{macro}^^A \svg@set@input@path
% \begin{macro}{\svg@normalize@path}
% \changes{v2.02}{2018/09/07}{usage of \cs{svg@deactivate@dq}}^^A
% \begin{macro}{\svg@normalize@@path}
% If any path is given, a trailing slash is needed. These two macros ensure
% that this condition is fulfilled in any case, even if this is not considered
% by the user. As before, a macro containing the path string has to be used as 
% mandatory argument. The optional argument can be used to define, what path
% should be stored, if the found path was empty.
%    \begin{macrocode}
\newcommand*\svg@normalize@path[2][\@currdir]{%
  \begingroup%
    \svg@quotes@remove*[{#2}]{\svg@tempa}%
    \ifstr{\svg@tempa}{/}{\let\svg@tempa\@currdir}{}%
    \expandafter\svg@normalize@@path\svg@tempa//\@nil%
%    \end{macrocode}
% The optional argument is to be considered as normalized and no action is 
% taken onto this. The current directory is never enquoted.
%    \begin{macrocode}
    \ifx\svg@tempa\@empty%
      \edef\svg@tempa{#1}%
      \@svg@quotes@foundfalse%
    \else%
      \ifx\svg@tempa\@currdir%
        \@svg@quotes@foundfalse%
      \fi%
    \fi%
    \edef\svg@tempb{%
      \endgroup%
      \if@svg@quotes@found%
        \def\noexpand#2{"\svg@tempa"}%
      \else%
        \def\noexpand#2{\svg@tempa}%
      \fi%
    }%
  \svg@tempb%
}
\newcommand*\svg@normalize@@path{}
%    \end{macrocode}
% As directories should not start or end with spaces, these are removed.
%    \begin{macrocode}
\def\svg@normalize@@path#1/#2/\@nil{%
  \IfArgIsEmpty{#2}{%
    \let\svg@tempa\@empty%
    \IfArgIsEmpty{#1}{}{\edef\svg@tempa{\trim@spaces{#1}/}}%
  }{%
    \svg@normalize@@path#2/\@nil%
    \edef\svg@tempa{\trim@spaces{#1}/\svg@tempa}%
  }%
}
%    \end{macrocode}
% \end{macro}^^A \svg@normalize@@path
% \end{macro}^^A \svg@normalize@path
%
% \iffalse
%</option>
%<*body>
% \fi
%
% \begin{macro}{\svg@filename@parse}
% \changes{v2.02}{2018/09/07}{usage of \cs{svg@sanitize@dq}}^^A
% \changes{v2.02}{2018/09/07}{usage of \cs{svg@remove@leadingchar}}^^A
% \changes{v2.02}{2018/09/07}{usage of \cs{svg@extension@parse}}^^A
% As the internal \LaTeXe{} command \cs{filename@parse} is not able to split a 
% given file name containing quotes, \cs{svg@filename@parse} is defined to 
% resolve this problem. The optional argument can be used to give a specific 
% file extension, which should be searched within \cs{filename@ext}. If found 
% at the very end, the previous part is appended to \cs{filename@base}.
%    \begin{macrocode}
\newcommand*\svg@filename@parse[2][]{%
  \begingroup%
%    \end{macrocode}
% The given path and file is parsed with \cs{filename@parse}. The optional 
% argument is secured in \cs{svg@tempb} beforehand.
%    \begin{macrocode}
    \edef\svg@tempb{#1}%
    \svg@sanitize@dq\svg@tempa{#2}%
    \expandafter\filename@parse\expandafter{\svg@tempa}%
%    \end{macrocode}
% The found extension is parsed against the optional argument. Everything 
% before the desired extension is appended to \cs{filename@base}. Afterwards, 
% quotes within are normalized.
%    \begin{macrocode}
    \svg@normalize@path[\@empty]{\filename@area}%
    \svg@remove@leadingchar"\filename@base%
    \ifx\filename@ext\relax\else%
      \expandafter\svg@quotes@check\expandafter{\filename@ext}%
      \if@svg@quotes@found%
        \edef\filename@base{\filename@base"}%
      \fi%
      \expandafter\svg@extension@parse\expandafter{\svg@tempb}%
    \fi%
    \svg@quotes@normalize{\filename@base}%
%    \end{macrocode}
% With \cs{svg@tempa} the group is closed and the results are saved in the
% macros \cs{filename@\dots}.
%    \begin{macrocode}
    \edef\svg@tempa{%
      \endgroup%
      \def\noexpand\filename@area{\filename@area}%
      \def\noexpand\filename@base{\filename@base}%
      \ifx\filename@ext\relax%
        \let\noexpand\filename@ext\noexpand\relax%
      \else%
        \def\noexpand\filename@ext{\filename@ext}%
      \fi%
    }%
  \svg@tempa%
}
%    \end{macrocode}
% \end{macro}^^A \svg@filename@parse
% \begin{macro}{\svg@extension@parse}
% \changes{v2.02}{2018/09/07}{new}^^A
% This macro is used to permit multiple dots in file names. The content of 
% \cs{filename@ext} is split at each occurence of \val{.} and the trailing part 
% is compared against the content of the argument of \cs{svg@extension@parse}, 
% which is probably \cs{svg@file@ext}. If they are equal, the previous part is 
% appended to \cs{filename@base} and \cs{filename@ext} is set to the content of 
% the first argument.
%    \begin{macrocode}
\newcommand*\svg@extension@parse[1]{%
  \ifx\filename@ext\relax\else%
    \begingroup%
%    \end{macrocode}
% First of all, any given space \emph{after} file extension itself is stripped.
%    \begin{macrocode}
      \svg@quotes@remove{\filename@ext}%
      \trim@post@space@in\filename@ext%
      \ifx\filename@ext\@empty\else%
        \if@svg@quotes@found%
          \edef\filename@ext{"\filename@ext"}%
        \fi%
      \fi%
%    \end{macrocode}
% If file extension is the same as desired, nothing to do.
%    \begin{macrocode}
      \svg@ifstr@nocase{#1}{\filename@ext}{}{%
%    \end{macrocode}
% Macro \cs{svg@tempa} is recursively called as long as there are any dots or
% the searched extension is found. Anything before the searched extension at
% the end of \cs{filename@ext} is appended to \cs{filename@base}.
%    \begin{macrocode}
        \def\svg@tempa##1.##2\@nil##3\@nil{%
%    \end{macrocode}
% If the second argument is empty, last dot was found and the extension is 
% stored in \cs{filename@ext}. Otherwise, something inbetween two dots was 
% found and is appended to \cs{filename@base}.
%    \begin{macrocode}
          \IfArgIsEmpty{##2}{%
            \edef\filename@ext{##1}%
          }{%
            \edef\filename@base{\filename@base.##1}%
%    \end{macrocode}
% If the remaining part equals the searched extension, we can stop here. We use 
% the second argument as it is the current extension with given letter case. 
%Therefore, the trailing dot has to be removed.
%    \begin{macrocode}
            \svg@ifstr@nocase{##2}{##3.}{%
              \def\svg@tempb####1.\@nil{\edef\filename@ext{####1}}%
              \svg@tempb##2\@nil%
%    \end{macrocode}
% The searched extension was not found. So we call the macro with the remaning 
% part again.
%    \begin{macrocode}
            }{%
              \svg@tempa##2\@nil##3\@nil%
            }%
          }%
        }%
%    \end{macrocode}
% Calling \cs{svg@tempa} with the expanded arguments.
%    \begin{macrocode}
        \edef\svg@tempb{%
          \noexpand\svg@tempa\filename@ext.\noexpand\@nil#1\noexpand\@nil%
        }%
        \svg@tempb%
      }%
%    \end{macrocode}
% After the actual parsing is done, some reworking is necessary. Everything is 
% fine if either the found extension is the searched one or no specific
% extension to search for was given---an empty argument leads to returning the 
% part after the last dot. But if the desired extension was not found---which 
% is always the case if \cs{relax} is used as argument---the part found after 
% the last dot, is appended to \cs{filename@base} as well.
%    \begin{macrocode}
      \svg@ifstr@nocase{#1}{\filename@ext}{}{%
        \IfArgIsEmpty{#1}{}{%
          \ifx\filename@ext\@empty\else%
            \edef\filename@base{\filename@base.\filename@ext}%
            \let\filename@ext\@empty%
          \fi%
        }%
      }%
%    \end{macrocode}
% The macros \cs{filename@base} and \cs{filename@ext} are redefined with 
% normalized quotes after closing the group.
%    \begin{macrocode}
      \svg@quotes@normalize{\filename@base}%
      \svg@quotes@normalize{\filename@ext}%
      \edef\svg@tempa{%
        \endgroup%
        \def\noexpand\filename@base{\filename@base}%
        \ifx\filename@ext\@empty%
          \let\noexpand\filename@ext\relax%
        \else%
          \def\noexpand\filename@ext{\filename@ext}%
        \fi%
      }%
    \svg@tempa%
  \fi%
}
%    \end{macrocode}
% \end{macro}^^A \svg@extension@parse
% \begin{macro}{\svg@IfFileExists}
% \changes{v2.03}{2019/08/28}{new}^^A
% Using \cs{IfFileExists} with eventual quotes in file path and/or name 
% espacially together with evaluating \cs{input@path} is a mess. In order to
% get everything working well, \cs{svg@IfFileExists} is defined. This command
% evaluates \cs{input@path} before executing \cs{IfFileExists} and formats the 
% composed path like needed. Be aware, that \cs{svg@IfFileExists} does not 
% search the TDS of the distribution.
%    \begin{macrocode}
\newcommand*\svg@IfFileExists[1]{%
  \begingroup%
    \@svg@tempswafalse%
%    \end{macrocode}
% Expanding the given argument.
%    \begin{macrocode}
    \edef\svg@tempa{%
      \noexpand\svg@quotes@normalize[{#1}]{\noexpand\svg@tempa}%
      \noexpand\svg@quotes@remove*[{#1}]{\noexpand\svg@tempb}%
    }%
    \svg@tempa%
    \ifx\svg@tempb\@empty\else%
%    \end{macrocode}
% The optional argument of \cs{svg@filename@parse} appends any given extension 
% inclusive the dot to \cs{filename@base}.
%    \begin{macrocode}
      \svg@filename@parse[\relax]{\svg@tempa}%
      \ifx\filename@area\@currdir%
        \let\filename@area\@empty%
      \fi%
      \edef\svg@tempa{\filename@area\filename@base}%
%    \end{macrocode}
% After parsing the given file name, \cs{input@path} is evaluated. As TDS of 
% the distribution should not be searched the file name is prepended with 
% \cs{@currdir}. It has to be ensured, that a maybe given absolute path is also 
% considered. If there is a path within the argument, it is searched as well. 
% Afterwards, the additional paths from \cs{input@path} are seeked.
%    \begin{macrocode}
      \ifx\input@path\@undefined%
        \let\input@path\@empty%
      \fi%
      \edef\input@path{%
        {\@currdir}\ifx\filename@area\@empty\else{}\fi\input@path%
      }%
      \expandafter\@tfor\expandafter\svg@tempb\expandafter:\expandafter=%
          \input@path\do{%
%    \end{macrocode}
% The composed file name is stored in \cs{svg@tempb} and \cs{input@path} is set 
% to \cs{@undefined} in order to use the original \cs{IfFileExists}.
%    \begin{macrocode}
        \edef\svg@tempb{\svg@tempb\svg@tempa}%
        \let\input@path\@undefined%
        \begingroup%
          \svg@quotes@remove*{\svg@tempb}%
          \IfFileExists{"\svg@tempb"}{\aftergroup\@svg@tempswatrue}{}%
        \endgroup%
        \if@svg@tempswa%
          \@break@tfor%
        \fi%
      }%
    \fi%
    \edef\svg@tempa{%
      \endgroup%
      \if@svg@tempswa%
        \def\noexpand\@filef@und{\svg@tempb}%
        \noexpand\@firstoftwo%
      \else%
        \noexpand\@secondoftwo%
      \fi%
    }%
  \svg@tempa%
}
%    \end{macrocode}
% \end{macro}^^A \svg@IfFileExists
% \begin{macro}{\svg@get@file}
% \changes{v2.01}{2017/11/28}{using \cs{svg@set@input@path}}^^A
% \changes{v2.01}{2017/11/28}{using \cs{trim@spaces}}^^A
% \begin{macro}{\svg@@get@file}
% \begin{macro}{\svg@@@get@file}
% \begin{macro}{\svg@@@@get@file}
% \begin{macro}{\if@svg@file@found}
% \begin{macro}{\svg@file@path}
% \begin{macro}{\svg@file@name}
% \begin{macro}{\svg@file@base}
% The command \cs{svg@get@file} tries to find a file given by it's name and 
% extension within the two mandatory arguments. If the searched file wasn't
% found in the current path, all paths given with \cs{svgpath} are evaluated.
% If there was no appropriate file again, all paths given by \cs{graphicspath}
% are examined. In the last step, a given path within the optional argument is
% browsed. The results for file path and name are stored in \cs{svg@file@path}
% and \cs{svg@file@name} as well as the compound of both in \cs{svg@file@base}.
% The star version does not raise an error, if no file was found.
%    \begin{macrocode}
\newif\if@svg@file@found
\newcommand*\svg@file@path{}
\newcommand*\svg@file@name{}
\newcommand*\svg@file@base{}
\newcommand*\svg@@@@get@file{}
\newcommand*\svg@get@file{%
  \kernel@ifstar%
    {\let\svg@@@@get@file\@gobble\svg@@get@file}%
    {\let\svg@@@@get@file\@firstofone\svg@@get@file}%
}
\newcommand*\svg@@get@file[3][]{%
  \begingroup%
%    \end{macrocode}
% After calling \cs{svg@set@input@path}, all search paths are stored in
% \cs{input@path}, a single path given in the optional argument will also be 
% considered.
%    \begin{macrocode}
    \svg@set@input@path[{#1}]%
%    \end{macrocode}
% The file is searched by it's specified extension with \cs{svg@@@get@file}.
%    \begin{macrocode}
    \@svg@file@foundfalse%
    \svg@@@get@file{#2}{#3}%
%    \end{macrocode}
% If the file search was successful, the macro \cs{svg@filename@parse} is called
% with the result. If no path was parsed, the file is located in the current 
% directory.
%    \begin{macrocode}
    \if@svg@file@found%
      \svg@filename@parse[\filename@ext]{\@filef@und}%
      \edef\svg@file@path{%
        \ifx\filename@area\@empty\@currdir\else\filename@area\fi%
      }%
    \else%
      \edef\svg@file@path{\ifx\filename@area\@currdir\else\filename@area\fi}%
    \fi%
    \edef\svg@file@name{\filename@base}%
    \edef\svg@file@base{%
      \svg@file@path\svg@file@name\ifx\filename@ext\relax\else.\filename@ext\fi%
    }%
%    \end{macrocode}
% Remove potential multiple quotes from \cs{svg@file@base}.
%    \begin{macrocode}
    \svg@quotes@normalize{\svg@file@base}%
    \if@svg@file@found\else%
%    \end{macrocode}
% If no file was found, an error message is raised for the non star version.
%    \begin{macrocode}
      \svg@@@@get@file{%
        \edef\svg@tempb{\svg@file@name.#3}%
        \edef\svg@tempa{%
          There's no file `\svg@tempb'\MessageBreak%
          neither in\space%
          \ifx\svg@file@path\@empty%
            the current directory%
          \else%
            folder `\svg@file@path'%
          \fi%
          \space nor any other\MessageBreak searched path given by\space% 
          \string\svgpath\space or \string\graphicspath. \MessageBreak%
        }%
%    \end{macrocode}
% Collecting all considered path for the error message and appending it.
%    \begin{macrocode}
        \ifx\input@path\@undefined%
          \edef\svg@tempa{\svg@tempa No additional path was given.}%
        \else%
          \edef\svg@tempa{%
            \svg@tempa Following folders have additionally been searched:%
          }%
          \expandafter\@tfor\expandafter\input@path\expandafter:\expandafter=%
              \input@path\do{%
            \edef\input@path{\input@path\svg@file@path}%
            \svg@normalize@path{\input@path}%
            \edef\svg@tempa{\svg@tempa\MessageBreak\input@path}%
          }%
        \fi%
        \PackageError{svg}{File `\svg@tempb' is missing}{\svg@tempa}%
      }%
    \fi%
    \edef\svg@tempa{%
      \endgroup%
      \if@svg@file@found%
        \noexpand\@svg@file@foundtrue%
      \else%
        \noexpand\@svg@file@foundfalse%
      \fi%
      \def\noexpand\svg@file@path{\svg@file@path}%
      \def\noexpand\svg@file@name{\svg@file@name}%
      \def\noexpand\svg@file@base{\svg@file@base}%
    }%
  \svg@tempa%
}
%    \end{macrocode}
% A needed file extension in the second argument but may be not given one 
% within the file name in first argument is appended to \cs{filename@ext}.
%    \begin{macrocode}
\newcommand*\svg@@@get@file[2]{%
  \svg@filename@parse[{#2}]{#1}%
  \ifx\filename@ext\relax%
    \edef\filename@ext{#2}%
  \fi%
  \svg@IfFileExists{\filename@area\filename@base.\filename@ext}{%
    \@svg@file@foundtrue%
  }{%
%    \end{macrocode}
% If no file was found, we are trying again and trim surrounding spaces at the 
% given file name as they are maybe unwanted.
%    \begin{macrocode}
    \svg@quotes@normalize*[\filename@base]{\svg@tempa}%
    \svg@IfFileExists{\filename@area\svg@tempa.\filename@ext}{%
      \@svg@file@foundtrue%
%    \end{macrocode}
% If nothing was found, a maybe empty file extension is set to \cs{relax}.
%    \begin{macrocode}
    }{%
      \ifx\filename@ext\@empty%
        \let\filename@ext\relax%
      \fi%
    }%
  }%
}
%    \end{macrocode}
% \end{macro}^^A \svg@file@base
% \end{macro}^^A \svg@file@name
% \end{macro}^^A \svg@file@path
% \end{macro}^^A \if@svg@file@found
% \end{macro}^^A \svg@@@@get@file
% \end{macro}^^A \svg@@@get@file
% \end{macro}^^A \svg@@get@file
% \end{macro}^^A \svg@get@file
% \begin{macro}{\svg@file@missing}
% \changes{v2.02}{2018/09/07}{notify svg file when missing exported files}^^A
% The error message, which is raised, if a file derived is missing.
%    \begin{macrocode}
\newcommand*\svg@file@missing[2][]{%
  \begingroup%
    \svg@quotes@normalize[{#2}]{\svg@tempa}%
    \PackageError{svg}{File `\svg@tempa' is missing}{%
      Did you run the Inkscape export%
      \if@svg@ink@latex\space with latex output\fi?\MessageBreak%
      There's no file `\svg@tempa'%
      \IfArgIsEmpty{#1}{}{\MessageBreak although `#1' was found}.%
      \if@svg@ink@latex%
        \space Maybe\MessageBreak%
        you should use optional parameter `inkscapelatex=false'.%
      \fi%
    }%
  \endgroup%
}
%    \end{macrocode}
% \end{macro}^^A \svg@file@missing
%
% \iffalse
%</body>
%</package&base>
% \fi
%
% \subsection{User commands}
% \subsubsection{Optional parameters for user commands}
%
% The family member is defined for both \pkg{svg} and \pkg{svg-extract}.
%    \begin{macrocode}
%<*body>
\DefineFamilyMember[.param]{SVG}
%</body>
%    \end{macrocode}
%
% \iffalse
%<*package&base>
%<*body>
% \fi
%
% \begin{macro}{\svg@local@param@set}
% \changes{v2.02}{2018/09/07}{reasonable value for key \opt{distort}}^^A
% \begin{macro}{\svg@local@param@use}
% \begin{macro}{\svg@local@param@def}
% Most of the package options can also be used as optional parameters for 
% \cs{includesvg} or \cs{includeinkscape}. Some of them are overloaded for the 
% usage as optional argument and there are some keys, which \emph{only} can be
% used as optional parameters. This is realized in such a way that
% \cs{svg@local@param@use} is extended with \cs{svg@local@param@def} by the
% definition of local keys during the loading of package \pkg{svg}.
%    \begin{macrocode}
\newcommand*\svg@local@param@set[1]{%
  \svg@local@param@use%
  \FamilyOptions{SVG}{#1}%
%    \end{macrocode}
% As \cs{svg@local@param@set} is always used in a local group, it is possible 
% to set \opt{inkscapelatex} to \val{false}, if the output format was set to 
% \val{png} with option \opt{inkscapeformat}.
%    \begin{macrocode}
  \ifstr{\svg@ink@format}{png}{\FamilyOptions{SVG}{inkscapelatex=false}}{}%
%    \end{macrocode}
% Using \opt{distort=true} is only reasonable, if \opt{height} and \opt{width} 
% are given.
%    \begin{macrocode}
  \@svg@tempswatrue%
  \ifdim\svg@param@width>\z@\relax\ifdim\svg@param@height>\z@\relax%
    \@svg@tempswafalse%
  \fi\fi%
  \if@svg@tempswa%
    \FamilyExecuteOptions[.svg.sty]{SVG}{distort=false}%
  \fi%
}
\newcommand*\svg@local@param@use{\svg@local@param@deprecated}
\newcommand*\svg@local@param@def[1]{\l@addto@macro\svg@local@param@use{#1}}
%    \end{macrocode}
% \end{macro}^^A \svg@local@param@def
% \end{macro}^^A \svg@local@param@use
% \end{macro}^^A \svg@local@param@set
% \begin{macro}{\svg@local@param@deprecated}
% \changes{v2.03}{2019/07/23}{more specifig warning message}^^A
% This macro checks, if \cs{svgwidth} or \cs{svgscale} are defined. In this 
% case, the given values are passed to the correlating parameters and a warning 
% is raised.
%    \begin{macrocode}
\newcommand*\svg@local@param@deprecated{%
  \begingroup%
    \let\svg@tempa\@empty%
    \ifx\svgwidth\@undefined\else%
      \ifx\svgwidth\relax%
        \l@addto@macro\svg@tempa{,width=\relax}%
      \else%
        \l@addto@macro\svg@tempa{,width=\svgwidth}%
      \fi%
    \fi%
    \ifx\svgscale\@undefined\else%
      \ifx\svgscale\relax%
        \l@addto@macro\svg@tempa{,scale=\relax}%
      \else%
        \l@addto@macro\svg@tempa{,scale=\svgscale}%
      \fi%
    \fi%
    \ifx\svg@tempa\@empty%
      \def\svg@tempa{\endgroup}%
    \else%
      \def\svg@tempb,##1\@nil{\edef\svg@tempa{##1}}%
      \expandafter\svg@tempb\svg@tempa\@nil%
      \PackageWarning{svg}{%
        You should specify the image size with parameters\MessageBreak%
        `width' and `height' or `scale' instead of using\MessageBreak%
        `\string\svgwidth' or `\string\svgscale'. Nevertheless,\MessageBreak%
        parameter(s) `\svg@tempa' will be set%
      }%
      \global\let\svgwidth\@undefined%
      \global\let\svgscale\@undefined%
      \edef\svg@tempa{%
        \endgroup%
        \noexpand\FamilyOptions{SVG}{\svg@tempa}%
      }%
    \fi%
  \svg@tempa%
}
%    \end{macrocode}
% \end{macro}^^A \svg@local@param@deprecated
%
% \subsubsection{Definition of user commands}
%
% \begin{macro}{\svgsetup}
% \changes{v2.00}{2017/02/17}{new}^^A
% \begin{macro}{\setsvg}
% \changes{v2.00}{2017/02/17}{deprecated}^^A
% The macro \cs{svgsetup} can be used to change options after loading the 
% package \pkg{svg} both in preamble and the document body. For compatibility 
% reasons, \cs{setsvg} is also defined.
% \ToDo{replace \cs{FamilyOptions\{SVG\}} with \cs{svgsetup}}[v2.03]
%    \begin{macrocode}
\newcommand*\svgsetup{\FamilyOptions{SVG}}
\newcommand*\setsvg{\FamilyOptions{SVG}}
%    \end{macrocode}
% \end{macro}^^A \setsvg
% \end{macro}^^A \svgsetup
% \begin{macro}{\svgpath}
% \changes{v2.00}{2017/02/12}{new}^^A
% \changes{v2.02}{2018/09/07}{%
%   parse argument for enclosing braces and provide if necessary%
% }^^A
% \begin{macro}{\svg@input@path}
% With \cs{svgpath} the user can give several root paths to SVG~files in the
% same way as \cs{graphicspath} is used. The only difference is that a missing
% slash is added at the end of the path, if needed.
%    \begin{macrocode}
\newcommand*\svg@input@path{}
\let\svg@input@path\input@path
\newcommand*\svgpath[1]{%
  \def\svg@tempa##1\@nil{%
    \ifx\svg@tempb\bgroup%
      \def\svg@input@path{#1}%
    \else%
      \def\svg@input@path{{#1}}%
    \fi%
  }%
  \futurelet\svg@tempb\svg@tempa#1\@nil%
}
%    \end{macrocode}
% \end{macro}^^A \svg@input@path
% \end{macro}^^A \svgpath
% \begin{macro}{\svg@out@file}
% These macros are used by \cs{includesvg} to compose the export file name.
%    \begin{macrocode}
\newcommand*\svg@out@file{}
%    \end{macrocode}
% \end{macro}^^A \svg@out@file
% \begin{macro}{\includesvg}
% \changes{v2.00}{2017/02/17}{changes, especially to optional parameters}^^A
% \changes{v2.02}{2018/09/07}{switched to \cs{svg@filename@parse}}^^A
% \changes{v2.03}{2019/07/26}{check for empty export file name}^^A
% For the inclusion of SVG~files the command \cs{includesvg} is defined.
%    \begin{macrocode}
\newcommand*\includesvg[2][]{%
  \begingroup%
%    \end{macrocode}
% \begin{parameter}{inkscape}
% \changes{v2.00}{2017/02/17}{new}^^A
% \begin{parameter}{inkscapeformat}
% \changes{v2.00}{2017/02/17}{new}^^A
% \begin{parameter}{inkscapelatex}
% \changes{v2.00}{2017/02/17}{new}^^A
% \begin{parameter}{inkscapearea}
% \changes{v2.00}{2017/02/17}{new}^^A
% \begin{parameter}{inkscapedpi}
% \changes{v2.00}{2017/02/17}{new}^^A
% \begin{parameter}{inkscapeopt}
% \changes{v2.00}{2017/02/17}{new}^^A
% \begin{parameter}{svgextension}
% \changes{v2.01}{2017/11/27}{new}^^A
% \begin{parameter}{width}
% \begin{parameter}{height}
% \changes{v2.00}{2017/02/17}{new}^^A
% \begin{parameter}{distort}
% \changes{v2.02}{2018/07/09}{new}^^A
% \begin{parameter}{scale}
% \changes{v2.00}{2017/02/17}{new}^^A
% \begin{parameter}{pretex}
% \begin{parameter}{apptex}
% \begin{parameter}{draft}
% \changes{v2.00}{2017/02/17}{new}^^A
% Most of the optional parameters have the same effect as the identically named
% options. Only parameter \prm{lastpage} is extended (see below). Moreover,
% there are some additional parameters, which can only be used as optional
% argument for \cs{includesvg} (\prm{angle} and \prm{origin}) but not as an 
% option. Now all parameters are set in local context (within a group).
%    \begin{macrocode}
    \svg@local@param@set{#1}%
%    \end{macrocode}
% \end{parameter}^^A draft
% \end{parameter}^^A apptex
% \end{parameter}^^A pretex
% \end{parameter}^^A scale
% \end{parameter}^^A distort
% \end{parameter}^^A height
% \end{parameter}^^A width
% \end{parameter}^^A svgextension
% \end{parameter}^^A inkscapeopt
% \end{parameter}^^A inkscapedpi
% \end{parameter}^^A inkscapearea
% \end{parameter}^^A inkscapelatex
% \end{parameter}^^A inkscapeformat
% \end{parameter}^^A inkscape
% Searching all given paths for the relevant SVG~file.
%    \begin{macrocode}
    \svg@get@file{#2}{\svg@file@ext}%
    \if@svg@file@found%
%    \end{macrocode}
% When a SVG~file was found, the resulting export file name is checked, if it 
% is not empty and a maybe given path is moved to \cs{svg@out@path}.
%    \begin{macrocode}
      \svg@normalize@name{svg}{inkscape}{svg@out}%
      \ifx\svg@out@name\@empty\else%
%    \end{macrocode}
% Running the export with \app{Inkscape} (if necessary) and checking the 
% required files for graphic inclusion.
%    \begin{macrocode}
        \edef\svg@out@file{\svg@out@path\svg@out@name.\svg@ink@format}%
        \svg@ink@run{\svg@file@base}{\svg@out@file}{\svg@ink@format}%
        \IfFileExists{\svg@out@file}{}{%
          \@svg@file@foundfalse%
          \svg@file@missing[\svg@file@base]{\svg@out@file}%
        }%
        \if@svg@ink@latex%
          \IfFileExists{\svg@out@file_tex}{}{%
            \@svg@file@foundfalse%
            \svg@file@missing[\svg@file@base]{\svg@out@file_tex}%
          }%
        \fi%
%    \end{macrocode}
% Include the resulting graphic file and maybe extract independent files.
%    \begin{macrocode}
        \if@svg@file@found%
          \begingroup%
            \svg@process{\svg@out@path}{\svg@out@name}{\svg@ink@format}%
          \endgroup%
        \fi%
      \fi%
    \fi%
  \endgroup%
}
%    \end{macrocode}
% \begin{parameter}{lastpage}
% \changes{v2.00}{2017/02/17}{new}^^A
% \changes{v2.03}{2019/07/25}{revised}^^A
% In addition to the automatic finding of the last page, which is included, it 
% can also be given directly as parameter.
% \ToDo{%
%   remove duplicated part of key \val{[.svg.sty]\{SVG\}\{lastpage\}}%
% }[KOMA 3.27]
%    \begin{macrocode}
\svg@local@param@def{%
  \DefineFamilyKey[.param]{SVG}{lastpage}[true]{%
    \FamilySetNumerical{SVG}{lastpage}{svg@tempa}{%
      {false}{0},{off}{0},{no}{0},{ignore}{0},%
      {true}{1},{on}{1},{yes}{1},{auto}{1}%
    }{#1}%
    \ifx\FamilyKeyState\FamilyKeyStateProcessed%
      \ifcase\svg@tempa\relax% false
        \FamilySetCounter{SVG}{lastpage}{svg@param@lastpage}{\m@ne}%
      \or% true
        \FamilySetCounter{SVG}{lastpage}{svg@param@lastpage}{\z@}%
      \fi%
    \else%
      \FamilySetCounter{SVG}{lastpage}{svg@param@lastpage}{#1}%
      \ifnum\value{svg@param@lastpage}<\@ne\relax%
        \FamilyKeyStateUnknownValue%
      \fi%
    \fi%
  }%
}
%    \end{macrocode}
% \end{parameter}^^A lastpage
% \begin{parameter}{angle}
% \changes{v2.00}{2017/02/17}{new}^^A
% \changes{v2.02}{2018/09/07}{validation of argument}^^A
% \begin{parameter}{origin}
% \changes{v2.00}{2017/02/17}{new}^^A
% The parameters \prm{angle} and \prm{origin} are definied as pendants to the 
% keys provided by \cs{includegraphics}.
%    \begin{macrocode}
\newcommand*\svg@param@angle{0}
\svg@local@param@def{%
  \DefineFamilyKey[.param]{SVG}{angle}{%
    \FamilyKeyStateUnknownValue%
    \ifisdimension{#1\p@}{%
      \renewcommand*\svg@param@angle{#1}%
      \FamilyKeyStateProcessed%
    }{}%
  }%
}
\newcommand*\svg@param@origin{c}
\svg@local@param@def{%
  \DefineFamilyKey[.param]{SVG}{origin}[c]{%
    \renewcommand*\svg@param@origin{#1}%
    \FamilyKeyStateProcessed%
  }%
}
%    \end{macrocode}
% \end{parameter}^^A origin
% \end{parameter}^^A angle
% \end{macro}^^A \includesvg
% \begin{macro}{\includeinkscape}
% \changes{v2.00}{2017/02/23}{new}^^A
% \changes{v2.02}{2018/09/07}{usage of \cs{svg@extension@parse}}^^A
% The command \cs{includeinkscape} can be used for including the export results 
% of \app{Inkscape}, if this part of the job was done in another way.
%    \begin{macrocode}
\newcommand*\includeinkscape[2][]{%
  \begingroup%
%    \end{macrocode}
% The given file extension is examined, where a known extension overwrites the 
% current setting for \opt{inkscapeformat}. If there's a suffix \file{\_tex}, 
% the option \opt{inkscapelatex} is set to \val{true} by default.
%    \begin{macrocode}
    \svg@filename@parse{#2}%
    \ifx\filename@ext\relax\else%
      \@svg@tempswafalse%
%    \end{macrocode}
% Within macro \cs{svg@tempb} the setup for a known extension will be stored. 
% If no extentsion will be found, the parsing of the file name will be done 
% again and any found extension will be considered to be part of the file name 
% itself (cs{\filename@base}).
%    \begin{macrocode}
      \def\svg@tempb{\svg@filename@parse[\relax]{#2}}%
      \@tfor\svg@tempa:={pdf}{eps}{ps}{png}\do{%
%    \end{macrocode}
% If the found extension is equal to a known one, \opt{inkscapeformat} is set 
% and the extension is appended to \cs{filename@base} in order to parse it 
% again with \cs{svg@get@file} and preserve the letter case.
%    \begin{macrocode}
        \svg@ifstr@nocase{\svg@tempa}{\filename@ext}{%
          \edef\svg@tempb{%
            \noexpand\FamilyOptions{SVG}{inkscapeformat=\svg@tempa}%
            \def\noexpand\filename@base{\filename@base.\filename@ext}%
          }%
          \@svg@tempswatrue%
        }{%
%    \end{macrocode}
% If a known extension with appended part \file{\_tex} is found, it is used as
% well. Additionally, \opt{inkscapelatex}|=|\val{true} is set. For preserving
% the letter case, the first part of \cs{filename@ext} is parsed and appended
% to \cs{filename@base}.
%    \begin{macrocode}
          \svg@ifstr@nocase{\svg@tempa_tex}{\filename@ext}{%
            \def\svg@tempb##1_##2\@nil{\def\svg@tempb{##1}}%
            \expandafter\svg@tempb\filename@ext\@nil%
            \edef\svg@tempb{%
              \noexpand\FamilyOptions{SVG}{inkscapeformat=\svg@tempa}%
              \noexpand\FamilyOptions{SVG}{inkscapelatex=true}%
              \def\noexpand\filename@base{\filename@base.\svg@tempb}%
            }%
            \@svg@tempswatrue%
          }{}%
        }%
%    \end{macrocode}
% Break for loop, if valid extension was found.
%    \begin{macrocode}
        \if@svg@tempswa%
          \@break@tfor%
        \fi%
      }%
      \svg@tempb%
    \fi%
%    \end{macrocode}
% \begin{parameter}{inkscapeformat}
% \begin{parameter}{inkscapelatex}
% \begin{parameter}{width}
% \begin{parameter}{height}
% \begin{parameter}{distort}
% \begin{parameter}{scale}
% \begin{parameter}{pretex}
% \begin{parameter}{apptex}
% \begin{parameter}{draft}
% \begin{parameter}{lastpage}
% \begin{parameter}{angle}
% \begin{parameter}{origin}
% Parameters, which are supported by \cs{includesvg}, can also be used with 
% \cs{includeinkscape} even if some of them---more precisely those that control 
% the export with \app{Inkscape}---don't have an effect at all. Nevertheless, 
% they are set right now in local context (within a group).
%    \begin{macrocode}
    \svg@local@param@set{#1}%
%    \end{macrocode}
% \end{parameter}^^A origin
% \end{parameter}^^A angle
% \end{parameter}^^A lastpage
% \end{parameter}^^A draft
% \end{parameter}^^A apptex
% \end{parameter}^^A pretex
% \end{parameter}^^A scale
% \end{parameter}^^A distort
% \end{parameter}^^A height
% \end{parameter}^^A width
% \end{parameter}^^A inkscapelatex
% \end{parameter}^^A inkscapeformat
% Searching all given paths for the relevant PDF/EPS~file.
%    \begin{macrocode}
    \svg@get@file[\svg@out@path]{\filename@area\filename@base}{\svg@ink@format}%
    \if@svg@file@found%
%    \end{macrocode}
% Checking the required files for graphic inclusion.
%    \begin{macrocode}
      \if@svg@ink@latex%
        \IfFileExists{\svg@file@base_tex}{}{%
          \@svg@file@foundfalse%
          \svg@file@missing[\svg@file@base]{\svg@file@base_tex}%
        }%
      \fi%
%    \end{macrocode}
% Include the resulting graphic file and maybe extract independent files.
%    \begin{macrocode}
      \if@svg@file@found%
        \begingroup%
          \svg@process{\svg@file@path}{\svg@file@name}{\svg@ink@format}%
        \endgroup%
      \fi%
    \fi%
  \endgroup%
}
%    \end{macrocode}
% \end{macro}^^A \includeinkscape
%
% \iffalse
%</body>
%<*option>
% \fi
%
%
%
% \subsection{Auxiliary macros}
% \ToDo{sort genreal helper macros for both packages  }[v2.03]
% \ToDo{move general macros up to "Helper macros"}[v2.03]
%
% \begin{macro}{\svg@lowercase}
% \changes{v2.03}{2019/07/25}{new}^^A
% Defines the second argument to the lowercase expansion of the first argument, 
% if given. Otherwise, the expansion of the second argument itself is 
% lowercased.
%    \begin{macrocode}
\newcommand*\svg@lowercase[2][]{%
  \begingroup%
    \edef\svg@tempa{%
      \endgroup%
      \lowercase{\noexpand\def\noexpand#2{\IfArgIsEmpty{#1}{#2}{#1}}}%
    }%
  \svg@tempa%
}
%    \end{macrocode}
% \end{macro}^^A \svg@lowercase
% \begin{macro}{\svg@ifstr@nocase}
% \changes{v2.03}{2019/08/29}{new}^^A
% Compares two string like \cs{ifstr} from \pkg {scrbase} but case-insensitive 
% and ignoring any quotes. The arguments are expanded in a macro in order to 
% prevent overwriting of \cs{svg@tempa} or \cs{svg@tempb} if used in argument.
%    \begin{macrocode}
\newcommand*\svg@ifstr@nocase[2]{%
  \begingroup%
    \edef\svg@tempa{%
      \noexpand\svg@quotes@remove[{#1}]{\noexpand\svg@tempa}%
      \noexpand\svg@quotes@remove[{#2}]{\noexpand\svg@tempb}%
    }%
    \svg@tempa%
    \edef\svg@tempa{%
      \endgroup%
      \lowercase{\noexpand\ifstr{\svg@tempa}{\svg@tempb}}%
    }%
  \svg@tempa%
}
%    \end{macrocode}
% \end{macro}^^A \svg@ifstr@nocase
% \begin{macro}{\svg@remove@leadingchar}
% \changes{v2.02}{2018/09/07}{new}^^A
% This command removes the single character in given with the first argument 
% from the expanded macro in the second argument.
%    \begin{macrocode}
\newcommand*\svg@remove@leadingchar[2]{%
  \begingroup%
    \svg@sanitize@dq\svg@tempa{#2}%
    \def\svg@tempb{%
      \def\svg@tempa####1\@nil{\def\svg@tempa{####1}}%
      \kernel@ifnextchar#1%
        {\expandafter\svg@tempa\@gobble}%
        {\svg@tempa}%
    }%
    \expandafter\svg@tempb\svg@tempa\@nil%
    \edef\svg@tempb{%
      \endgroup%
      \def\noexpand#2{\svg@tempa}%
    }%
  \svg@tempb%
}
%    \end{macrocode}
% \end{macro}^^A \svg@remove@leadingchar
% \begin{macro}{\svg@ifnumber}
% \changes{v2.03}{2019/08/20}{new}^^A
% Checks for a positive integer in the expanded argument.
%    \begin{macrocode}
\newcommand*\svg@ifnumber[1]{%
  \begingroup%
    \edef\svg@tempa{#1}%
    \def\svg@tempb{\aftergroup\@secondoftwo}%
    \ifisinteger{\svg@tempa}{%
      \ifnumber{\svg@tempa}{%
        \def\svg@tempb{\aftergroup\@firstoftwo}%
      }{}%
    }{}%
    \svg@tempb%
  \endgroup%
}
%    \end{macrocode}
% \end{macro}^^A \svg@ifnumber
% \begin{macro}{\svg@IfFileNewer}
% \changes{v2.03}{2019/07/18}{Xe\TeX{} provides \cs{filemoddate}}^^A
% The macro \cs{svg@IfFileNewer} is used to decide, whether the export with 
% \app{Inkscape} is necessary due to an updated SVG~file. This is checked with 
% either \cs{pdf@filemoddate} (pdf\TeX{} and Lua\TeX{}) or \cs{filemoddate} 
% (Xe\TeX{}).
%    \begin{macrocode}
\ifx\pdf@filemoddate\@undefined
  \ifx\filemoddate\@undefined\else
    \ifx\strcmp\@undefined\else
      \newcommand*\svg@IfFileNewer[2]{%
        \begingroup%
          \edef\svg@tempa{\filemoddate{#1}}%
          \edef\svg@tempb{\filemoddate{#2}}%
          \ifnum\strcmp{\svg@tempa}{\svg@tempb}>\z@\relax%
            \aftergroup\@firstoftwo%
          \else%
            \aftergroup\@secondoftwo%
          \fi%
        \endgroup%
      }
    \fi
  \fi
\else
  \ifx\pdf@strcmp\@undefined\else
    \newcommand*\svg@IfFileNewer[2]{%
      \begingroup%
        \edef\svg@tempa{\pdf@filemoddate{#1}}%
        \edef\svg@tempb{\pdf@filemoddate{#2}}%
        \ifnum\pdf@strcmp{\svg@tempa}{\svg@tempb}>\z@\relax%
          \aftergroup\@firstoftwo%
        \else%
          \aftergroup\@secondoftwo%
        \fi%
      \endgroup%
    }
  \fi
\fi
%    \end{macrocode}
% \end{macro}^^A \svg@IfFileNewer
%
% \iffalse
%</option>
%<*body>
% \fi
%
% \begin{macro}{\svg@ink@run}
% \changes{v2.03}{2019/06/25}{%
%   exported files are not moved but created directly in \cs{svg@out@path}
% }^^A
% \begin{macro}{\if@svg@ink@run}
% The command, which performs the call of \app{Inkscape} via \cs{ShellEscape}.
%    \begin{macrocode}
\newif\if@svg@ink@run
\newcommand*\svg@ink@run[3]{%
  \ifcase\svg@ink@mode\relax% false
    \@svg@ink@runfalse%
  \or% true
%    \end{macrocode}
% If the mode for \opt{inkscape} was set to \val{true}, some checks are
% performed to detect, if a run of \app{Inkscape} is actually necessary. This 
% is the case when the SVG~file is newer than the corresponding exported file,
% or if the latter isn't present at all.
%    \begin{macrocode}
    \svg@IfFileNewer{#1}{#2}%
      {\@svg@ink@runtrue}{\@svg@ink@runfalse}%
%    \end{macrocode}
% The same is true, when the associated \LaTeX~file is missing. But when this 
% file already exists, maybe the user did some changes to this file. In this 
% case, overwriting this file is maybe not intended.
%    \begin{macrocode}
    \if@svg@ink@latex%
      \IfFileExists{#2_tex}{%
        \if@svg@ink@run\ifnum\pdf@shellescape=\@ne\relax%
          \svg@IfFileNewer{#2_tex}{#2}{%
            \@svg@ink@runfalse%
            \PackageWarning{svg}{%
              Since the encountered filedate of file\MessageBreak%
              `#2_tex'\MessageBreak%
              is newer than\MessageBreak%
              `#2'\MessageBreak%
              it's supposed that you customized this file. To\MessageBreak%
              avoid an accidental overwriting of this file, the\MessageBreak%
              Inkscape export won't be done. If you want to\MessageBreak%
              overwrite the existing file, please set parameter\MessageBreak%
              `inkscape=force'%
            }%
          }{}%
        \fi\fi%
      }{\@svg@ink@runtrue}%
    \fi%
%    \end{macrocode}
% If the mode for \opt{inkscape} was set to \val{forced}, \app{Inkscape} will 
% be called in any case.
%    \begin{macrocode}
  \or% forced
    \@svg@ink@runtrue%
  \fi%
%    \end{macrocode}
% If all checks were positive, the export with \app{Inkscape} can be done in 
% case \file{-{}-shell-escape} is enabled.
%    \begin{macrocode}
  \if@svg@ink@run%
    \ifnum\pdf@shellescape=\@ne\relax%
%    \end{macrocode}
% For exporting PNG~files, the used density ist set to \val{300dpi}, if no
% value was given.
%    \begin{macrocode}
      \ifstr{#3}{png}{%
        \ifx\svg@ink@dpi\relax%
          \FamilyOptions{SVG}{inkscapedpi=300}%
        \fi%
      }{}%
%    \end{macrocode}
% Executing \app{Inkscape} on command line. Afterwards, the export results are 
% moved into the given output path.
%    \begin{macrocode}
      \svg@shell@mkdir{\svg@out@path}%
      \PackageInfo{svg}{%
        Exporting `#1'%
        \ifx\svg@ink@opt\@empty\else\space with added options `\svg@ink@opt'\fi%
      }%
%    \end{macrocode}
% Expanded definition in order to prevent overwriting of \cs{svg@tempa} or 
% \cs{svg@tempb} if given as argument.
%    \begin{macrocode}
      \begingroup%
        \edef\svg@tempa{%
          \noexpand\svg@quotes@remove[{#1}]{\noexpand\svg@tempa}%
          \noexpand\svg@quotes@remove[{#2}]{\noexpand\svg@tempb}%
        }%
        \svg@tempa%
        \edef\svg@tempa{%
          \endgroup%
          \noexpand\ShellEscape{%
            \noexpand\svg@ink@cmd{\svg@tempa}{\svg@tempb}{#3}%
          }%
        }%
      \svg@tempa%
      \IfFileExists{#2}{}{\svg@troubleshooting{svg}{Exporting}{#1}{#2}}%
%    \end{macrocode}
% If \file{-{}-shell-escape} wasn't enabled, a warning is issued.
%    \begin{macrocode}
    \else%
      \PackageWarning{svg}{%
        You didn't enable `--shell-escape' option so it\MessageBreak%
        wasn't possible to launch the Inkscape export for\MessageBreak%
        `#1'%
      }%
    \fi%
  \fi%
}
%    \end{macrocode}
% \end{macro}^^A \if@svg@ink@run
% \end{macro}^^A \svg@ink@run
% \begin{macro}{\svg@normalize@name}
% \changes{v2.03}{2019/09/02}{new}^^A
% This macro is used to normalize a given name of an output file. It's intended 
% to use it for given names via \opt{inkscapename} or \opt{extractname} and the 
% according paths \opt{inkscapepath} or \opt{extractpath}. Any given path 
% in front of a file name is passed to the corresponding \opt{\dots path} 
% option. If the expansion of the name is empty, an error is thrown.
%    \begin{macrocode}
\newcommand*\svg@normalize@name[3]{%
  \begingroup%
    \svg@sanitize@dq\svg@tempa{\csname #3@name\endcsname}%
    \svg@filename@parse[\relax]{\svg@tempa}%
    \ifx\filename@area\@empty\else%
      \PackageWarning{#1}{%
        You've tried to setup `#2name=\svg@tempa'\MessageBreak%
        with a preceeded path. This is not recommended. You\MessageBreak%
        should use `#2path=\filename@area'\MessageBreak%
        and `#2name=\filename@base'\MessageBreak%
        instead, which is done now%
      }%
    \fi%
    \ifx\filename@base\@empty%
      \PackageError{#1}{`#2name' is empty}{%
        The file name given with option `#2name' is empty.%
      }%
    \fi%
    \edef\svg@tempb{%
      \endgroup%
      \ifx\filename@area\@empty\else%
        \noexpand\@namedef{#3@path}{\filename@area}%
      \fi%
      \noexpand\@namedef{#3@name}{\filename@base}%
    }%
  \svg@tempb%
}
%    \end{macrocode}
% \end{macro}^^A \svg@normalize@name
% \begin{macro}{\svg@troubleshooting}
% \changes{v2.03}{2019/08/22}{new}^^A
% This macro is used to generate a specific warning with some hints for 
% troubleshooting, if probably something during the invocation of an external
% programm went wrong and so the process of exporting/extracting/conversion 
% was not finished sucessfully.
%    \begin{macrocode}
\newcommand*\svg@troubleshooting[4]{%
  \PackageWarning{#1}{%
    #2 from input file\MessageBreak`#3'\MessageBreak%
    failed for output file\MessageBreak`#4'\MessageBreak%
    Troubleshooting: Check `\jobname.log' for\MessageBreak%
    > Package #1 Info:\MessageBreak%
    > #2 `#3'\MessageBreak%
    and try to examine occuring problems by executing\MessageBreak%
    following invocations manually%
  }%
}
%    \end{macrocode}
% \end{macro}^^A \svg@troubleshooting
% \begin{macro}{\svg@ink@ver@settings}
% \changes{v2.03}{2019/08/20}{new}^^A
% \begin{macro}{\svg@ink@ver@detect}
% \changes{v2.03}{2019/06/28}{new}^^A
% \begin{macro}{\svg@ink@ver@@detect}
% \changes{v2.03}{2019/08/20}{new}^^A
% \begin{macro}{\if@svg@ink@ver@detect}
% \changes{v2.03}{2019/06/28}{new}^^A
% As the interface of the command line tool of \app{Inkscape} has changed 
% between versions \val{0.x} and \val{1.x}, we have to check the used version 
% in order to define the calling macro \cs{svg@ink@cmd}. The found version is 
% stored in \cs{svg@ink@ver}, whereas the following meanings are applied:
% \begin{description}
%   \item[\val{-1}] version check has not be done or \app{Inkscape} could not 
%     be found/executed
%   \item[\val{0}] \app{Inkscape} version \val{0.x} was found
%   \item[\val{1}] \app{Inkscape} version \val{1.x} or later was found
% \end{description}
%
% All necessary information are stored within \cs{svg@ink@ver@settings} as 
% three arguments, whereas the first one is the manually set version, the 
% second is the used \opt{inkscapeexe} for automatic detection and the third 
% one is the detected version itself.
%    \begin{macrocode}
\newcommand*\svg@ink@ver@settings{{\svg@ink@ver}{\svg@ink@exe}{\m@ne}}
\newif\if@svg@ink@ver@detect
%    \end{macrocode}
% In order to run the check for \app{Inkscape} version at the beginning of the 
% document only if needed, the version stored in \cs{svg@ink@ver} is written to 
% the aux-file. As a change of the given \opt{inkscapeexe} must also enforce 
% the version check as well as a change of \cs{svg@ink@ver}, these values are 
% also written.
%    \begin{macrocode}
\newcommand*\svg@ink@ver@detect[3]{%
%    \end{macrocode}
% The first two arguments to \cs{svg@ink@ver@settings} are temporary stored in 
% \cs{svg@tempb} as they are maybe manipulated in further process.
%    \begin{macrocode}
  \svg@sanitize@dq\svg@tempb{{\svg@ink@ver}{#2}}%
  \@svg@ink@ver@detectfalse%
  \ifnum\svg@ink@ver=\m@ne\relax%
    \ifnum\pdf@shellescape=\@ne\relax%
%    \end{macrocode}
% If \opt{inkscapeexe} was not changed\dots
%    \begin{macrocode}
      \svg@sanitize@dq\svg@tempa{#2}%
      \ifx\svg@tempa\svg@ink@exe%
%    \end{macrocode}
% \dots then enforce the check after a change of mode to \val{detect}\dots
%    \begin{macrocode}
        \ifnum#1>\m@ne\relax%
          \@svg@ink@ver@detecttrue%
%    \end{macrocode}
% \dots or if detection was never invoked, do so.
%    \begin{macrocode}
        \else%
          \ifnum#3=\m@ne\relax%
            \@svg@ink@ver@detecttrue%
          \fi%
        \fi%
%    \end{macrocode}
% Enforce the check after a change of \opt{inkscapeexe} and store the new one 
% in \cs{svg@tempb}.
%    \begin{macrocode}
      \else%
        \edef\svg@tempb{{\svg@ink@ver}{\svg@ink@exe}}%
        \@svg@ink@ver@detecttrue%
      \fi%
    \fi%
  \fi%
%    \end{macrocode}
% Run detection if necessary and store the detected version to 
% \cs{svg@ink@ver@settings}. Otherwise set previous detected version in 
% automatic mode.
%    \begin{macrocode}
  \if@svg@ink@ver@detect%
    \svg@ink@ver@@detect%
    \edef\svg@ink@ver@settings{\svg@tempb{\svg@ink@ver}}%
%    \end{macrocode}
% If no version was found, an error message is shown.
%    \begin{macrocode}
    \svg@ifnumber{\svg@ink@ver}{}{%
      \PackageErrorNoLine{svg}{Inkscape version not found}{%
        It was tried to invoke `\svg@ink@exe\space-V'\MessageBreak%
        but no version info for Inkscape could gathered. Please\MessageBreak%
        set it up manually with `inkscapeversion=<version>'%
      }%
    }%
  \else%
    \ifnum\svg@ink@ver=\m@ne\relax%
      \def\svg@ink@ver{#3}%
    \fi%
    \edef\svg@ink@ver@settings{\svg@tempb{#3}}%
  \fi%
}
%    \end{macrocode}
% The detection is only done with extended shell access. Therefore, 
% \cs{if@svg@ink@ver@detect} is used and set by \cs{svg@ink@settings}. The
% call of \app{Inkscape} stored in \cs{svg@ink@exe} is done with 
% \cs{@@input\textbar"'\dots'" -V} in order to read the result from
% \val{stdout}.
% \ToDo{%
%   \url{https://tex.stackexchange.com/questions/9792}%
%   \val{-enable-pipes} in MikTeX?!
% }[v2.03]
%    \begin{macrocode}
\newcommand*\svg@ink@ver@@detect{%
  \begingroup%
    \endlinechar=\m@ne%
    \everyeof{\noexpand}%
    \svg@quotes@remove{\svg@ink@exe}%
    \edef\svg@tempa{%
      \edef\noexpand\svg@tempa{%
        \noexpand\@@input|"'\svg@ink@exe'\space-V" %
      }%
    }%
    \svg@tempa%
%    \end{macrocode}
% The found version is stored in \cs{svg@tempa} and parsed afterwards.
%    \begin{macrocode}
    \def\svg@tempb Inkscape ##1.##2\@nil{%
      \gdef\svg@ink@ver{##1}%
    }%
    \expandafter\svg@tempb\svg@tempa Inkscape \m@ne.\@nil%
  \endgroup%
}
%    \end{macrocode}
% Comparing the stored settings from last the last run with current settings.
%    \begin{macrocode}
\AtBeginDocument{\expandafter\svg@ink@ver@detect\svg@ink@ver@settings}
%    \end{macrocode}
% Writing \cs{svg@ink@exe} and \cs{svg@ink@ver} to the main aux-file.
%    \begin{macrocode}
\BeforeClosingMainAux{%
  \if@filesw%
    \immediate\write\@mainaux{%
      \string\gdef\string\svg@ink@ver@settings{\svg@ink@ver@settings}%
    }%
  \fi%
}
%    \end{macrocode}
% \end{macro}^^A \if@svg@ink@ver@detect
% \end{macro}^^A \svg@ink@ver@@detect
% \end{macro}^^A \svg@ink@ver@detect
% \end{macro}^^A \svg@ink@ver@settings
% \begin{macro}{\svg@ink@cmd}
% \changes{v2.03}{2019/06/28}{distinguish \app{Inkscape} versions}^^A
% The actual call of \app{Inkscape} at command line.
%    \begin{macrocode}
\newcommand*\svg@ink@cmd[3]{%
  \svg@ink@exe\space"#1"\space%
  --without-gui\space\svg@ink@area\space%
  \ifx\svg@ink@dpi\relax\else--export-dpi=\svg@ink@dpi\space\fi%
  \if@svg@ink@latex--export-latex\space\fi%
  \ifx\svg@ink@opt\@empty\else\svg@ink@opt\space\fi%
  \ifnum\svg@ink@ver<\@ne%
    --export-#3="#2"\space%
  \else%
    --export-type=#3\space%
    --export-file="#2"\space%
  \fi%
}
%    \end{macrocode}
% \end{macro}^^A \svg@ink@cmd
% \begin{macro}{\svg@get@lastpage}
% This macro is used to circumvent the multiple pages bug for PDF~files of
% \app{Inkscape}~0.91, when the the \LaTeX~export was enabled. For this 
% purpose, the total page number is read from the PDF~file. 
%    \begin{macrocode}
\newcommand*\svg@get@lastpage[1]{%
  \begingroup%
    \@tempcnta=\m@ne\relax%
    \ifx\XeTeXpdfpagecount\@undefined%
      \ifpdf%
        \ifx\pdfximage\@undefined%
          \ifx\saveimageresource\@undefined\else%
            \saveimageresource{#1}%
            \@tempcnta=\lastsavedimageresourcepages\relax%
          \fi%
        \else%
          \pdfximage{#1}%
          \@tempcnta=\pdflastximagepages\relax%
        \fi%
      \fi%
    \else%
      \@tempcnta=\XeTeXpdfpagecount#1\relax%
    \fi%
    \ifnum\@tempcnta=\m@ne\relax%
      \PackageWarning{svg}{%
        It wasn't possible to detect the last page\MessageBreak%
        of `#1'%
      }%
    \else%
      \PackageInfo{svg}{Last page of `#1'\MessageBreak is `\the\@tempcnta'}%
    \fi%
    \edef\svg@tempa{%
      \endgroup%
      \noexpand\FamilyExecuteOptions[.param]{SVG}{lastpage=\the\@tempcnta}%
    }%
  \svg@tempa%
}
%    \end{macrocode}
% \end{macro}^^A \svg@get@lastpage
% \begin{macro}{\svg@transparent}
% \changes{v2.03}{2019/07/10}{new}^^A
% \begin{macro}{\svg@transparent@saved}
% \changes{v2.03}{2019/07/10}{new}^^A
% If package \pkg{transparent} was not or could not be loaded---currently 
% Xe\TeX{} isn't supported---and macro \cs{transparent} is used anyway within 
% an associated \LaTeX~file, normally an error is raised. This workaround fixes 
% this issue and creates a warning, for every included file only once. As the 
% associated \LaTeX~file uses a group, the redefintions have to be global.
%    \begin{macrocode}
\newcommand*\svg@transparent@saved{}
\newcommand*\svg@transparent[1]{%
  \PackageWarning{svg}{%
    The command \string\transparent\space does not exist yet,\MessageBreak%
    so it is defined as a gobbling auxiliary macro by now. \MessageBreak%
    Maybe you should load package `transparent'\MessageBreak%
  }%
  \global\let\transparent\@gobble%
}
%    \end{macrocode}
% \end{macro}^^A \svg@transparent@saved
% \end{macro}^^A \svg@transparent
% \begin{macro}{\svg@param@size@check}
% \changes{v2.03}{2019/08/21}{new}^^A
% \begin{macro}{\svg@param@size@wrn}
% \changes{v2.03}{2019/07/23}{more specifig warning message}^^A
% The option \opt{scale} respectively the parameter \prm{scale} is only 
% considered if the size was not specified.
%    \begin{macrocode}
\newcommand*\svg@param@size@check{%
  \begingroup%
    \ifdim\dimexpr\svg@param@scale\p@\relax=\p@\relax\else%
      \let\svg@tempb\@empty%
      \@for\svg@tempa:={scale,width,height}\do{%
        \edef\svg@tempb{%
          \etex@unexpanded\expandafter{\svg@tempb}%
          {\svg@tempa}{\csname svg@param@\svg@tempa\endcsname}%
        }%
      }%
      \expandafter\svg@param@size@wrn\svg@tempb%
    \fi%
  \endgroup%
}
\newcommand*\svg@param@size@wrn[6]{%
  \begingroup%
    \ifdim\dimexpr#2\p@\relax=\p@\relax\else%
      \let\svg@tempa\@empty%
      \ifdim#4>\z@\relax%
        \l@addto@macro\svg@tempa{#3=#4}%
      \fi%
      \ifdim#6>\z@\relax%
        \ifx\svg@tempa\@empty\else%
          \l@addto@macro\svg@tempa{,}%
        \fi%
        \l@addto@macro\svg@tempa{#5=#6}%
      \fi%
      \ifx\svg@tempa\@empty\else%
        \PackageWarning{svg}{%
          The parameter `#1=#2' is not applied\MessageBreak%
          because `\svg@tempa'\MessageBreak%
          is given as well%
        }%
      \fi%
    \fi%
  \endgroup%
}
%    \end{macrocode}
% \end{macro}^^A \svg@param@size@check
% \end{macro}^^A \svg@param@size@wrn
% \begin{macro}{\svg@process}
% The macro \cs{svg@process} is defined in order to realize the additional 
% features of package \pkg{svg-extract} like extraction and conversion of 
% independent graphic files including option \opt{exclude} as well as macro 
% \cs{includesvgx}. Therefore it is redefined later.
%    \begin{macrocode}
\newcommand*\svg@process[3]{\svg@ink@include{#1}{#2}{#3}}
%    \end{macrocode}
% \end{macro}^^A \svg@process
% \begin{macro}{\svg@ink@include}
% With macro \cs{svg@ink@include} the export result of \app{Inkscape} is 
% included. The macro \cs{svg@set@input@path} is called to support input
% commands like \cs{input}\marg{tex filename} within SVG~files.
%    \begin{macrocode}
\newcommand*\svg@ink@include[3]{%
  \svg@param@size@check%
  \svg@set@input@path%
%    \end{macrocode}
% If the export with \app{Inkscape} was done with \LaTeX~support enabled, the 
% corresponding file will be used together with \cs{input}. The necessary 
% patches to environment \env{picture} as well as command \cs{includegraphics} 
% are made beforehand with \cs{svg@patches}.
%    \begin{macrocode}
  \@svg@tempswatrue%
  \if@svg@draft%
    \@svg@tempswafalse%
  \fi%
  \if@svg@ink@latex\else%
    \@svg@tempswafalse%
  \fi%
%    \end{macrocode}
% In order to support file names with multiple dots, \cs{svg@tempa} is defined 
% with the file name itself within enclosing braces followed by the extension.
%    \begin{macrocode}
  \edef\svg@tempa{{#1#2}.#3}%
  \if@svg@tempswa%
    \svg@patches{\svg@tempa}%
%    \end{macrocode}
% After setting the file name for \cs{includegraphics}, it is stored without 
% enclosing braces and quotes in \cs{svg@tempa} .
%    \begin{macrocode}
    \svg@quotes@remove[\expandafter\@firstofone\svg@tempa]{\svg@tempa}%
%    \end{macrocode}
% If \pkg{graphicx} is in draft mode, the first page is sufficient. Otherwise, 
% the last page is detected if set so.
%    \begin{macrocode}
    \ifGin@draft%
      \FamilySetCounter{SVG}{lastpage}{svg@param@lastpage}{1}%
    \else%
      \ifstr{#3}{pdf}{%
        \ifnum\value{svg@param@lastpage}=\z@\relax%
          \edef\svg@tempb{\noexpand\svg@get@lastpage{"\svg@tempa"}}%
          \svg@tempb%
        \fi%
      }{}%
    \fi%
    \edef\svg@tempa{%
      \ifx\svg@param@pretex\relax\else%
        \noexpand\svg@param@pretex%
      \fi%
      \noexpand\input{"\svg@tempa_tex"}%
      \ifx\svg@param@apptex\relax\else%
        \noexpand\svg@param@apptex%
      \fi%
    }%
%    \end{macrocode}
% If \opt{distort=true} is desired, the input is resized with \cs{resizebox*}.
%    \begin{macrocode}
    \if@svg@param@distort%
      \def\svg@tempb{\resizebox*{\svg@param@width}{\svg@param@height}}%
    \else%
      \let\svg@tempb\@firstofone%
    \fi%
%    \end{macrocode}
% The input is stored in a box in order to do a possible rotation afterwards. 
% Additionally, the workaround for the maybe missing macro \cs{transparent} is 
% applied.
%    \begin{macrocode}
    \global\let\svg@transparent@saved\transparent%
    \providecommand*\transparent[1]{\svg@transparent{##1}}%
    \sbox\svg@box{\svg@tempb{\svg@tempa}}%
    \global\let\transparent\svg@transparent@saved%
%    \end{macrocode}
% If a rotation angle was given, the input is done within \cs{rotatebox}.
%    \begin{macrocode}
    \ifdim\dimexpr\svg@param@angle\p@\relax=\z@\relax%
      \let\svg@tempb\@firstofone%
    \else%
      \edef\svg@tempb{%
        \noexpand\rotatebox[origin=\svg@param@origin]{\svg@param@angle}%
      }%
    \fi%
%    \end{macrocode}
% The output of the resulting image stored in \cs{svg@box}.
%    \begin{macrocode}
    \svg@tempb{\usebox\svg@box}%
  \else%
%    \end{macrocode}
% If the export with \app{Inkscape} was done without \LaTeX~support or 
% parameter \prm{draft} was given, the resulting graphic file will be included
% with \cs{includegraphics}.
%    \begin{macrocode}
    \def\svg@tempb{%
      draft\if@svg@draft\else=false\fi%
    }%
    \ifdim\svg@param@width>\z@\relax%
      \l@addto@macro\svg@tempb{,width=\svg@param@width}%
    \fi%
    \ifdim\svg@param@height>\z@\relax%
      \l@addto@macro\svg@tempb{,height=\svg@param@height}%
    \fi%
    \l@addto@macro\svg@tempb{,keepaspectratio\if@svg@param@distort=false\fi}%
    \ifdim\dimexpr\svg@param@scale\p@\relax=\p@\relax\else%
      \l@addto@macro\svg@tempb{,scale=\svg@param@scale}%
    \fi%
    \ifdim\dimexpr\svg@param@angle\p@\relax=\z@\relax\else%
      \l@addto@macro\svg@tempb{%
        ,origin=\svg@param@origin,angle=\svg@param@angle%
      }%
    \fi%
    \edef\svg@tempb{\svg@tempb}%
    \expandafter\includegraphics\expandafter[\svg@tempb]{\svg@tempa}%
  \fi%
}
%    \end{macrocode}
% \end{macro}^^A \svg@ink@include
%
%
%
% \subsection{Patches}
%
% \begin{macro}{\svg@patches}
% \changes{v2.02b}{2018/11/12}{%
%   fix bug for package \pkg{tikzscale}: store original definitions of 
%   \cs{picture} and \cs{includegraphics} right after loading package \pkg{svg}
% }^^A
% \begin{macro}{\svg@picture@saved}
% \begin{macro}{\svg@includegraphics@saved}
% For including the export results from \app{Inkscape} with \LaTeX~support 
% enabled, there are some patches necessary for environment \env{picture} and 
% \cs{includegraphics}. These patches are done with \cs{svg@patches}.
%    \begin{macrocode}
\newcommand*\svg@picture@saved{}
\let\svg@picture@saved\picture
\newcommand*\svg@includegraphics@saved{}
\let\svg@includegraphics@saved\includegraphics
\newcommand*\svg@patches[1]{%
  \let\picture\svg@picture@patched%
  \let\includegraphics\svg@includegraphics@patched%
  \edef\svg@includegraphics@file{#1}%
}
%    \end{macrocode}
% \end{macro}^^A \svg@includegraphics@saved
% \end{macro}^^A \svg@picture@saved
% \end{macro}^^A \svg@patches
% \begin{macro}{\svg@pictur@patched}
% In order to provide the possibility specify the desired width of a graphic, 
% the appropriate \cs{unitlength} is calculated at the beginning of the 
% \env{picture}~environment.
%    \begin{macrocode}
\newcommand*\svg@picture@patched{}
\newcommand*\svg@pictur@patched{}
\long\def\svg@picture@patched#1{\svg@pictur@patched@#1}
\def\svg@pictur@patched@(#1,#2){%
%    \end{macrocode}
% If a desired height is present, the resulting \cs{unitlength} is calculated 
% with the ratio of the coordinates of the \env{picture}~environment given as 
% arguments for x- and y-direction by using \cs{Gscale@div}. With this factor, 
% \cs{unitlength}---which is connected to the x-coordinate---can be scaled in a 
% suitable manner.
%    \begin{macrocode}
  \ifdim\svg@param@height>\z@\relax%
    \Gscale@div\svg@tempa{#1\p@}{#2\p@}%
    \setlength\unitlength{\svg@param@height}%
    \setlength\unitlength{\svg@tempa\unitlength}%
    \ifdim\svg@param@width>\z@\relax%
      \ifdim\unitlength>\svg@param@width\relax%
        \setlength\unitlength{\svg@param@width}%
      \fi%
    \fi%
  \else%
%    \end{macrocode}
% If no height is given, \cs{unitlength} can be set easily.
%    \begin{macrocode}
    \ifdim\svg@param@width>\z@\relax%
      \setlength\unitlength{\svg@param@width}%
    \else%
      \setlength\unitlength{\svg@param@scale\unitlength}%
    \fi%
  \fi%
%    \end{macrocode}
% After setting \cs{unitlength}, the \env{picture}~environment can be called 
% with its original definition.
%    \begin{macrocode}
  \svg@picture@saved(#1,#2)%
}
%    \end{macrocode}
% \end{macro}^^A \svg@pictur@patched
% \begin{macro}{\svg@includegraphics@patched}
% \begin{macro}{\svg@includegraphics@file}
% The patch to \cs{includegraphics} is meant to dissolve the \app{Inkscape} bug 
% concerning the inclusion of more PDF~pages than actually are existing.
%
% The parameters for \cs{includegraphics} are processed and counter
% \cnt{svg@param@currpage} is set to the value of a given \prm{page}. The value
% of parameter \prm{width} given in the inkscape export file---composed by
% |\svg@out@path\svg@out@name.\svg@ink@format_tex|---is ignored.
%    \begin{macrocode}
\DefineFamily{SVGpatch}
\DefineFamilyMember{SVGpatch}
\newcounter{svg@param@currpage}
\setcounter{svg@param@currpage}{\m@ne}
\FamilyCounterKey{SVGpatch}{page}{svg@param@currpage}
\DefineFamilyKey{SVGpatch}{width}{\FamilyKeyStateProcessed}
\newcommand*\svg@includegraphics@file{}
\newcommand*\svg@includegraphics@patched[2][]{%
  \FamilyOptions{SVGpatch}{#1}%
%    \end{macrocode}
% If option \opt{lastpage} was set to \val{false}, each page is included---even 
% if it doesn't exist, which may cause errors.
%    \begin{macrocode}
  \ifnum\value{svg@param@lastpage}<\z@\relax%
    \FamilySetCounter{SVGpatch}{page}{svg@param@currpage}{%
      \the\value{svg@param@lastpage}%
    }%
  \fi%
%    \end{macrocode}
% Only if counter \cnt{svg@param@lastpage} is smaller than 
% \cnt{svg@param@currpage}, pages are included, where \cnt{svg@param@lastpage}
% was either given as a number with parameter \prm{lastpage} or was
% automatically calculated with \cs{svg@get@lastpage}.
%    \begin{macrocode}
  \ifnum\value{svg@param@currpage}>\value{svg@param@lastpage}\relax\else%
%    \end{macrocode}
% A page is included with the original definition of \cs{includegraphics}. All 
% optional parameters are passed.
%    \begin{macrocode}
    \svg@includegraphics@saved[{#1}]{\svg@includegraphics@file}%
  \fi%
}
%    \end{macrocode}
% \end{macro}^^A \svg@includegraphics@file
% \end{macro}^^A \svg@includegraphics@patched
%
% \iffalse
%</body>
%</package&base>
%<*package&option>
% \fi
%
%
%
% \section{Extracting independent graphic files with \pkg{svg-extract}}
% \subsection{Options}
%
% For package \pkg{svg-extract} the user interface is extended. The following 
% options can either be set with \cs{svgsetup} or be used as local optional 
% parameters for \cs{includesvg} and \cs{includeinkscape}.
%
% \begin{macro}{\svg@dummy@key}
% \begin{macro}{\svg@dummy@key@wrn}
% \changes{v2.03}{2019/06/28}{new}^^A
% If package \pkg{svg-extract} wasn't loaded, the following options are defined 
% for package \pkg{svg} in order to raise a warning message. Primarily this is 
% done for compatibility reasons.
%    \begin{macrocode}
%<*base>
\DefineFamilyMember[.dummy]{SVG}
\newcommand*\svg@dummy@key[2][]{%
  \IfArgIsEmpty{#1}{%
    \DefineFamilyKey[.dummy]{SVG}{#2}{%
      \svg@dummy@key@wrn{#2}%
      \FamilyKeyStateProcessed%
    }%
  }{%
    \DefineFamilyKey[.dummy]{SVG}{#2}[{#1}]{%
      \svg@dummy@key@wrn{#2}%
      \FamilyKeyStateProcessed%
    }%
  }%
%    \end{macrocode}
% Before package \pkg{svg-extract} the given key |#2| of family member |.dummy| 
% is relaxed.
%    \begin{macrocode}
  \AfterPackage{svg-extract}{\RelaxFamilyKey[.dummy]{SVG}{#2}}%
}
\newcommand*\svg@dummy@key@wrn[1]{%
  \PackageWarning{svg}{%
    The option key `#1' can only\MessageBreak%
    be used with package `svg-extract', but\MessageBreak%
    you didn't load it%
  }%
}
%</base>
%    \end{macrocode}
% \end{macro}^^A \svg@dummy@key@wrn
% \end{macro}^^A \svg@dummy@key
%
%
%
% \subsubsection{Controlling the extract process}
%
% \begin{option}{extract}
% \changes{v2.00}{2017/02/24}{new}^^A
% \begin{macro}{\svgx@mode}
% \changes{v2.03}{2019/07/14}{new}^^A
% With option \opt{extract} it can be controlled, if the extraction of 
% independent graphic files should be done.
%    \begin{macrocode}
%<*base>
\svg@dummy@key[true]{extract}
%</base>
%<*extract>
\newcommand*\svgx@mode{1}
\DefineFamilyKey{SVG}{extract}[true]{%
  \svg@lowercase[{#1}]{\svg@tempa}%
  \FamilySetNumerical{SVG}{extract}{svg@tempa}{%
    {false}{0},{off}{0},{no}{0},%
    {true}{1},{on}{1},{yes}{1},{auto}{1},{onlynewer}{1},{newer}{1},%
    {forced}{2},{force}{2},{overwrite}{2},%
    {pdf}{3},{eps}{4},{ps}{5}%
  }{\svg@tempa}%
  \ifx\FamilyKeyState\FamilyKeyStateProcessed%
    \ifcase\svg@tempa\relax% false
      \def\svgx@mode{0}%
    \or% true
      \ifx\svg@IfFileNewer\@undefined%
        \def\svgx@mode{2}%
      \else%
        \def\svgx@mode{1}%
      \fi%
    \or% forced
      \def\svgx@mode{2}%
    \or% pdf
      \FamilyOptions{SVG}{extractformat=pdf}%
    \or% eps
      \FamilyOptions{SVG}{extractformat=eps}%
    \or% ps
      \FamilyOptions{SVG}{extractformat=ps}%
    \fi%
  \fi%
}
%</extract>
%    \end{macrocode}
% \end{macro}^^A \svgx@mode
% \end{option}^^A extract
% \begin{option}{on}
% \changes{v2.00}{2017/02/24}{new}^^A
% \begin{option}{forced}
% \changes{v2.03}{2019/07/25}{new}^^A
% \begin{option}{auto}
% \changes{v2.03}{2019/07/25}{new}^^A
% \begin{option}{off}
% \changes{v2.00}{2017/02/24}{new}^^A
% Package options which can be used to switch functionality on or off during 
% the loading of package \pkg{svg-extract}.
%    \begin{macrocode}
%<*extract>
\DeclareOption{on}{\AtEndOfPackage{\FamilyOptions{SVG}{extract=forced}}}
\DeclareOption{forced}{\AtEndOfPackage{\FamilyOptions{SVG}{extract=forced}}}
\DeclareOption{auto}{\AtEndOfPackage{\FamilyOptions{SVG}{extract=true}}}
\DeclareOption{off}{\AtEndOfPackage{\FamilyOptions{SVG}{extract=false}}}
%</extract>
%    \end{macrocode}
% \end{option}^^A off
% \end{option}^^A auto
% \end{option}^^A forced
% \end{option}^^A on
% \begin{option}{extractformat}
% \changes{v2.00}{2017/02/24}{new}^^A
% \changes{v2.03}{2019/07/13}{uses \cs{svgx@normalize@list}}^^A
% \changes{v2.03}{2019/08/22}{validates given formats}^^A
% \begin{macro}{\svgx@format}
% \begin{option}{pdf}
% \changes{v2.00}{2017/03/02}{deprecated}^^A
% \begin{option}{eps}
% \changes{v2.00}{2017/03/02}{deprecated}^^A
% Option \opt{extractformat} controls the output format (\val{pdf/eps/ps}). It
% is set to \val{pdf} or, if dvi output could be detected, to \val{eps} during
% initialization.
%    \begin{macrocode}
%<*base>
\svg@dummy@key{extractformat}
\svg@dummy@key[true]{pdf}
\svg@dummy@key[true]{eps}
%</base>
%<*extract>
\newcommand*\svgx@format{pdf}
\ifxetex\else\ifpdf\else
  \renewcommand*\svgx@format{eps}
\fi\fi
\DefineFamilyKey{SVG}{extractformat}{%
  \svg@lowercase[{#1}]{\svgx@format}%
  \svgx@normalize@list{\svgx@format}%
%    \end{macrocode}
% The given formats are validated, only \val{pdf,ps,eps} are supported. Any 
% other format is stored in \cs{svg@tempb}, which is used to create a specific 
% warning, and removed from list \cs{svgx@format}.
%    \begin{macrocode}
  \let\svg@tempb\@empty%
  \@for\svg@tempa:=\svgx@format\do{%
    \svgx@ifin@list{pdf,ps,eps}{\svg@tempa}{}{%
      \edef\svg@tempb{\svg@tempb,\svg@tempa}%
      \edef\svg@tempa{%
        \etex@unexpanded{\svgx@removekey@list{\svgx@format}}{\svg@tempa}%
      }%
      \svg@tempa%
    }%
  }%
  \ifx\svg@tempb\@empty\else%
    \svgx@normalize@list{\svg@tempb}%
    \PackageWarning{svg-extract}{%
      The format(s) `\svg@tempb'are invalid. \MessageBreak%
      Only formats `pdf,ps,eps' are supported, so\MessageBreak%
      `exportformat=\svgx@format' was set%
    }%
  \fi%
  \FamilyKeyStateProcessed%
}
\DefineFamilyKey{SVG}{pdf}[true]{%
  \FamilySetBool{SVG}{pdf}{@svg@tempswa}{#1}%
  \ifx\FamilyKeyState\FamilyKeyStateProcessed%
    \if@svg@tempswa%
      \svgx@ifin@list{\svgx@format}{pdf}{}{%
        \edef\svgx@format{\svgx@format,pdf}%
      }%
      \svg@deprecated@key[svg-extract]{pdf}{extractformat={\svgx@format}}%
    \else%
      \FamilyKeyStateUnknownValue%
    \fi%
  \fi%
}
\DefineFamilyKey{SVG}{eps}[true]{%
  \FamilySetBool{SVG}{eps}{@svg@tempswa}{#1}%
  \ifx\FamilyKeyState\FamilyKeyStateProcessed%
    \if@svg@tempswa%
      \svgx@ifin@list{\svgx@format}{eps}{}{%
        \edef\svgx@format{\svgx@format,eps}%
      }%
      \svg@deprecated@key[svg-extract]{eps}{extractformat={\svgx@format}}%
    \else%
      \FamilyKeyStateUnknownValue%
    \fi%
  \fi%
}
%</extract>
%    \end{macrocode}
% \end{option}^^A eps
% \end{option}^^A pdf
% \end{macro}^^A \svgx@format
% \end{option}^^A extractformat
% \begin{option}{extractpreamble}
% \changes{v2.00}{2017/02/24}{new}^^A
% \begin{option}{preamble}
% \changes{v2.00}{2017/02/24}{deprecated}^^A
% \begin{macro}{\svgx@preamble}
% \begin{option}{extractpreambleend}
% \changes{v2.00}{2017/02/24}{new}^^A
% \begin{option}{end}
% \changes{v2.00}{2017/02/24}{deprecated}^^A
% \begin{macro}{\svgx@endpreamble}
% For the extraction process, a preamble is necessary for a separate auxiliary 
% \LaTeX~file. By default, the preamble of the main document is used, which end
% is detected at \cs{begin}|{document}|.
%    \begin{macrocode}
%<*base>
\svg@dummy@key{extractpreamble}
\svg@dummy@key{preamble}
\svg@dummy@key{extractpreambleend}
\svg@dummy@key{end}
%</base>
%<*extract>
\newcommand*\svgx@preamble{\jobname}%
\DefineFamilyKey{SVG}{extractpreamble}{%
  \renewcommand*\svgx@preamble{#1}%
  \FamilyKeyStateProcessed%
}
\DefineFamilyKey{SVG}{preamble}{%
  \svg@deprecated@key[svg-extract]{preamble=#1}{extractpreamble=#1}%
}
\newcommand*\svgx@endpreamble{}
\expandafter\def\expandafter\svgx@endpreamble\expandafter{%
  \csname begin\endcsname{document}%
}
\DefineFamilyKey{SVG}{extractpreambleend}{%
  \renewcommand*\svgx@endpreamble{#1}%
  \FamilyKeyStateProcessed%
}
\DefineFamilyKey{SVG}{end}{%
  \svg@deprecated@key[svg-extract]{end=#1}{extractpreambleend=#1}%
}
%</extract>
%    \end{macrocode}
% \end{macro}^^A \svgx@endpreamble
% \end{option}^^A end
% \end{option}^^A extractpreambleend
% \end{macro}^^A \svgx@preamble
% \end{option}^^A preamble
% \end{option}^^A extractpreamble
% \begin{option}{extractruns}
% \begin{option}{extractiterations}
% \changes{v2.00}{2017/02/24}{new}^^A
% \begin{counter}{svgx@iterations}
% With this option, the number of \LaTeX~iterations for the separate auxiliary
% file can be set.
%    \begin{macrocode}
%<*base>
\svg@dummy@key{extractruns}
\svg@dummy@key{extractiterations}
%</base>
%<*extract>
\newcounter{svgx@iterations}
\DefineFamilyKey{SVG}{extractruns}{%
  \FamilySetCounter{SVG}{extractruns}{svgx@iterations}{#1}%
  \ifx\FamilyKeyState\FamilyKeyStateProcessed%
    \ifnum\value{svgx@iterations}<\@ne\relax%
      \PackageWarning{svg-extract}{%
        The count for iterations has to be at least one%
      }%
      \FamilySetCounter{SVG}{extractruns}{svgx@iterations}{\@ne}%
    \fi%
  \fi%
}
\DefineFamilyKey{SVG}{extractiterations}{\FamilyOptions{SVG}{extractruns=#1}}
%</extract>
%    \end{macrocode}
% \end{counter}^^A svgx@iterations
% \end{option}^^A extractiterations
% \end{option}^^A extractruns
% \begin{option}{latexexe}
% \changes{v2.00}{2017/02/24}{new}^^A
% \begin{option}{pdflatex}
% \changes{v2.00}{2017/02/24}{deprecated}^^A
% \begin{macro}{\svgx@latex@exe}
% \changes{v2.03}{2019/07/10}{simpler definition for \val{dvi} output}^^A
% \begin{option}{latexext}
% \changes{v2.00}{2017/02/24}{new}^^A
% \begin{macro}{\svgx@latex@ext}
% \begin{option}{latexopt}
% \changes{v2.00}{2017/02/24}{new}^^A
% \begin{macro}{\svgx@latex@opt}
% The command and facultative options for the \LaTeX~call of the separate 
% auxiliary file. The default is set according to the currently used compiler.
%    \begin{macrocode}
%<*base>
\svg@dummy@key{latexexe}
\svg@dummy@key{pdflatex}
\svg@dummy@key{latexext}
\svg@dummy@key{latexopt}
%</base>
%<*extract>
\ifxetex
  \newcommand*\svgx@latex@exe{xelatex}
\else
  \ifluatex
    \newcommand*\svgx@latex@exe{lualatex}
  \else
    \newcommand*\svgx@latex@exe{pdflatex}
  \fi
  \ifpdf\else
    \edef\svgx@latex@exe{\svgx@latex@exe\space--output-format=dvi}
  \fi
\fi
\DefineFamilyKey{SVG}{latexexe}{%
  \renewcommand*\svgx@latex@exe{#1}%
  \FamilyKeyStateProcessed%
}
\DefineFamilyKey{SVG}{pdflatex}{%
  \svg@deprecated@key[svg-extract]{pdflatex=#1}{latexexe=#1}%
}
\newcommand*\svgx@latex@ext{tex}
\DefineFamilyKey{SVG}{latexext}{%
  \renewcommand*\svgx@latex@ext{#1}%
  \FamilyKeyStateProcessed%
}
\newcommand*\svgx@latex@opt{}
\DefineFamilyKey{SVG}{latexopt}{%
  \renewcommand*\svgx@latex@opt{#1}%
  \FamilyKeyStateProcessed%
}
%</extract>
%    \end{macrocode}
% \end{macro}^^A \svgx@latex@opt
% \end{option}^^A latexopt
% \end{macro}^^A \svgx@latex@ext
% \end{option}^^A latexext
% \end{macro}^^A \svgx@latex@exe
% \end{option}^^A latexexe
% \end{option}^^A pdflatexexe
% \begin{option}{dvipsopt}
% \changes{v2.00}{2017/02/24}{new}^^A
% \begin{macro}{\svgx@dvips@exe}
% \begin{macro}{\svgx@dvips@opt}
% \begin{option}{pstoepsopt}
% \changes{v2.00}{2017/02/24}{new}^^A
% \begin{macro}{\svgx@pstoeps@exe}
% \begin{macro}{\svgx@pstoeps@opt}
% \begin{option}{pstopdfopt}
% \changes{v2.00}{2017/02/24}{new}^^A
% \begin{macro}{\svgx@pstopdf@exe}
% \begin{macro}{\svgx@pstopdf@opt}
% \begin{option}{pdftoepsopt}
% \changes{v2.00}{2017/02/24}{new}^^A
% \begin{macro}{\svgx@pdftoeps@exe}
% \begin{macro}{\svgx@pdftoeps@opt}
% \begin{option}{pdftopsopt}
% \changes{v2.00}{2017/02/24}{new}^^A
% \begin{macro}{\svgx@pdftops@exe}
% \begin{macro}{\svgx@pdftops@opt}
% \begin{option}{pdftops}
% \changes{v2.00}{2017/02/24}{deprecated}^^A
% Options and macros for calling convert commands, which are supplied by most
% \LaTeXe~distributions. These are used to generate all files, which are 
% supported by option \opt{extractformat}, as they don't need an additional
% application.
%    \begin{macrocode}
%<*base>
\svg@dummy@key{dvipsopt}
\svg@dummy@key{pstoepsopt}
\svg@dummy@key{pstopdfopt}
\svg@dummy@key{pdftoepsopt}
\svg@dummy@key{pdftopsopt}
\svg@dummy@key{pdftops}
%</base>
%<*extract>
\newcommand*\svgx@dvips@exe{dvips}
\newcommand*\svgx@dvips@opt{}
\DefineFamilyKey{SVG}{dvipsopt}{%
  \renewcommand*\svgx@dvips@opt{#1}%
  \FamilyKeyStateProcessed%
}
\newcommand*\svgx@pstoeps@exe{ps2eps}
\newcommand*\svgx@pstoeps@opt{-B -C}
\DefineFamilyKey{SVG}{pstoepsopt}{%
  \renewcommand*\svgx@pstoeps@opt{#1}%
  \FamilyKeyStateProcessed%
}
\newcommand*\svgx@pstopdf@exe{ps2pdf}
\newcommand*\svgx@pstopdf@opt{}
\DefineFamilyKey{SVG}{pstopdfopt}{%
  \renewcommand*\svgx@pstopdf@opt{#1}%
  \FamilyKeyStateProcessed%
}
\newcommand*\svgx@pdftoeps@exe{pdftops -eps}
\newcommand*\svgx@pdftoeps@opt{}
\DefineFamilyKey{SVG}{pdftoepsopt}{%
  \renewcommand*\svgx@pdftoeps@opt{#1}%
  \FamilyKeyStateProcessed%
}
\newcommand*\svgx@pdftops@exe{pdftops}
\newcommand*\svgx@pdftops@opt{}
\DefineFamilyKey{SVG}{pdftopsopt}{%
  \renewcommand*\svgx@pdftops@opt{#1}%
  \FamilyKeyStateProcessed%
}
\DefineFamilyKey{SVG}{pdftops}{%
  \PackageWarning{svg-extract}{%
    The option key `pdftops' is deprecated. \MessageBreak%
    You should use either `pdftoepsopt' or\MessageBreak%
    `pdftopsopt' instead. See the manual for\MessageBreak%
    more. Nothing was done%
  }%
  \FamilyKeyStateProcessed%
}
%</extract>
%    \end{macrocode}
% \end{option}^^A pdftops
% \end{macro}^^A \svgx@pdftops@opt
% \end{macro}^^A \svgx@pdftops@exe
% \end{option}^^A pdftopsopt
% \end{macro}^^A \svgx@pdftoeps@opt
% \end{macro}^^A \svgx@pdftoeps@exe
% \end{option}^^A pdftoepsopt
% \end{macro}^^A \svgx@pstopdf@opt
% \end{macro}^^A \svgx@pstopdf@exe
% \end{option}^^A pstopdfopt
% \end{macro}^^A \svgx@pstoeps@opt
% \end{macro}^^A \svgx@pstoeps@exe
% \end{option}^^A pstoepsopt
% \end{macro}^^A \svgx@dvips@opt
% \end{macro}^^A \svgx@dvips@exe
% \end{option}^^A dvipsopt
%
%
%
% \subsubsection{Invoking external application for graphic conversion}
%
% Besides the use of a conversion tool supplied by \LaTeXe{}, the applications 
% \app{ImageMagick} and \app{Ghostscript} can be used for converting graphics.
%
% \begin{option}{convert}
% \changes{v2.00}{2017/02/24}{changed/extended}^^A
% \changes{v2.03}{2019/07/15}{improved handling of legacy option}^^A
% \begin{macro}{\svgx@cnv@mode}
% \changes{v2.03}{2019/07/14}{new}^^A
% \begin{macro}{\svgx@cnv@cmd}
% The option \opt{convert} can be used to define, which of both applications 
% should be use. \app{ImageMagick} is set by default.
%    \begin{macrocode}
%<*base>
\svg@dummy@key[true]{convert}
%</base>
%<*extract>
\newcommand*\svgx@cnv@mode{0}
\newcommand*\svgx@cnv@cmd{\svgx@magick@cmd}
\DefineFamilyKey{SVG}{convert}[true]{%
  \FamilySetNumerical{SVG}{convert}{svg@tempa}{%
    {false}{0},{off}{0},{no}{0},%
    {true}{1},{on}{1},{yes}{1},{auto}{1},{onlynewer}{1},{newer}{1},%
    {forced}{2},{force}{2},{overwrite}{2},%
    {magick}{3},{imagemagick}{3},{convert}{3},%
    {gs}{4},{ghostscript}{4},%
    {gs64}{5},{ghostscript64}{5},%
    {gs32}{6},{ghostscript32}{6}%
  }{#1}%
  \ifx\FamilyKeyState\FamilyKeyStateProcessed%
    \ifcase\svg@tempa\relax% false
      \def\svgx@cnv@mode{0}%
    \or% true
      \ifx\svg@IfFileNewer\@undefined%
        \def\svgx@cnv@mode{2}%
      \else%
        \def\svgx@cnv@mode{1}%
      \fi%
    \or% forced
      \def\svgx@cnv@mode{2}%
    \or% magick
      \renewcommand*\svgx@cnv@cmd{\svgx@magick@cmd}%
      \ifnum\svgx@cnv@mode=\z@\relax%
        \def\svgx@cnv@mode{1}%
      \fi%
    \or% gs
      \renewcommand*\svgx@cnv@cmd{\svgx@gs@cmd}%
      \ifnum\svgx@cnv@mode=\z@\relax%
        \def\svgx@cnv@mode{1}%
      \fi%
    \or% gs64
      \renewcommand*\svgx@cnv@cmd{\svgx@gs@cmd}%
      \svgx@onlywindows{%
        \renewcommand*\svgx@gs@exe{gswin64c}%
      }%
      \ifnum\svgx@cnv@mode=\z@\relax%
        \def\svgx@cnv@mode{1}%
      \fi%
    \or% gs32
      \renewcommand*\svgx@cnv@cmd{\svgx@gs@cmd}%
      \svgx@onlywindows{%
        \renewcommand*\svgx@gs@exe{gswin32c}%
      }%
      \ifnum\svgx@cnv@mode=\z@\relax%
        \def\svgx@cnv@mode{1}%
      \fi%
    \fi%
%    \end{macrocode}
% In version~v1.0 the option \opt{convert} was used to set both the executable 
% and options for the conversion application, meant for the usage of 
% \app{ImageMagick}. This is taken into account here.
%    \begin{macrocode}
  \else% legacy option
%    \end{macrocode}
% Same doing like with legacy part of option \opt{inkscape}.
%    \begin{macrocode}
    \svg@quotes@remove[{#1}]{\svg@tempb}%
    \def\svg@tempa##1-##2\@nil{%
      \IfArgIsEmpty{##2}{\let\svg@tempb\@empty}{%
        \def\svg@tempa####1-\@nil{\def\svg@tempb{-####1}}%
        \svg@tempa##2\@nil%
      }%
      \edef\svg@tempa{\trim@spaces{##1}}%
    }%
    \expandafter\svg@tempa\svg@tempb-\@nil%
    \ifx\svg@tempa\@empty\else%
      \if@svg@quotes@found%
        \edef\svg@tempa{"\svg@tempa"}%
      \fi%
    \fi%
    \@svg@tempswafalse%
    \ifx\svg@tempa\@empty\else\@svg@tempswatrue\fi%
    \ifx\svg@tempb\@empty\else\@svg@tempswatrue\fi%
    \if@svg@tempswa%
      \PackageWarning{svg-extract}{%
        Setting%
        \ifx\svg@tempa\@empty\else%
          \space the executable \ifx\svg@tempb\@empty\else and \fi%
        \fi%
        \ifx\svg@tempb\@empty\else%
          \space associated options%
        \fi%
        \MessageBreak% 
        for ImageMagick should be done with%
        \ifx\svg@tempa\@empty\else%
          \MessageBreak%
          `magickexe=\svg@tempa'%
        \fi%
        \ifx\svg@tempb\@empty\else%
          \MessageBreak%
          \ifx\svg@tempa\@empty\else and \fi\space%
          `magicksetting' and/or `magickoperator'%
        \fi%
        \MessageBreak% 
        Nevertheless, this was done by now anyway%
        \ifx\svg@tempb\@empty\else%
          , whereby \MessageBreak `magicksetting=\svg@tempb' was used%
        \fi%
      }%
      \FamilyOptions{SVG}{convert=magick}%
      \edef\svg@tempa{%
        \ifx\svg@tempa\@empty\else%
          \noexpand\FamilyOptions{SVG}{magickexe=\svg@tempa}%
        \fi%
        \ifx\svg@tempb\@empty\else%
          \noexpand\FamilyOptions{SVG}{magicksetting=\svg@tempb}%
        \fi%
      }%
      \svg@tempa%
    \fi%
  \fi%
}
%</extract>
%    \end{macrocode}
% \end{macro}^^A \svgx@cnv@cmd
% \end{macro}^^A \svgx@cnv@mode
% \end{option}^^A convert
% \begin{option}{convertformat}
% \changes{v2.00}{2017/02/24}{new}^^A
% \changes{v2.03}{2019/07/13}{uses \cs{svgx@normalize@list}}^^A
% \begin{macro}{\svgx@cnv@format}
% \begin{option}{png}
% \changes{v2.00}{2017/03/02}{deprecated}^^A
% Option \opt{convertformat} controls the output format for converted files. It
% is set to \val{png} by default.
%    \begin{macrocode}
%<*base>
\svg@dummy@key{convertformat}
\svg@dummy@key[true]{png}
%</base>
%<*extract>
\newcommand*\svgx@cnv@format{png}
\DefineFamilyKey{SVG}{convertformat}{%
  \svg@lowercase[{#1}]{\svgx@cnv@format}%
  \svgx@normalize@list{\svgx@cnv@format}%
  \FamilyKeyStateProcessed%
}
\DefineFamilyKey{SVG}{png}[true]{%
  \FamilySetBool{SVG}{png}{@svg@tempswa}{#1}%
  \ifx\FamilyKeyState\FamilyKeyStateProcessed%
    \if@svg@tempswa%
      \svgx@ifin@list{\svgx@cnv@format}{png}{}{%
        \edef\svgx@cnv@format{\svgx@cnv@format,png}%
      }%
      \svg@deprecated@key[svg-extract]{png}{convertformat={\svgx@cnv@format}}%
    \else%
      \FamilyKeyStateUnknownValue%
    \fi%
  \fi%
}
%</extract>
%    \end{macrocode}
% \end{option}^^A png
% \end{macro}^^A \svgx@cnv@format
% \end{option}^^A convertformat
% \begin{option}{convertdpi}
% \changes{v2.00}{2017/02/24}{new}^^A
% \begin{option}{convertdensity}
% \begin{macro}{\svgx@cnv@dpi}
% The option \opt{convertdpi} is meant to define the used density during the 
% conversion process. It can be set either for all designated output formats or 
% targeted for a specific format. It's also possible to use something like 
% \val{500x300}. Given values are resolved by \cs{svgx@cnv@get@dpi}. It's used 
% like \opt{convertdpi}|=|\val{300} or \opt{convertdpi}|=|\val{\{png=600\}} If 
% the option is used for a specific or for all output formats is recocnized by 
% \cs{svgx@ifkeyandval}.
%    \begin{macrocode}
%<*base>
\svg@dummy@key{convertdpi}
\svg@dummy@key{convertdensity}
%</base>
%<*extract>
\newcommand*\svgx@cnv@dpi{}
\let\svgx@cnv@dpi\relax
\DefineFamilyKey{SVG}{convertdpi}{%
  \FamilyKeyStateUnknownValue%
  \svgx@ifkeyandval{#1}{%
    \svgx@cnv@get@dpi{##2}%
    \ifx\svg@tempa\relax\else%
      \expandafter\edef\csname svgx@cnv@dpi@##1\endcsname{\svg@tempa}%
      \FamilyKeyStateProcessed%
    \fi%
  }{%
    \svgx@cnv@get@dpi{##1}%
    \ifx\svg@tempa\relax\else%
      \edef\svgx@cnv@dpi{\svg@tempa}%
      \FamilyKeyStateProcessed%
    \fi%
  }%
}
\DefineFamilyKey{SVG}{convertdensity}{\FamilyOptions{SVG}{convertdpi=#1}}
%</extract>
%    \end{macrocode}
% \end{macro}^^A \svgx@cnv@dpi
% \end{option}^^A convertdensity
% \end{option}^^A convertdpi
% \begin{option}{magickexe}
% \changes{v2.00}{2017/02/24}{new}^^A
% \begin{macro}{\svgx@magick@exe}
% \begin{option}{magicksetting}
% \changes{v2.00}{2017/02/24}{new}^^A
% \begin{macro}{\svgx@magick@set}
% \begin{option}{magickoperator}
% \changes{v2.00}{2017/02/24}{new}^^A
% \begin{macro}{\svgx@magick@opr}
% Setting the command including maybe the path to \app{ImageMagick}. The keys
% \opt{magicksetting} and \opt{magickoperator} should be used to add optional
% arguments before (\emph{Settings}) or after (\emph{Operators}) the input
% file. They can either be set for all or a specific output format as like
% option \opt{convertdpi}. For this \cs{svgx@setformatkey} is used.
%    \begin{macrocode}
%<*base>
\svg@dummy@key{magickexe}
\svg@dummy@key{magicksetting}
\svg@dummy@key{magickoperator}
%</base>
%<*extract>
\newcommand*\svgx@magick@exe{}
\DefineFamilyKey{SVG}{magickexe}{%
  \renewcommand*\svgx@magick@exe{#1}%
  \FamilyKeyStateProcessed%
}
\newcommand*\svgx@magick@set{}
\DefineFamilyKey{SVG}{magicksetting}{%
  \FamilyKeyStateUnknownValue%
  \svgx@setformatkey{#1}{svgx@magick@set}%
}
\newcommand*\svgx@magick@opr{}
\DefineFamilyKey{SVG}{magickoperator}{%
  \FamilyKeyStateUnknownValue%
  \svgx@setformatkey{#1}{svgx@magick@opr}%
}
%</extract>
%    \end{macrocode}
% \end{macro}^^A \svgx@magick@opr
% \end{option}^^A magickoperator
% \end{macro}^^A \svgx@magick@set
% \end{option}^^A magicksetting
% \end{macro}^^A \svgx@magick@exe
% \end{option}^^A magickexe
% \begin{option}{gsexe}
% \changes{v2.00}{2017/02/24}{new}^^A
% \begin{macro}{\svgx@gs@exe}
% \begin{option}{gsopt}
% \changes{v2.00}{2017/02/24}{new}^^A
% \begin{macro}{\svgx@gs@opt}
% \begin{option}{gsdevice}
% \changes{v2.00}{2017/02/24}{new}^^A
% \begin{macro}{\svgx@gs@device}
% Options to set the command including maybe the path to \app{Ghostscript}. As 
% \app{Ghostscript} needs a specific device defined for different output
% formats, the option \opt{gsdevice} can be used. It can either be set for all 
% or a specific output format just like \opt{gsopt} in the same manner like 
% option \opt{convertdpi}. 
%    \begin{macrocode}
%<*base>
\svg@dummy@key{gsexe}
\svg@dummy@key{gsopt}
\svg@dummy@key{gsdevice}
%</base>
%<*extract>
\newcommand*\svgx@gs@exe{}
\DefineFamilyKey{SVG}{gsexe}{%
  \renewcommand*\svgx@gs@exe{#1}%
  \FamilyKeyStateProcessed%
}
\newcommand*\svgx@gs@opt{}
\DefineFamilyKey{SVG}{gsopt}{%
  \FamilyKeyStateUnknownValue%
  \svgx@setformatkey{#1}{svgx@gs@opt}%
}
\newcommand*\svgx@gs@device{}
\DefineFamilyKey{SVG}{gsdevice}{%
  \FamilyKeyStateUnknownValue%
  \svgx@setformatkey{#1}{svgx@gs@device}%
}
%</extract>
%    \end{macrocode}
% \end{macro}^^A \svgx@gs@device
% \end{option}^^A gsdevice
% \end{macro}^^A \svgx@gs@opt
% \end{option}^^A gsopt
% \end{macro}^^A \svgx@gs@exe
% \end{option}^^A gsexe
%
%
%
% \subsubsection{Setting output folder}
%
% \begin{option}{extractpath}
% \changes{v2.00}{2017/02/24}{new}^^A
% \changes{v2.02}{2018/09/07}{usage of \cs{svg@sanitize@dq}}^^A
% \changes{v2.03}{2019/07/10}{%
%   \val{inkscapepath} and \val{inkscapesubpath} added%
% }^^A
% \begin{option}{path}
% \changes{v2.00}{2017/02/24}{deprecated}^^A
% \begin{macro}{\svgx@out@path}
% The option \opt{extractpath} controls, in which folder the results both of 
% the extraction as well as the conversion of \app{ImageMagick} or 
% \app{Ghostscript} will be located.
%    \begin{macrocode}
%<*base>
\svg@dummy@key{extractpath}
\svg@dummy@key{path}
%</base>
%<*extract>
\newcommand*\svgx@out@path{}
\DefineFamilyKey{SVG}{extractpath}{%
  \svg@sanitize@dq\svg@tempb{#1}%
  \FamilySetNumerical{SVG}{extractpath}{svg@tempa}{%
    {svgpath}{0},{svgdir}{0},%
    {svgsubpath}{1},{svgsubdir}{1},%
    {basepath}{2},{basedir}{2},{jobpath}{2},{jobdir}{2},%
    {basesubpath}{3},{basesubdir}{3},{jobsubpath}{3},{jobsubdir}{3},%
    {inkscapepath}{4},{inkscapedir}{4},%
    {inkscapesubpath}{5},{inkscapesubdir}{5}%
  }{\svg@tempb}%
  \ifx\FamilyKeyState\FamilyKeyStateProcessed%
    \ifcase\svg@tempa\relax% svgpath
      \renewcommand*\svgx@out@path{\svg@file@path}%
    \or% svgsubpath
      \renewcommand*\svgx@out@path{\svg@file@path svg-extract/}%
    \or% basepath
      \renewcommand*\svgx@out@path{\@currdir}%
    \or% basesubpath
      \renewcommand*\svgx@out@path{\@currdir svg-extract/}%
    \or% inkscapepath
      \renewcommand*\svgx@out@path{\svg@out@path}%
    \or% inkscapesubpath
      \renewcommand*\svgx@out@path{\svg@out@path svg-extract/}%
    \fi%
  \else%
    \edef\svgx@out@path{\svg@tempb}%
    \svg@normalize@path{\svgx@out@path}%
    \FamilyKeyStateProcessed%
  \fi%
}
\DefineFamilyKey{SVG}{path}{%
  \svg@deprecated@key[svg-extract]{path=#1}{extractpath=#1}%
}
%</extract>
%    \end{macrocode}
% \end{macro}^^A \svgx@out@path
% \end{option}^^A path
% \end{option}^^A extractpath
% \begin{option}{extractname}
% \changes{v2.00}{2017/02/24}{new}^^A
% \changes{v2.02}{2018/09/07}{usage of \cs{svg@quotes@remove}}^^A
% \changes{v2.03}{2019/07/25}{\opt{extractname} permits value \cs{relax}}^^A
% \begin{option}{name}
% \changes{v2.00}{2017/02/24}{deprecated}^^A
% \changes{v2.00}{2017/02/24}{support of \pkg{subfig} removed}^^A
% \begin{macro}{\svgx@out@name}
% \begin{macro}{\if@svgx@out@sec}
% \begin{counter}{svgx@out@count}
% \begin{macro}{\svgx@out@sec}
% With option \opt{extractname} the name of the extracted and maybe converted
% file itself can be changed.
% \ToDo{option to derive file name from \cs{svg@file@name}}[v2.03]
%    \begin{macrocode}
%<*base>
\svg@dummy@key{extractname}
\svg@dummy@key{name}
%</base>
%<*extract>
\newcommand*\svgx@out@name{\the\value{svgx@out@count}-\svg@out@name}
\newif\if@svgx@out@sec
\newcounter{svgx@out@count}
\newcommand*\svgx@out@sec{unknown}
\DefineFamilyKey{SVG}{extractname}{%
  \svg@ifvalueisrelax{#1}{%
    \renewcommand*\svgx@out@name{\the\value{svgx@out@count}-\svg@out@name}%
    \FamilyKeyStateProcessed%
  }{%
    \svg@sanitize@dq\svg@tempb{#1}%
    \FamilySetNumerical{SVG}{extractname}{svg@tempa}{%
      {filename}{0},{name}{0},%
      {filenamenumbered}{1},{namenumbered}{1},%
      {numberedfilename}{1},{numberedname}{1},%
      {numbered}{2},{section}{2},{numberedsection}{2},{sectionnumbered}{2}%
    }{\svg@tempb}%
    \@svgx@out@secfalse%
    \ifx\FamilyKeyState\FamilyKeyStateProcessed%
      \ifcase\svg@tempa\relax% filename
        \renewcommand*\svgx@out@name{\svg@out@name-extract}%
      \or% filenamenumbered
        \renewcommand*\svgx@out@name{\the\value{svgx@out@count}-\svg@out@name}%
      \or% numbered
        \renewcommand*\svgx@out@name{\the\value{svgx@out@count}-\svgx@out@sec}%
        \@svgx@out@sectrue%
      \fi%
    \else%
      \renewcommand*\svgx@out@name{#1}%
      \FamilyKeyStateProcessed%
    \fi%
  }%
}
\DefineFamilyKey{SVG}{name}{%
  \svg@deprecated@key[svg-extract]{name=#1}{extractname=#1}%
}
%</extract>
%    \end{macrocode}
% \end{macro}^^A \svgx@out@sec
% \end{counter}^^A svgx@out@count
% \end{macro}^^A \if@svgx@out@sec
% \end{macro}^^A \svgx@out@name
% \end{option}^^A name
% \end{option}^^A extractname
%
%
%
% \subsubsection{Options for the extraction of graphics}
%
% \begin{option}{extractwidth}
% \changes{v2.00}{2017/02/20}{new}^^A
% \begin{macro}{\svgx@param@width}
% \begin{option}{extractheight}
% \changes{v2.00}{2017/02/20}{new}^^A
% \begin{macro}{\svgx@param@width}
% \begin{option}{extractdistort}
% \changes{v2.02}{2018/09/07}{new}^^A
% \begin{option}{extractkeepaspectratio}
% \begin{macro}{\svgx@param@distort}
% \begin{option}{extractscale}
% \changes{v2.00}{2017/02/20}{new}^^A
% \begin{macro}{\svgx@param@scale}
% For graphic extraction, the given settings regarding the size for inclusion 
% can be overwritten with these options. Using \val{\cs{relax}} as value leads 
% to reseting an option as unset, regardless of what was previously given. The
% value \val{inherit} means, that the actual option for including is used for
% extraction as well. This is the default setting.
% \ToDo{implement key \opt{extractsize} like \opt{size=5cm:6cm}}[v2.03]
%    \begin{macrocode}
%<*base>
\svg@dummy@key{extractwidth}
\svg@dummy@key{extractheight}
\svg@dummy@key{extractdistort}
\svg@dummy@key{extractkeepaspectratio}
\svg@dummy@key{extractscale}
%</base>
%<*extract>
\newcommand*\svgx@param@width{\svg@param@width}
\DefineFamilyKey{SVG}{extractwidth}{%
  \svg@ifvalueisrelax{#1}{%
    \renewcommand*\svgx@param@width{\z@}%
    \FamilyKeyStateProcessed%
  }{%
    \svgx@ifvalueisinherit{#1}{%
      \renewcommand*\svgx@param@width{\svg@param@width}%
      \FamilyKeyStateProcessed%
    }{%
      \FamilySetUseLengthMacro{SVG}{extractwidth}{\svgx@param@width}{#1}%
      \ifx\FamilyKeyState\FamilyKeyStateProcessed%
        \ifdim\svgx@param@width<\z@\relax%
          \FamilyKeyStateUnknownValue%
        \fi%
      \fi%
    }%
  }%
}
\newcommand*\svgx@param@height{\svg@param@height}
\DefineFamilyKey{SVG}{extractheight}{%
  \svg@ifvalueisrelax{#1}{%
    \renewcommand*\svgx@param@height{\z@}%
    \FamilyKeyStateProcessed%
  }{%
    \svgx@ifvalueisinherit{#1}{%
      \renewcommand*\svgx@param@height{\svg@param@height}%
      \FamilyKeyStateProcessed%
    }{%
      \FamilySetUseLengthMacro{SVG}{extractheight}{\svgx@param@height}{#1}%
      \ifx\FamilyKeyState\FamilyKeyStateProcessed%
        \ifdim\svgx@param@height<\z@\relax%
          \FamilyKeyStateUnknownValue%
        \fi%
      \fi%
    }%
  }%
}
\newif\if@svgx@param@distort
\DefineFamilyKey{SVG}{extractdistort}[true]{%
  \svgx@ifvalueisinherit{#1}{%
    \renewcommand*\if@svgx@param@distort{\if@svg@param@distort}%
    \FamilyKeyStateProcessed%
  }{%
    \FamilySetBool{SVG}{extractdistort}{@svgx@param@distort}{#1}%
  }%
}
\DefineFamilyKey{SVG}{extractkeepaspectratio}[true]{%
  \FamilySetBool{SVG}{extractkeepaspectratio}{@svg@tempswa}{#1}%
  \ifx\FamilyKeyState\FamilyKeyStateProcessed%
    \if@svg@tempswa%
      \FamilyOptions{SVG}{extractdistort=false}%
    \else%
      \FamilyOptions{SVG}{extractdistort=true}%
    \fi%
  \else%
    \FamilyOptions{SVG}{extractdistort=#1}%
  \fi%
}
\newcommand*\svgx@param@scale{\svg@param@scale}
\DefineFamilyKey{SVG}{extractscale}{%
  \FamilyKeyStateUnknownValue%
  \svg@ifvalueisrelax{#1}{%
    \renewcommand*\svgx@param@scale{1}%
    \FamilyKeyStateProcessed%
  }{%
    \svgx@ifvalueisinherit{#1}{%
      \renewcommand*\svgx@param@scale{\svg@param@scale}%
      \FamilyKeyStateProcessed%
    }{%
      \ifisdimension{#1\p@}{%
        \ifdim\dimexpr#1\p@\relax>\z@\relax%
          \renewcommand*\svgx@param@scale{#1}%
          \FamilyKeyStateProcessed%
        \fi%
      }{}%
    }%
  }%
}
%</extract>
%    \end{macrocode}
% \end{macro}^^A \svgx@param@scale
% \end{option}^^A extractscale
% \end{macro}^^A \svgx@param@distort
% \end{option}^^A extractkeepaspectratio
% \end{option}^^A extractdistort
% \end{macro}^^A \svgx@param@height
% \end{option}^^A extractheight
% \end{macro}^^A \svgx@param@width
% \end{option}^^A extractwidth
% \begin{option}{extractpretex}
% \changes{v2.00}{2017/02/20}{new}^^A
% \begin{macro}{\svgx@param@pretex}
% \begin{option}{extractapptex}
% \changes{v2.00}{2017/02/20}{new}^^A
% \begin{macro}{\svgx@param@apptex}
% \begin{option}{extractpostex}
% The similar hooks for executing code right before or after the graphic 
% extraction.
%    \begin{macrocode}
%<*base>
\svg@dummy@key{extractpretex}
\svg@dummy@key{extractapptex}
\svg@dummy@key{extractpostex}
%</base>
%<*extract>
\newcommand*\svgx@param@pretex{\svg@param@pretex}
\DefineFamilyKey{SVG}{extractpretex}{%
  \svg@ifvalueisrelax{#1}{%
    \let\svgx@param@pretex\relax%
  }{%
    \svgx@ifvalueisinherit{#1}{%
      \renewcommand*\svgx@param@pretex{\svg@param@pretex}%
    }{%
      \renewcommand*\svgx@param@pretex{#1}%
    }%
  }%
  \FamilyKeyStateProcessed%
}
\newcommand*\svgx@param@apptex{\svg@param@apptex}
\DefineFamilyKey{SVG}{extractapptex}{%
  \svg@ifvalueisrelax{#1}{%
    \let\svgx@param@apptex\relax%
  }{%
    \svgx@ifvalueisinherit{#1}{%
      \renewcommand*\svgx@param@apptex{\svg@param@apptex}%
    }{%
      \renewcommand*\svgx@param@apptex{#1}%
    }%
  }%
  \FamilyKeyStateProcessed%
}
\DefineFamilyKey{SVG}{extractpostex}{%
  \svg@deprecated@key[svg-extract]{extractpostex=#1}{extractapptex=#1}%
}
%</extract>
%    \end{macrocode}
% \end{option}^^A extractpostex
% \end{macro}^^A \svgx@param@apptex
% \end{option}^^A extractapptex
% \end{macro}^^A \svgx@param@pretex
% \end{option}^^A extractpretex
%
%
%
% \subsubsection{Miscellaneous options}
%
% \begin{option}{clean}
% \changes{v2.00}{2017/02/24}{changes, file list possible}^^A
% \changes{v2.03}{2019/07/10}{value \val{nolatex} added}^^A
% \changes{v2.03}{2019/07/13}{uses \cs{svgx@normalize@list}}^^A
% \begin{option}{clear}
% \begin{macro}{\svgx@clean}
% With option \opt{clean} files generated during the extraction process can be 
% deleted. Setting \val{true} will remove all files, \val{false} won't clear 
% any file. Additionally, a specific file list of suffixes can be given.
%    \begin{macrocode}
%<*base>
\svg@dummy@key[true]{clean}
\svg@dummy@key[true]{clear}
%</base>
%<*extract>
\newcommand*\svgx@clean{}
\DefineFamilyKey{SVG}{clean}[true]{%
  \FamilySetNumerical{SVG}{clean}{svg@tempa}{%
    {false}{0},{off}{0},{no}{0},%
    {true}{1},{on}{1},{yes}{1},{full}{1},{complete}{1},{entire}{1},%
    {nolatex}{2},{notex}{2},{most}{2},{almost}{2},{nearly}{2}%
  }{#1}%
  \ifx\FamilyKeyState\FamilyKeyStateProcessed%
    \ifcase\svg@tempa\relax% false
      \renewcommand*\svgx@clean{}%
    \or% true
      \renewcommand*\svgx@clean{log,aux,dvi,out,ps,eps,pdf,\svgx@latex@ext}%
    \or% nolatex
      \renewcommand*\svgx@clean{log,aux,dvi,out,ps,eps,pdf}%
    \fi%
  \else%
    \svg@lowercase[{#1}]{\svgx@clean}%
    \svgx@normalize@list{\svgx@clean}%
    \FamilyKeyStateProcessed%
  \fi%
}
\DefineFamilyKey{SVG}{clear}[true]{\FamilyOptions{SVG}{clean=#1}}
%</extract>
%    \end{macrocode}
% \end{macro}^^A \svgx@clean
% \end{option}^^A clear
% \end{option}^^A clean
% \begin{option}{exclude}
% \changes{v2.03}{2019/07/12}{%
%   revised and simplified, moved functionality to \cs{svg@process}%
% }^^A
% \begin{macro}{\if@svgx@exclude}
% \changes{v2.03}{2019/07/12}{new}^^A
% If it is desired not to include but only extract graphics with package 
% \pkg{svg-extract}, option \opt{exclude} can be used.
%    \begin{macrocode}
%<*base>
\svg@dummy@key[true]{exclude}
%</base>
%<*extract>
\newif\if@svgx@exclude
\FamilyBoolKey{SVG}{exclude}{@svgx@exclude}
%</extract>
%    \end{macrocode}
% \end{macro}^^A \if@svgx@exclude
% \end{option}^^A exclude
%
% \iffalse
%</package&option>
%<*package&body>
% \fi
%
%
%
% \subsection{User commands}
%
% \begin{macro}{\includesvg}
% \begin{parameter}{extract}
% \begin{parameter}{extractpreamble}
% \begin{parameter}{extractformat}
% \begin{parameter}{extractwidth}
% \begin{parameter}{extractheight}
% \begin{parameter}{extractdistort}
% \changes{v2.02}{2018/09/07}{new}^^A
% \begin{parameter}{extractscale}
% \begin{parameter}{extractangle}
% \changes{v2.02}{2018/09/07}{new}^^A
% \begin{parameter}{extractpretex}
% \begin{parameter}{extractapptex}
% \begin{parameter}{extractruns}
% \begin{parameter}{latexexe}%
% \begin{parameter}{latexopt}%
% \begin{parameter}{latexext}%
% \begin{parameter}{dvipsopt}%
% \begin{parameter}{pstoepsopt}%
% \begin{parameter}{pstopdfopt}%
% \begin{parameter}{pdftoepsopt}%
% \begin{parameter}{pdftopsopt}%
% \begin{parameter}{convert}
% \begin{parameter}{convertformat}
% \begin{parameter}{convertdpi}
% \begin{parameter}{magicksetting}
% \begin{parameter}{magickoperator}
% \begin{parameter}{gsopt}
% \begin{parameter}{gsdevice}
% \begin{parameter}{clean}
% \begin{parameter}{exclude}
% \begin{macro}{\includeinkscape}
% \begin{parameter}{extract}
% \begin{parameter}{extractpreamble}
% \begin{parameter}{extractformat}
% \begin{parameter}{extractwidth}
% \begin{parameter}{extractheight}
% \begin{parameter}{extractdistort}
% \changes{v2.02}{2018/09/07}{new}^^A
% \begin{parameter}{extractscale}
% \begin{parameter}{extractangle}
% \changes{v2.02}{2018/09/07}{new}^^A
% \begin{parameter}{extractpretex}
% \begin{parameter}{extractapptex}
% \begin{parameter}{extractruns}
% \begin{parameter}{latexexe}%
% \begin{parameter}{latexopt}%
% \begin{parameter}{latexext}%
% \begin{parameter}{dvipsopt}%
% \begin{parameter}{pstoepsopt}%
% \begin{parameter}{pstopdfopt}%
% \begin{parameter}{pdftoepsopt}%
% \begin{parameter}{pdftopsopt}%
% \begin{parameter}{convert}
% \begin{parameter}{convertformat}
% \begin{parameter}{convertdpi}
% \begin{parameter}{magicksetting}
% \begin{parameter}{magickoperator}
% \begin{parameter}{gsopt}
% \begin{parameter}{gsdevice}
% \begin{parameter}{clean}
% \begin{parameter}{exclude}
% The parameters \prm{angle} and \prm{origin} are definied as pendants to the 
% keys provided by \cs{includegraphics}.
%    \begin{macrocode}
%<*extract>
\newcommand*\svgx@param@angle{0}
\svg@local@param@def{%
  \DefineFamilyKey[.param]{SVG}{extractangle}{%
    \FamilyKeyStateUnknownValue%
    \svg@ifvalueisrelax{#1}{%
      \renewcommand*\svgx@param@angle{0}%
      \FamilyKeyStateProcessed%
    }{%
      \svgx@ifvalueisinherit{#1}{%
        \renewcommand*\svgx@param@angle{\svg@param@angle}%
        \FamilyKeyStateProcessed%
      }{%
        \ifisdimension{#1\p@}{%
          \renewcommand*\svgx@param@angle{#1}%
          \FamilyKeyStateProcessed%
        }{}%
      }%
    }%
  }%
}
%</extract>
%    \end{macrocode}
% \end{parameter}^^A exclude
% \end{parameter}^^A clean
% \end{parameter}^^A gsdevice
% \end{parameter}^^A gsopt
% \end{parameter}^^A magickoperator
% \end{parameter}^^A magicksetting
% \end{parameter}^^A convertdpi
% \end{parameter}^^A convertformat
% \end{parameter}^^A convert
% \end{parameter}^^A pdftopsopt
% \end{parameter}^^A pdftoepsopt
% \end{parameter}^^A pstopdfopt
% \end{parameter}^^A pstoepsopt
% \end{parameter}^^A dvipsopt
% \end{parameter}^^A latexext
% \end{parameter}^^A latexopt
% \end{parameter}^^A latexexe
% \end{parameter}^^A extractruns
% \end{parameter}^^A extractapptex
% \end{parameter}^^A extractpretex
% \end{parameter}^^A extractangle
% \end{parameter}^^A extractscale
% \end{parameter}^^A extractdistort
% \end{parameter}^^A extractheight
% \end{parameter}^^A extractwidth
% \end{parameter}^^A extractformat
% \end{parameter}^^A extractpreamble
% \end{parameter}^^A extract
% \end{macro}^^A \includeinkscape
% \end{parameter}^^A exclude
% \end{parameter}^^A clean
% \end{parameter}^^A gsdevice
% \end{parameter}^^A gsopt
% \end{parameter}^^A magickoperator
% \end{parameter}^^A magicksetting
% \end{parameter}^^A convertdpi
% \end{parameter}^^A convertformat
% \end{parameter}^^A convert
% \end{parameter}^^A pdftopsopt
% \end{parameter}^^A pdftoepsopt
% \end{parameter}^^A pstopdfopt
% \end{parameter}^^A pstoepsopt
% \end{parameter}^^A dvipsopt
% \end{parameter}^^A latexext
% \end{parameter}^^A latexopt
% \end{parameter}^^A latexexe
% \end{parameter}^^A extractruns
% \end{parameter}^^A extractapptex
% \end{parameter}^^A extractpretex
% \end{parameter}^^A extractangle
% \end{parameter}^^A extractscale
% \end{parameter}^^A extractdistort
% \end{parameter}^^A extractheight
% \end{parameter}^^A extractwidth
% \end{parameter}^^A extractformat
% \end{parameter}^^A extractpreamble
% \end{parameter}^^A extract
% \end{macro}^^A \includesvg
% \begin{macro}{\svghidepreamblestart}
% \changes{v2.00}{2017/02/24}{new}^^A
% \begin{macro}{\svghidepreambleend}
% \changes{v2.00}{2017/02/24}{new}^^A
% Some dummys for package \pkg{svg}.
%    \begin{macrocode}
%<*base>
\newcommand*\svghidepreamblestart{%
  \PackageWarning{svg}{%
    The macro `\string\svghidepreamblestart' is only meant\MessageBreak%
    to be used together with package `svg-extract'. \MessageBreak%
    Nevertheless, nothing will happen%
  }%
}
\newcommand*\svghidepreambleend{%
  \PackageWarning{svg}{%
    The macro `\string\svghidepreambleend' is only meant\MessageBreak%
    to be used together with package `svg-extract'. \MessageBreak%
    Nevertheless, nothing will happen%
  }%
}
%</base>
%    \end{macrocode}
% These two macros can be used to hide some parts of the preamble during 
% reading the preamble of the main document.
%    \begin{macrocode}
%<*extract>
\let\svghidepreamblestart\relax
\let\svghidepreambleend\relax
%</extract>
%    \end{macrocode}
% \end{macro}^^A \svghidepreambleend
% \end{macro}^^A \svghidepreamblestart
% \begin{macro}{\includesvgx}
% \changes{v2.03}{2019/07/11}{new}^^A
% \begin{macro}{\includeinkscapex}
% \changes{v2.03}{2019/07/29}{new}^^A
% \begin{macro}{\svgx@include@cmd}
% \changes{v2.03}{2019/07/29}{new}^^A
% \begin{macro}{\svgx@include}
% \changes{v2.03}{2019/07/15}{new}^^A
% \begin{macro}{\svgx@@include}
% \changes{v2.03}{2019/07/15}{new}^^A
% \begin{macro}{\svgx@@@include}
% \changes{v2.03}{2019/07/15}{new}^^A
% \ToDo{source doc}[v2.03]
%    \begin{macrocode}
%<*base>
\newcommand*\includesvgx{%
  \PackageError{svg}{Package `svg-extract' not loaded}{%
    The macro `\string\includesvgx' can only\MessageBreak%
    be used together with package `svg-extract'.%
  }%
}
\newcommand*\includeinkscapex{%
  \PackageError{svg}{Package `svg-extract' not loaded}{%
    The macro `\string\includeinkscapex' can only\MessageBreak%
    be used together with package `svg-extract'.%
  }%
}
%</base>
%    \end{macrocode}
% \ToDo{how to handle/map parameters?}[v2.03]
% \ToDo{warning with hint to run with --shell-escape if file is missing}[v2.03]
% \ToDo{option to delete generated inkscape output}[v2.03]
% \ToDo{handle option \opt{size} to pass to includegraphics}[v2.03]
%    \begin{macrocode}
%<*extract>
\newcommand*\svgx@include@cmd{}
\renewcommand*\includesvgx{%
  \def\svgx@include@cmd{\includesvg}%
  \kernel@ifnextchar[%]
  {\svgx@include}{\svgx@@@include{}{}}%
}
\renewcommand*\includeinkscapex{%
  \def\svgx@include@cmd{\includeinkscape}%
  \kernel@ifnextchar[%]
  {\svgx@include}{\svgx@@@include{}{}}%
}
\newcommand*\svgx@include{}
\def\svgx@include[#1]{%
  \kernel@ifnextchar[%]
  {\svgx@@include{#1}}{%
  }%
}
\newcommand*\svgx@@include{}
\def\svgx@@include#1[#2]{%
  \svgx@@@include{#1}{#2}%
}
\newcommand*\svgx@@@include[3]{%
%^^A  \begingroup%
%^^A    \includesvg[{#2},exclude]{#3}%
%^^A    \IfFileExists{\svgx@out@base.pdf}{%
%^^A      \includegraphics[{#1}]{\svgx@out@base.pdf}%
%^^A    }{%
%^^A    }%
%^^A  \endgroup%
}
%</extract>
%    \end{macrocode}
% \end{macro}^^A \svgx@@@include
% \end{macro}^^A \svgx@@include
% \end{macro}^^A \svgx@include
% \end{macro}^^A \svgx@include@cmd
% \end{macro}^^A \includeinkscapex
% \end{macro}^^A \includesvgx
%
% \iffalse
%</package&body>
%<*package&extract&body>
% \fi
%
%
%
% \subsection{Auxiliary macros}
% \ToDo{move general macros up to "Helper macros"}[v2.03]
%
% \begin{macro}{\svgx@param@size@check}
% \changes{v2.03}{2019/08/21}{new}^^A
% The option \opt{extractscale} respectively the parameter \prm{extractscale}
% is only considered if the size was not specified.
%    \begin{macrocode}
\newcommand*\svgx@param@size@check{%
  \begingroup%
    \ifdim\dimexpr\svgx@param@scale\p@\relax=\p@\relax\else%
      \let\svg@tempb\@empty%
      \@for\svg@tempa:={scale,width,height}\do{%
%    \end{macrocode}
% In order to inform, which parameter of \prm{extract[scale/width/height]} is 
% inherited, the definition of \cs{svgx@param@[scale/width/height]} is checked, 
% if it is set to the corresponding \cs{svg@param@[scale/width/height]}.
%    \begin{macrocode}
        \begingroup%
          \edef\svg@tempb{%
            \expandafter\noexpand\csname svg@param@\svg@tempa\endcsname%
          }%
          \expandafter\ifx\csname svgx@param@\svg@tempa\endcsname\svg@tempb%
            \aftergroup\@svg@tempswatrue%
          \else%
            \aftergroup\@svg@tempswafalse%
          \fi%
        \endgroup%
        \edef\svg@tempb{%
          \etex@unexpanded\expandafter{\svg@tempb}%
          \if@svg@tempswa%
            {[inherited]\svg@tempa}%
          \else%
            {extract\svg@tempa}%
          \fi%
          {\csname svgx@param@\svg@tempa\endcsname}%
        }%
      }%
      \expandafter\svg@param@size@wrn\svg@tempb%
    \fi%
  \endgroup%
}
%    \end{macrocode}
% \end{macro}^^A \svgx@param@size@check
% \begin{macro}{\svgx@afterdocument@toks}
% \changes{v2.03}{2019/07/10}{new}^^A
% \begin{macro}{\svgx@afterdocument@xaddto}
% \changes{v2.03}{2019/07/10}{new}^^A
% \begin{macro}{\svgx@afterdocument@append}
% \changes{v2.03}{2019/08/22}{new}^^A
% As the extraction maybe needs to include the main~auxiliary~file with 
% \cs{externaldocument} provided by package \pkg{xr} it is necessary to do all 
% related stuff after the main~auxiliary~file was read. This is done with
% \cs{AfterReadingMainAux} provided by package \pkg{scrlfile}. For this 
% purpose, infos as well as all shell commands are stored within the token
% register \cs{svgx@afterdocument@toks}.
%    \begin{macrocode}
\newtoks\svgx@afterdocument@toks
\newcommand*\svgx@afterdocument@xaddto[1]{%
  \begingroup%
    \edef\svg@tempa{%
      \endgroup%
      \global\noexpand\addto@hook\noexpand\svgx@afterdocument@toks{#1}%
    }%
  \svg@tempa%
}
\newcommand*\svgx@afterdocument@append{%
  \expandafter\AfterReadingMainAux\expandafter{\the\svgx@afterdocument@toks}%
  \global\svgx@afterdocument@toks{}%
}
%    \end{macrocode}
% \end{macro}^^A \svgx@afterdocument@append
% \end{macro}^^A \svgx@afterdocument@xaddto
% \end{macro}^^A \svgx@afterdocument@toks
% \begin{macro}{\svg@process}
% \changes{v2.03}{2019/07/12}{new}^^A
% \changes{v2.03}{2019/07/26}{check for empty extract file name}^^A
% \begin{macro}{\if@svgx@run}
% \changes{v2.03}{2019/07/12}{new}^^A
% \begin{macro}{\svgx@out@base}
% \changes{v2.03}{2019/07/12}{new}^^A
% This macro is originally defined by package \pkg{svg} and used as a hook 
% to realize the additional features of package \pkg{svg-extract}. The task is 
% splitted in two parts: inputing the graphics exported by \app{Inkscape} 
% (\cs{svg@ink@include}) and doing all necessary for extraction and conversion 
% of independent graphic files afterwards.
%    \begin{macrocode}
\newif\if@svgx@run
\newcommand*\svgx@out@base{}
\renewcommand*\svg@process[3]{%
  \if@svgx@exclude\else%
    \svg@ink@include{#1}{#2}{#3}%
  \fi%
%    \end{macrocode}
% First some preparing for extraction/conversion. Macro\cs{svgx@get@out@sec} is 
% used to get the current level section numbering within the document and the
% counter for extracted graphics is stepped. Additionally, the base name for
% extracted and converted files is globally stored in \cs{svgx@out@base} in
% order to include extracted graphics as well. 
%    \begin{macrocode}
  \if@svgx@out@sec%
    \svgx@get@out@sec%
  \fi%
  \stepcounter{svgx@out@count}%
%    \end{macrocode}
% Now it is checked, if \cs{svgx@out@name} does not contain any path and 
% expands to an empty string or not.
%    \begin{macrocode}
  \svg@normalize@name{svg-extract}{extract}{svgx@out}%
  \ifx\svgx@out@name\@empty\else%
%    \end{macrocode}
% This is the part, where the conditional invocation of extraction/conversion 
% is examined. Therefore, \cs{svgx@ifexpired} is called in order to check, if 
% all desired files exist. The origin SVG~file is given via the mandatory 
% arguments.
%    \begin{macrocode}
    \edef\svg@out@file{#1#2.#3}%
    \ifcase\svgx@mode\relax% false
      \@svgx@runfalse%
    \or% true
      \svgx@ifexpired[\svgx@format]{\svgx@out@path\svgx@out@name}%
        {\svg@out@file}{\@svgx@runtrue}{\@svgx@runfalse}%
    \or% forced
      \@svgx@runtrue%
    \fi%
    \ifcase\svgx@mode\relax\else%
      \svgx@param@size@check%
      \if@svgx@run\else%
        \ifcase\svgx@cnv@mode\relax\or% true
          \svgx@ifexpired[\svgx@cnv@format]{\svgx@out@path\svgx@out@name}%
            {\svg@out@file}{\@svgx@runtrue}{\@svgx@runfalse}%
        \or% forced
          \@svgx@runtrue%
        \fi%
      \fi%
    \fi%
%    \end{macrocode}
% The process of extraction and conversion is only done, if at least one
% extraction or conversion format was defined.
%    \begin{macrocode}
    \ifstr{\svgx@format\svgx@cnv@format}{}{%
      \PackageWarning{svg-extract}{%
        Both keys `extractformat' and `convertformat' are\MessageBreak%
        empty, so nothing to do so far%
      }%
      \@svgx@runfalse%
    }{}%
%    \end{macrocode}
% Finally, \cs{svgx@run} is called to do the actual job of both extracting
% and converting independent graphic files.
%    \begin{macrocode}
    \if@svgx@run%
      \svgx@run{#1}{#2}{#3}%
    \else%
%    \end{macrocode}
% Warning, if option \opt{convert} is active and option \opt{extract} is not.
%    \begin{macrocode}
      \ifnum\svgx@cnv@mode>\z@\relax%
        \PackageWarning{svg-extract}{%
          Image `\svg@out@file'\MessageBreak%
          could not be converted as `extract=false' was set%
        }%
      \fi%
%    \end{macrocode}
% Warning, if option \opt{exclude} is active and no files were extracted.
%    \begin{macrocode}
      \if@svgx@exclude%
        \svgx@ifexpired{\svgx@out@path\svgx@out@name}%
          {\svg@out@file}{\@svg@tempswatrue}{\@svg@tempswafalse}%
        \if@svg@tempswa%
          \PackageWarning{svg-extract}{%
            Image `\svg@out@file'\MessageBreak%
            was neither extracted/converted nor included%
          }%
        \fi%
      \fi%
    \fi%
%    \end{macrocode}
% At the very end, all unwanted auxiliary files are deleted.
%    \begin{macrocode}
    \@for\svg@tempa:=\svgx@clean\do{%
      \svgx@ifin@list{\svgx@format,\svgx@cnv@format}{\svg@tempa}{}{%
        \svgx@afterdocument@xaddto{%
          \noexpand\svg@shell@rm{\svgx@out@name.\svg@tempa}%
        }%
      }%
    }%
    \svgx@afterdocument@append%
  \fi%
}
%    \end{macrocode}
% \end{macro}^^A \svgx@out@base
% \end{macro}^^A \if@svgx@run
% \end{macro}^^A \svg@process
% \begin{macro}{\svgx@get@out@sec}
% The macro \cs{svgx@get@out@sec} reads all sectioning counters in order to 
% get the numbering of the current sectioning level. The value is stored in 
% \cs{svgx@out@sec}.
%    \begin{macrocode}
\newcommand*\svgx@get@out@sec{%
  \begingroup%
    \let\svg@tempa\@empty%
    \@for\svg@tempb:={%
      part,chapter,section,subsection,subsubsection,paragraph,subparagraph%
    }\do{%
      \scr@ifundefinedorrelax{the\svg@tempb}{}{%
        \ifnum\value{\svg@tempb}>\z@\relax%
          \edef\svg@tempa{\svg@tempb}%
        \fi%
      }%
    }%
    \edef\svg@tempb{%
      \endgroup%
      \ifx\svg@tempa\@empty\else%
        \def\noexpand\svgx@out@sec{\csname the\svg@tempa\endcsname}%
      \fi%
    }%
  \svg@tempb%
}
%    \end{macrocode}
% \end{macro}^^A \svgx@get@out@sec
% \begin{macro}{\svgx@ifexpired}
% \changes{v2.03}{2019/07/14}{new}^^A
% The macro compares a file given in the second mandatory argument with a list 
% of files, which is constructed with the base file name of the first mandatory 
% argument and a list of file extensions in the optional argument. If at least 
% one of the files within the list does not exist or is older than that one in 
% the second mandatory argument, the third argument is executed, otherwise the 
% fourth.
%    \begin{macrocode}
\newcommand*\svgx@ifexpired[3][%
  \svgx@format\ifnum\svgx@cnv@mode>\z@,\svgx@cnv@format\fi%
]{%
  \begingroup%
    \@svg@tempswafalse%
    \edef\svg@tempa{%
      \noexpand\svgx@@normalize@list\noexpand\svg@tempa{#1}%
    }%
    \svg@tempa%
    \@for\svg@tempb:=\svg@tempa\do{%
      \svg@IfFileNewer{#2.\svg@tempb}{#3}{}{\@svg@tempswatrue}%
    }%
    \if@svg@tempswa%
      \aftergroup\@firstoftwo%
    \else%
      \aftergroup\@secondoftwo%
    \fi%
  \endgroup%
}
%    \end{macrocode}
% \end{macro}^^A \svgx@ifexpired
% \begin{macro}{\svgx@run}
% Since it's necessary to enable option \file{-{}-shell-escape} for extraction, 
% the command raises a warning if it is not activated. Afterwards, the package 
% is finished.
%    \begin{macrocode}
\ifnum\pdf@shellescape=\@ne\relax\else%
  \newcommand*\svgx@run[3]{%
    \PackageWarning{svg-extract}{%
      You didn't enable `--shell-escape' option so it\MessageBreak%
      wasn't possible to run extraction/conversion for\MessageBreak%
      `#1#2.#3'\MessageBreak%
    }%
  }%
  \expandafter\endinput%
\fi
%    \end{macrocode}
% If \file{-{}-shell-escape} is enabled, the command is defined with its
% intended functionality. It runs all necessary processes to extract and
% convert grahic files.
%    \begin{macrocode}
\newcommand*\svgx@run[3]{%
%    \end{macrocode}
% All generated files will be moved to the desired output folder, which is 
% given by option \opt{extractpath}. Therefor, this folder is created.
%    \begin{macrocode}
  \svg@shell@mkdir{\svgx@out@path}%
%    \end{macrocode}
% After creating the separate auxiliary \LaTeX~file, the actual extraction and 
% conversion can be done. 
%    \begin{macrocode}
  \svgx@generate@texaux{#1}{#2}{#3}%
  \svgx@extr@run{#1#2.#3}%
% Now the desired conversion tool is invoked if requested.
%    \begin{macrocode}
  \ifnum\svgx@cnv@mode>\z@\relax%
    \svgx@conv@run%
  \fi%
%    \end{macrocode}
% As both extraction and conversion are done, all files intended to be 
% extracted are moved to the desired output folder (\opt{extractpath}), if 
% existing. Otherwise a warning is given.
%    \begin{macrocode}
  \@for\svg@tempa:=\svgx@format\do{%
    \svgx@afterdocument@xaddto{%
      \noexpand\svg@shell@mv{\svgx@out@name.\svg@tempa}%
        {\svgx@out@path\svgx@out@name.\svg@tempa}%
    }%
  }%
  \svgx@afterdocument@xaddto{%
    \noexpand\PackageInfoNoLine{svg-extract}{Done for `#1#2.#3'}%
  }%
}
%    \end{macrocode}
% \end{macro}^^A \svgx@run
% \begin{macro}{\svgx@stream@in}
% \begin{macro}{\svgx@read@line}
% \begin{macro}{\svgx@stream@out}
% \begin{macro}{\if@svgx@preamble@write}
% Botah an input stream and an output stream are necessary for this as well as a
% token register, which is used to store all commands which should be executed
% from shell.
%    \begin{macrocode}
\newread\svgx@stream@in
\newcommand*\svgx@read@line{}
\newwrite\svgx@stream@out
\newif\if@svgx@preamble@write
%    \end{macrocode}
% \end{macro}^^A \if@svgx@preamble@write
% \end{macro}^^A \svgx@stream@out
% \end{macro}^^A \svgx@read@line
% \end{macro}^^A \svgx@stream@in
% \begin{macro}{\svgx@generate@texaux}
% \changes{v2.03}{2019/09/07}{new}^^A
% A separate auxiliary \LaTeX~file is created for extracting independent
% graphic files.
%    \begin{macrocode}
\newcommand*\svgx@generate@texaux[3]{%
%    \end{macrocode}
% The specified preamble for this task is searched with \cs{svg@get@file*}.
% It is first searched in the same folder as the SVG~file and if it wasn't 
% found, in any other valid folder for SVG~files. If no preamble file with the 
% specified name is found, the file name is appended with \opt{latexext} as 
% well as \file{tex}. During the search, all file name variations are stored 
% within \cs{svg@tempb} in order to create a warning, if necessary.
%    \begin{macrocode}
  \begingroup%
    \svg@quotes@remove*[\svgx@preamble]{\svg@tempa}%
    \ifx\svg@tempa\@empty%
      \@svg@file@foundfalse%
    \else%
      \let\svg@tempb\@empty%
      \@tfor\svg@tempa:={}{\svgx@latex@ext}{tex}\do{%
        \svg@get@file*[\svg@out@path]{\svgx@preamble}{\svg@tempa}%
        \if@svg@file@found%
          \@break@tfor%
        \fi%
        \svg@quotes@remove{\svg@file@base}%
        \svgx@ifin@list{\svg@tempb}{\svg@file@base}{}{%
          \edef\svg@tempb{\svg@tempb,\svg@file@base}%
        }%
      }%
%    \end{macrocode}
% If a file was found and it's name is identical to \cs{jobname}, the default 
% extension is appended.
%    \begin{macrocode}
      \if@svg@file@found%
        \ifstr{\svg@file@base}{\@currdir\jobname}{%
          \edef\svg@file@base{\svg@file@base.tex}%
        }{}%
%    \end{macrocode}
% If no file was found by the given name, a warning is issued.
%    \begin{macrocode}
      \else%
        \svg@remove@leadingchar,\svg@tempb%
        \let\svg@tempa\@empty%
        \@for\svg@tempb:=\svg@tempb\do{%
          \edef\svg@tempa{%
            \etex@unexpanded\expandafter{\svg@tempa}%
            \noexpand\MessageBreak`\svg@tempb'%
          }%
        }%
        \PackageWarning{svg-extract}{%
          The preamble file(s)\svg@tempa\MessageBreak could not be found%
        }%
      \fi%
    \fi%
    \edef\svg@tempa{%
      \endgroup%
      \if@svg@file@found%
        \def\noexpand\svgx@preamble{\svg@file@base}%
      \else%
        \let\noexpand\svgx@preamble\noexpand\@empty%
      \fi%
    }%
  \svg@tempa%
%    \end{macrocode}
% The given preamble file is read line by line and every line not enclosed 
% within the two macros \cs{svghidepreamblestart} and \cs{svghidepreambleend} 
% is temporarily stored in the auxiliary \LaTeX~file. The used file name is
% composed with \file{\cs{svgx@out@name}.\cs{svgx@latex@ext}} and expanded to
% \cs{svgx@out@name} within a local scope for an easier usage.
%    \begin{macrocode}
  \begingroup%
    \endlinechar=\m@ne%
    \svg@quotes@remove{\svgx@preamble}%
    \svg@quotes@remove[\svgx@out@name.\svgx@latex@ext]{\svgx@out@name}%
    \ifx\svgx@preamble\@empty%
      \PackageInfo{svg-extract}{%
        No specific preamble is used for the generation of\MessageBreak%
        the auxiliary file `\svgx@out@name'%
        \if@svg@ink@latex%
          \MessageBreak so the result of extraction/conversion will probably
          \MessageBreak look different from the appearance in this file%
        \fi%
      }%
    \else%
      \PackageInfo{svg-extract}{%
        The preamble of file `\svgx@preamble'\MessageBreak%
        is used for the generation of the auxiliary file\MessageBreak% 
        `\svgx@out@name'%
      }%
%    \end{macrocode}
% The catcodes for |#| need to be changed to prevent doublification when
% reading the line.
%    \begin{macrocode}
      \catcode`\#=12\relax%
      \immediate\openout\svgx@stream@out="\svgx@out@name"%
      \immediate\openin\svgx@stream@in="\svgx@preamble"%
      \@svg@tempswatrue%
      \@svgx@preamble@writetrue%
      \let\svgx@read@line\@empty%
%    \end{macrocode}
% The given preamble file is read line by line and written to the separate 
% auxiliary \LaTeX~file via the output stream.
%    \begin{macrocode}
      \@whilesw\if@svg@tempswa\fi{%
        \immediate\read\svgx@stream@in to\svgx@read@line%
        \ifx\svgx@read@line\@empty%
          \ifeof\svgx@stream@in\@svg@tempswafalse\fi%
        \else%
%    \end{macrocode}
% With \cs{svghidepreamblestart} and \cs{svghidepreambleend} it is possible for 
% the user to omit certain parts of the preamble. Therefor the two macros
% \cs{svgx@read@preamble@till} and \cs{svgx@read@preamble@from} are toggling
% the switch \cs{if@svgx@preamble@write}
%    \begin{macrocode}
          \svgx@read@preamble@till{\svghidepreamblestart}{}%
          \svgx@read@preamble@from{\svghidepreambleend}{}%
%    \end{macrocode}
% If the desired end of the preamble (\cs{svgx@endpreamble}) was found, the 
% readout is terminated by switching \cs{if@svg@tempswa} to \val{false}.
%    \begin{macrocode}
          \svgx@read@preamble@till{\svgx@endpreamble}{\@svg@tempswafalse}%
          \if@svgx@preamble@write%
%    \end{macrocode}
% During the readout process, it is searched with \cs{svgx@documentclass} for 
% the appearance of \cs{documentclass} and \cs{if@svgx@classfound} is set to 
% \val{true} if it was found.
%    \begin{macrocode}
            \if@svgx@classfound\else%
              \expandafter\svgx@documentclass%
                \svgx@read@line\documentclass\documentclass\@nil%
            \fi%
%    \end{macrocode}
% Writing out the---maybe manipulated---read in line.
%    \begin{macrocode}
            \ifx\svgx@read@line\@empty\else%
              \immediate\write\svgx@stream@out{%
                \etex@unexpanded\expandafter{\svgx@read@line}%
              }%
            \fi%
          \fi%
        \fi%
      }%
      \immediate\closein\svgx@stream@in%
      \immediate\closeout\svgx@stream@out%
      \catcode`\#=6\relax%
    \fi%
%    \end{macrocode}
% Once the separate auxiliary \LaTeX~file was written, it is read in again and 
% its content is stored in \cs{svg@tempa}, since it is necessary to prepend
% some stuff to the preamble, for example a maybe not existent document class.
%    \begin{macrocode}
    \let\svg@tempa\@empty%
    \ifx\svgx@preamble\@empty\else%
      \immediate\openin\svgx@stream@in="\svgx@out@name"%
      \loop\unless\ifeof\svgx@stream@in%
        \readline\svgx@stream@in to\svgx@read@line%
        \ifx\svgx@read@line\@empty\else%
          \edef\svg@tempa{%
            \etex@unexpanded\expandafter{\svg@tempa}%
            \etex@unexpanded\expandafter{\svgx@read@line}^^J%
          }%
        \fi%
      \repeat%
      \immediate\closein\svgx@stream@in%
    \fi%
%    \end{macrocode}
% After the preamble was read in and stored in \cs{svg@tempa}, the separate 
% auxiliary \LaTeX~file is written again. Some information are written right at
% the beginning of the file.
%    \begin{macrocode}
    \immediate\openout\svgx@stream@out="\svgx@out@name"%
    \def\svg@tempb{\@percentchar\@percentchar\space}%
    \immediate\write\svgx@stream@out{%
      \svg@tempb This file `\svgx@out@name' was generated^^J%
      \svg@tempb by package `svg-extract' from source `\jobname'^^J%
      \svg@tempb It's intended to be compiled with\space%
      `\svgx@latex@exe\ifx\svgx@latex@opt\@empty\else\space\svgx@latex@opt\fi'%
    }%
%    \end{macrocode}
% With the intention of passing the correct paper dimensions, the calculating 
% of the paper size is executed with \cs{AtBeginDocument} even before the
% document class, so that this is definitely the first thing to happen at the
% beginning of the document. Additionally, it is ensured that the \cs{special} 
% command is definitely used with the correct paper size, when creating a 
% DVI~file.
%    \begin{macrocode}
    \immediate\write\svgx@stream@out{%
      \string\AtBeginDocument{\@percentchar^^J%
        \space\space\string\svgxsetpapersize\@percentchar^^J%
        \ifxetex\else\ifpdf\else%
          \space\space\string\AtBeginDvi{\string\special{%
              papersize=\string\the\string\paperwidth,%
                \string\the\string\paperheight%
          }}\@percentchar^^J%
        \fi\fi%
      }^^J%
      \string\PassOptionsToPackage{hidelinks}{hyperref}%
    }%
%    \end{macrocode}
% If no document class was found during reading the preamble file, then class 
% \cs{article} is used.
%    \begin{macrocode}
    \if@svgx@classfound\else%
      \immediate\write\svgx@stream@out{\string\documentclass{article}}%
    \fi%
%    \end{macrocode}
% And now the stored preamble.
%    \begin{macrocode}
    \ifx\svg@tempa\@empty\else%
      \immediate\write\svgx@stream@out{%
        \etex@unexpanded\expandafter{\svg@tempa}%
      }%
    \fi%
%    \end{macrocode}
% After the given preamble was written, package \pkg{svg-extract} will be 
% loaded in case it was forgotten.
%    \begin{macrocode}
    \immediate\write\svgx@stream@out{\string\usepackage{svg-extract}}%
%    \end{macrocode}
% Package \pkg{xr} is used to evaluate possible labels within the included
% \app{Inkscape}~\LaTeX~file.
%    \begin{macrocode}
    \if@svg@ink@latex%
      \IfFileExists{xr.sty}{%
        \immediate\write\svgx@stream@out{%
          \string\usepackage{xr}^^J%
          \string\externaldocument{\jobname}^^J%
        }%
      }{}%
    \fi%
%    \end{macrocode}
% Now all parameters relevant for the extraction are evaluated and appended.
%    \begin{macrocode}
    \def\svg@tempa##1{%
      \immediate\write\svgx@stream@out{\string\svgsetup{##1}}%
    }%
    \if@svg@ink@latex\else%
      \svg@tempa{inkscapelatex=false}%
    \fi%
    \ifdim\svgx@param@width>\z@\relax%
      \svg@tempa{width=\svgx@param@width}%
    \fi%
    \ifdim\svgx@param@height>\z@\relax%
      \svg@tempa{height=\svgx@param@height}%
    \fi%
    \if@svgx@param@distort%
      \svg@tempa{distort=true}%
    \fi%
    \ifdim\dimexpr\svgx@param@scale\p@\relax=\p@\relax\else%
      \svg@tempa{scale=\svgx@param@scale}%
    \fi%
    \def\svg@tempb{\svg@param@pretex}%
    \ifx\svgx@param@pretex\svg@tempb%
      \let\svgx@param@pretex\svg@param@pretex%
    \fi%
    \ifx\svgx@param@pretex\relax\else%
      \svg@tempa{pretex=\etex@unexpanded\expandafter{\svgx@param@pretex}}%
    \fi%
    \def\svg@tempb{\svg@param@apptex}%
    \ifx\svgx@param@apptex\svg@tempb%
      \let\svgx@param@apptex\svg@param@apptex%
    \fi%
    \ifx\svgx@param@apptex\relax\else%
      \svg@tempa{apptex=\etex@unexpanded\expandafter{\svgx@param@apptex}}%
    \fi%
%    \end{macrocode}
% Parameter \prm{lastpage} is only considered for including PDF~files with 
% \LaTeX~support.
%    \begin{macrocode}
    \let\svg@tempa\@empty%
    \ifstr{#3}{pdf}{%
      \if@svg@ink@latex%
        \ifnum\value{svg@param@lastpage}>\z@\relax%
          \edef\svg@tempa{lastpage=\the\value{svg@param@lastpage}}%
        \else%
          \ifnum\value{svg@param@lastpage}=\z@\relax%
            \def\svg@tempa{lastpage=true}%
          \else%
            \def\svg@tempa{lastpage=false}%
          \fi%
        \fi%
      \fi%
    }{}%
%    \end{macrocode}
% The rotation angle, if given.
%    \begin{macrocode}
    \ifdim\dimexpr\svgx@param@angle\p@\relax=\z@\relax\else%
      \edef\svg@tempa{%
        angle=\svgx@param@angle\ifx\svg@tempa\@empty\else,\svg@tempa\fi%
      }%
    \fi%
%    \end{macrocode}
% As we are now at the end of the preamble and just before the beginning of the 
% document, the paper dimension are set again to make sure, that these settings 
% are active at the end of the preamble. Additionally, it is executed again at 
% the very end of \cs{AtBeginDocument} to ensure, that no other package used 
% this hook for manipulating the paper size.
%    \begin{macrocode}
    \edef\svg@tempa{%
      \noexpand\string\noexpand\svgxsetbox%
      \ifx\svg@tempa\@empty\else[\svg@tempa]\fi%
      {#1}{#2}{#3}%
    }%
    \immediate\write\svgx@stream@out{\svg@tempa}%
%    \end{macrocode}
% The actual document body is simple.
%    \begin{macrocode}
    \immediate\write\svgx@stream@out{%
      \string\begin{document}^^J%
      \string\pagestyle{empty}^^J%
      \string\svgxoutputbox\@percentchar^^J%
      \string\end{document}%
    }%
    \immediate\closeout\svgx@stream@out%
  \endgroup%
}
%    \end{macrocode}
% \end{macro}^^A \svgx@generate@texaux
% \begin{macro}{\svgx@documentclass}
% \begin{macro}{\if@svgx@classfound}
% This delimited macro is used to find the occurrence of \cs{documentclass} 
% within a read in line. The delinmiter \cs{documentclass} is used twice in
% order to ignore the possible occurrence of white space or anything else right
% before \cs{documentclass}.
%    \begin{macrocode}
\newif\if@svgx@classfound
\newcommand*\svgx@documentclass{}
\def\svgx@documentclass#1\documentclass#2\documentclass#3\@nil{%
  \IfArgIsEmpty{#2}{}{\@svgx@classfoundtrue}%
}
%    \end{macrocode}
% \end{macro}^^A \if@svgx@classfound
% \end{macro}^^A \svgx@documentclass
% \begin{macro}{\svgx@read@preamble@till}
% \begin{macro}{\svgx@read@preamble@from}
% \begin{macro}{\svgx@read@preamble@skip}
% These macros are used to skip some parts of a read in preamble file.
%    \begin{macrocode}
\newcommand*\svgx@read@preamble@till[2]{%
  \svgx@read@preamble@skip#1\@nil{till}{#2}%
}
\newcommand*\svgx@read@preamble@from[2]{%
  \svgx@read@preamble@skip#1\@nil{from}{#2}%
}
%    \end{macrocode}
% In principle, the functionality is the same as for \cs{svgx@documentclass}.
%    \begin{macrocode}
\newcommand*\svgx@read@preamble@skip{}
\def\svgx@read@preamble@skip#1\@nil#2#3{%
%    \end{macrocode}
% A given token is used to create the macro \cs{svg@tempa} delimited by the 
% token itself which is used twice to get any stuff right before or after the
% occurrence.
%    \begin{macrocode}
  \def\svg@tempa##1{%
    \def\svg@tempa####1##1####2##1####3\@nil{%
      \IfArgIsEmpty{####3}{}{%
%    \end{macrocode}
% Write everything which was found right before the macro which starts hiding 
% area to the output stream and stop writing with \cs{if@svgx@preamble@write}.
%    \begin{macrocode}
        \ifstr{#2}{till}{%
          \IfArgIsEmpty{####1}{}{%
            \immediate\write\svgx@stream@out{####1}%
          }%
          \@svgx@preamble@writefalse%
        }{%
%    \end{macrocode}
% Write everything which was found right after the macro which ends the hiding 
% area and start writing again with \cs{if@svgx@preamble@write}.
%    \begin{macrocode}
          \ifstr{#2}{from}{%
            \IfArgIsEmpty{####2}{%
              \let\svgx@read@line\@empty%
            }{%
              \def\svgx@read@line{####2}%
            }%
            \@svgx@preamble@writetrue%
          }{}%
        }%
%    \end{macrocode}
% Additonal stuff which should be done.
%    \begin{macrocode}
        #3%
      }%
    }%
  }%
%    \end{macrocode}
% Creating the macro \cs{svg@tempa} delimited by the first argument.
%    \begin{macrocode}
  \edef\svg@tempb{\expandafter\detokenize\expandafter{#1}}%
  \expandafter\svg@tempa\expandafter{\svg@tempb}%
%    \end{macrocode}
% Calling the created macro.
%    \begin{macrocode}
  \edef\svg@tempb{%
    \expandafter\detokenize\expandafter{\svgx@read@line}\svg@tempb\svg@tempb%
  }%
  \expandafter\svg@tempa\svg@tempb\@nil%
}
%    \end{macrocode}
% \end{macro}^^A \svgx@read@preamble@skip
% \end{macro}^^A \svgx@read@preamble@from
% \end{macro}^^A \svgx@read@preamble@till
% \begin{macro}{\svgx@extr@run}
% \changes{v2.03}{2019/09/07}{new}^^A
% First of all the separate auxiliary \LaTeX~file is compiled with the detected 
% \LaTeX~processor (\cs{svgx@latex@exe}) as often as defined by counter option 
% \opt{extractruns}.
%    \begin{macrocode}
\newcommand*\svgx@extr@run[1]{%
  \begingroup%
    \svg@quotes@remove{\svgx@out@name}%
    \svgx@afterdocument@xaddto{%
      \noexpand\PackageInfoNoLine{svg-extract}{Extracting `#1'}%
      \noexpand\PackageInfoNoLine{svg-extract}{%
        Executing LaTeX (\svgx@latex@exe) on\MessageBreak%
        `\svgx@out@name.\svgx@latex@ext'%
        \ifx\svgx@latex@opt\@empty\else%
          \MessageBreak with added options `\svgx@latex@opt'%
        \fi%
      }%
    }%
    \@tempcnta=\value{svgx@iterations}%
    \loop\ifnum\@tempcnta>\z@\relax%
      \svgx@afterdocument@xaddto{%
        \noexpand\ShellEscape{%
          \svgx@latex@exe\space%
          \ifx\svgx@latex@opt\@empty\else\svgx@latex@opt\space\fi%
          "\svgx@out@name.\svgx@latex@ext"%
        }%
      }%
      \advance\@tempcnta\m@ne%
    \repeat%
%    \end{macrocode}
% All files requested with option \opt{extractformat} are created with internal 
% conversion tools supplied by most \LaTeXe~distributions if necessary.
%    \begin{macrocode}
    \def\svg@tempa##1##2{%
      \svgx@afterdocument@xaddto{%
        \noexpand\PackageInfoNoLine{svg-extract}{%
          Executing ##1 on\MessageBreak%
          `\svgx@out@name.##2'%
        }%
        \noexpand\ShellEscape{%
          \@nameuse{svgx@##1@exe}\space\@nameuse{svgx@##1@opt}\space%
          "\svgx@out@name.##2"%
        }%
      }%
    }%
    \@svg@tempswafalse%
    \ifxetex\else\ifpdf\else%
      \@svg@tempswatrue%
    \fi\fi%
    \if@svg@tempswa%
      \svg@tempa{dvips}{dvi}%
      \svgx@ifin@list{\svgx@format}{eps}{\svg@tempa{pstoeps}{ps}}{}%
      \svgx@ifin@list{\svgx@format}{pdf}{\svg@tempa{pstopdf}{ps}}{}%
    \else%
      \svgx@ifin@list{\svgx@format}{eps}{\svg@tempa{pdftoeps}{pdf}}{}%
      \svgx@ifin@list{\svgx@format}{ps}{\svg@tempa{pdftops}{pdf}}{}%
    \fi%
    \ifx\svgx@format\@empty%
      \let\svg@tempa\svgx@cnv@input@format%
    \else%
      \let\svg@tempa\svgx@format%
    \fi%
    \@for\svg@tempa:=\svg@tempa\do{%
      \edef\svg@tempb{\svgx@out@name.\svg@tempa}%
      \svgx@afterdocument@xaddto{%
        \noexpand\IfFileExists{"\@currdir\svg@tempb"}{}{%
          \noexpand\svg@troubleshooting{svg-extract}%
            {Extracting}{#1}{\svg@tempb}%
        }%
      }%
    }%
  \endgroup%
}
%    \end{macrocode}
% \end{macro}^^A \svgx@extr@run
% \begin{macro}{\svgx@conv@run}
% \changes{v2.03}{2019/09/07}{new}^^A
% \begin{macro}{\svgx@cnv@input@format}
% \begin{macro}{\svgx@cnv@get@input@format}
% \changes{v2.03}{2019/07/11}{sets \cs{svgx@format} implicitly if necessary}^^A
% The macro \cs{svgx@cnv@get@input@format} extracts the first list entry from 
% \cs{svgx@format}. 
%    \begin{macrocode}
\newcommand*\svgx@cnv@input@format{}
\newcommand*\svgx@cnv@get@input@format{%
  \begingroup%
    \def\svg@tempa##1,##2\@nil{%
      \def\svg@tempa{##1}%
    }%
    \expandafter\svg@tempa\svgx@format,\@nil%
    \edef\svg@tempa{%
      \endgroup%
      \def\noexpand\svgx@cnv@input@format{\svg@tempa}%
    }%
  \svg@tempa%
%    \end{macrocode}
% If \cs{svgx@format} was empty, \cs{svgx@cnv@input@format} is set to the 
% file type, which is generated anyway\dots
%    \begin{macrocode}
  \ifx\svgx@cnv@input@format\@empty%
    \renewcommand*\svgx@cnv@input@format{pdf}%
    \ifxetex\else\ifpdf\else%
      \renewcommand*\svgx@cnv@input@format{ps}%
    \fi\fi%
%    \end{macrocode}
% \dots and if this file type is included in list \cs{svgx@cnv@format} then
% \cs{svgx@format} will be set to this file type as well in order to skip the
% unnecessary conversation.
%    \begin{macrocode}
    \svgx@ifin@list{\svgx@cnv@format}{\svgx@cnv@input@format}{%
      \let\svgx@format\svgx@cnv@input@format%
    }{}%
  \fi%
}
%    \end{macrocode}
% Running the conversion on extracted files.
%    \begin{macrocode}
\newcommand*\svgx@conv@run{%
  \begingroup%
%    \end{macrocode}
% Before the conversion tool is invoked, it is checked, if all desired 
% extraction files exist. At least, the input file for conversion.
%    \begin{macrocode}
    \svgx@cnv@get@input@format%
    \svg@quotes@remove{\svgx@out@path}%
    \svg@quotes@remove{\svgx@out@name}%
%    \end{macrocode}
% If no density was given at all, the density for PNG~files is set to 
% \val{300dpi} by default.
%    \begin{macrocode}
    \ifx\svgx@cnv@dpi\relax%
      \ifx\svgx@cnv@dpi@png\@undefined%
        \def\svgx@cnv@dpi@png{300}%
      \fi%
    \fi%
%    \end{macrocode}
% The conversion is done for each desired file type given in a list by option 
% \opt{convertformat}.
%    \begin{macrocode}
    \@for\svg@tempa:=\svgx@cnv@format\do{%
      \svgx@ifin@list{\svgx@format}{\svg@tempa}{%
        \svgx@afterdocument@xaddto{%
          \noexpand\PackageInfoNoLine{svg-extract}{%
            Generating file `\svgx@out@name.\svg@tempa'\MessageBreak%
            was specified with `extractformat' (\svgx@format)\MessageBreak%
            as well as with `convertformat' (\svgx@cnv@format)\MessageBreak%
            so the conversion won't be done%
          }%
        }%
      }{%
        \edef\svg@tempb{\svgx@out@path\svgx@out@name}%
        \svgx@afterdocument@xaddto{%
          \noexpand\PackageInfoNoLine{svg-extract}{%
            Converting (\svg@tempa) `\svgx@out@name.\svgx@cnv@input@format'%
          }%
          \noexpand\ShellEscape{%
            \svgx@cnv@cmd{\svgx@out@name}{\svgx@cnv@input@format}%
              {\svgx@out@path}{\svg@tempa}%
          }%
          \noexpand\svg@IfFileExists{\svg@tempb.\svg@tempa}{}{%
            \noexpand\svg@troubleshooting{svg-extract}%
              {Converting (\svg@tempa)}%
              {\svgx@out@name.\svgx@cnv@input@format}%
              {\svg@tempb.\svg@tempa}%
          }%
        }%
      }%
    }%
  \endgroup%
}
%    \end{macrocode}
% \end{macro}^^A \svgx@cnv@get@input@format
% \end{macro}^^A \svgx@cnv@input@format
% \end{macro}^^A \svgx@conv@run
% \begin{macro}{\svgx@magick@cmd}
% \changes{v2.03}{2019/07/11}{additional argument for output path}^^A
% \begin{macro}{\svgx@gs@cmd}
% \changes{v2.03}{2019/07/11}{additional argument for output path}^^A
% Depending on option \opt{convert}, one of these two macros is actually used
% by \cs{svgx@cnv@cmd}. For invoking the conversion process, the required 
% platform-dependent executable is set, if nothing was set by a package option.
%    \begin{macrocode}
\ifx\svgx@magick@exe\@empty
  \ifwindows
    \renewcommand*\svgx@magick@exe{magick}
  \else
    \renewcommand*\svgx@magick@exe{convert}
  \fi
\fi
\newcommand*\svgx@magick@cmd[5]{%
  \svgx@magick@exe\space%
  \svgx@useformatkey{svgx@cnv@dpi}{#5}{-density }%
  \svgx@useformatkey{svgx@magick@set}{#5}{}%
  "#2#1.#3"\space%
  \svgx@useformatkey{svgx@magick@opr}{#5}{}%
  "#4#1.#5"%
}
%    \end{macrocode}
%    \begin{macrocode}
\ifx\svgx@gs@exe\@empty
  \ifwindows
    \renewcommand*\svgx@gs@exe{gswin64c}
  \else
    \renewcommand*\svgx@gs@exe{gs}
  \fi
\fi
\newcommand*\svgx@gs@cmd[5]{%
  \svgx@gs@exe\space-dSAFER -dBATCH -dNOPAUSE\space%
  \svgx@useformatkey{svgx@gs@device}{#5}{-sDEVICE=}%
  \svgx@useformatkey{svgx@cnv@dpi}{#5}{-r}%
  \svgx@useformatkey{svgx@gs@opt}{#5}{}%
  -sOutputFile="#4#1.#5"\space"#2#1.#3"%
}
%    \end{macrocode}
% \end{macro}^^A \svgx@gs@cmd
% \end{macro}^^A \svgx@magick@cmd
%
% \iffalse
%</package&extract&body>
%<*package&extract&option>
% \fi
%
% \begin{macro}{\svgx@normalize@list}
% \changes{v2.03}{2019/07/13}{new}^^A
% \begin{macro}{\svgx@@normalize@list}
% \changes{v2.03}{2019/07/13}{new}^^A
% This macro is used to normalize a comma-separated list given in the second 
% argument, wherefore empty values and unnecessary spaces are removed. At the 
% end, the normalized list is expanded to the macro in the first or second
% argument. Command \cs{svgx@normalize@list} takes a macro in the second 
% argument and \cs{svgx@@normalize@list} needs a list itself.
%    \begin{macrocode}
\newcommand*\svgx@normalize@list[2][]{%
  \begingroup%
    \edef\svg@tempa{%
      \endgroup%
      \etex@unexpanded{\svgx@@normalize@list#2}{%
        \IfArgIsEmpty{#1}{%
          \etex@unexpanded\expandafter{#2}%
        }{%
          \etex@unexpanded\expandafter{#1}%
        }%
      }%
    }%
  \svg@tempa%
}
\newcommand*\svgx@@normalize@list[2]{%
  \begingroup%
    \let\svg@tempb\@empty%
    \@for\svg@tempa:={,#2}\do{%
      \edef\svg@tempa{\expandafter\trim@spaces@noexp\expandafter{\svg@tempa}}%
      \ifx\svg@tempa\@empty\else%
        \edef\svg@tempa{%
          \etex@unexpanded{\l@addto@macro\svg@tempb}%
            {,\etex@unexpanded\expandafter{\svg@tempa}}%
        }\svg@tempa%
      \fi%
    }%
    \ifx\svg@tempb\@empty%
      \def\svg@tempa{%
        \endgroup%
        \let#1\@empty%
      }%
%    \end{macrocode}
% Remove leading comma.
%    \begin{macrocode}
    \else%
      \def\svg@tempa,##1\@nil{%
        \def\svg@tempa{%
          \endgroup%
          \def#1{##1}%
        }%
      }%
      \expandafter\svg@tempa\svg@tempb\@nil%
    \fi%
  \svg@tempa%
}
%    \end{macrocode}
% \end{macro}^^A \svgx@@normalize@list
% \end{macro}^^A \svgx@normalize@list
% \begin{macro}{\svgx@ifin@list}
% \changes{v2.03}{2019/08/01}{swapped arguments}^^A
% \begin{macro}{\svgx@@ifin@list}
% \changes{v2.03}{2019/07/24}{new}^^A
% Check, if the first argument is included in a comma-separated list in the 
% second argument, whereas \cs{svgx@ifin@list} expands it's arguments and 
% \cs{svgx@@ifin@list} does not.
%    \begin{macrocode}
\newcommand*\svgx@ifin@list[2]{%
  \begingroup%
    \edef\svg@tempa{%
      \endgroup%
      \noexpand\svgx@@ifin@list{#1}{#2}%
    }%
  \svg@tempa%
}
\newcommand*\svgx@@ifin@list[2]{%
  \begingroup%
    \edef\svg@tempa{%
      \def\noexpand\svg@tempa####1,\detokenize{#2},####2\noexpand\@nil{%
        \def\noexpand\svg@tempb{####2}%
      }%
      \def\noexpand\svg@tempb{%
        ,\detokenize{#1},\detokenize{#2},%
      }%
    }%
    \svg@tempa%
    \expandafter\svg@tempa\svg@tempb\@nil%
    \ifx\svg@tempb\@empty%
      \aftergroup\@secondoftwo%
    \else%
      \aftergroup\@firstoftwo%
    \fi%
  \endgroup%
}
%    \end{macrocode}
% \end{macro}^^A \svgx@@ifin@list
% \end{macro}^^A \svgx@ifin@list
% \begin{macro}{\svgx@removekey@list}
% \changes{v2.03}{2019/08/01}{new}^^A
% This macro removes an entry or key (and it's value if existing) from a list.
%    \begin{macrocode}
\newcommand*\svgx@removekey@list[2]{%
  \begingroup%
    \def\svg@tempa##1,#2##2,##3\@nil{%
      \IfArgIsEmpty{##3}{\let\svg@tempb#1}{%
        \def\svg@tempa####1,#2,\@nil{%
          \l@addto@macro\svg@tempb{,####1}%
        }%
        \def\svg@tempb{##1}%
        \svg@tempa,##3\@nil%
      }%
    }%
    \expandafter\svg@tempa\expandafter,#1,#2,\@nil%
    \svgx@normalize@list{\svg@tempb}%
    \edef\svg@tempa{%
      \endgroup%
      \def\noexpand#1{\etex@unexpanded\expandafter{\svg@tempb}}%
    }%
  \svg@tempa%
}
%    \end{macrocode}
% \end{macro}^^A \svgx@removekey@list
% \begin{macro}{\svgx@ifvalueisinherit}
% \changes{v2.03}{2019/07/29}{new}^^A
%    \begin{macrocode}
\newcommand*\svgx@ifvalueisinherit[1]{%
  \begingroup%
    \def\svg@tempa{#1}%
    \def\svg@tempb{inherit}%
    \ifx\svg@tempa\svg@tempb%
      \aftergroup\@firstoftwo%
    \else%
      \aftergroup\@secondoftwo%
    \fi%
  \endgroup%
}
%    \end{macrocode}
% \end{macro}^^A \svgx@ifvalueisinherit
% \begin{macro}{\svgx@onlywindows}
% Do only some stuff, if Windows was detected.
%    \begin{macrocode}
\newcommand*\svgx@onlywindows[1]{}
\AfterPackage*{ifplatform}{\renewcommand*\svgx@onlywindows[1]{\ifwindows#1\fi}}
%    \end{macrocode}
% \end{macro}^^A \svgx@onlywindows
% \begin{macro}{\svgx@ifkeyandval}
% It is checked whether a key was given as \val{\meta{key}=\meta{value}} or 
% like \val{\meta{key}=\{\meta{format}=\meta{value}\}}.
%    \begin{macrocode}
\newcommand*\svgx@ifkeyandval[3]{%
  \def\svg@tempa##1=##2=##3\@nil{\IfArgIsEmpty{##3}{#3}{#2}}%
  \svg@tempa#1==\@nil%
}
%    \end{macrocode}
% \end{macro}^^A \svgx@ifkeyandval
% \begin{macro}{\svgx@cnv@get@dpi}
% This macro is used to resolve a given value to set the density for the 
% conversion. The delimited macros \cs{svg@tempa} and \cs{svg@tempb} are 
% defined to first crop any given suffix \val{dpi} and second to split two 
% numbers at |x|, if present. Pay attention how both macros are invoked. In the
% end, a passed value in any of the forms \val{300}, \val{300dpi}, \val{300x400}
% or \val{300x400dpi} and even \val{300dpix400dpi} is possible. The result is
% stored in \cs{svg@tempa}.
%    \begin{macrocode}
\newcommand*\svgx@cnv@get@dpi[1]{%
  \begingroup%
    \def\svg@tempa##1dpi##2x##3dpi##4\@nil{%
      \edef\svg@tempa{##1}%
%    \end{macrocode}
% Switch \cs{if@svg@tempswa} as \cs{iftrue} means, a valid value was found.
%    \begin{macrocode}
      \@svg@tempswafalse%
%    \end{macrocode}
% If only the first argument is a number and third is empty, a single number 
% was given and there's nothing more to do. If the argument is something like 
% \val{300dpix400dpi}, the third argument is the second number.
%    \begin{macrocode}
      \svg@ifnumber{##1}{%
        \IfArgIsEmpty{##3}{\@svg@tempswatrue}{%
          \svg@ifnumber{##3}{\edef\svg@tempa{##1x##3}}{}%
        }%
      }{}%
      \if@svg@tempswa\else%
        \expandafter\svg@tempb\svg@tempa xx\@nil%
      \fi%
    }%
%    \end{macrocode}
% Macro \cs{svg@tempb} splits at |x| and checks, if something valid like 
% \val{300x400} was given. If true, the value is stored in \cs{svg@tempa}.
%    \begin{macrocode}
    \def\svg@tempb##1x##2x##3\@nil{%
      \ifstr{##3}{x}{%
        \@svg@tempswatrue%
        \IfArgIsEmpty{##1}{\@svg@tempswafalse}{%
          \svg@ifnumber{##1}{}{\@svg@tempswafalse}%
        }%
        \IfArgIsEmpty{##2}{\@svg@tempswafalse}{%
          \svg@ifnumber{##2}{}{\@svg@tempswafalse}%
        }%
        \if@svg@tempswa%
          \edef\svg@tempa{##1x##2}%
        \fi%
      }{}%
    }%
    \IfArgIsEmpty{#1}{%
      \let\svg@tempa\@empty%
    }{%
      \lowercase{\svg@tempa#1dpi#1xdpi\@nil}%
      \if@svg@tempswa\else%
        \let\svg@tempa\relax%
      \fi%
    }%
    \edef\svg@tempb{%
      \endgroup%
      \ifx\svg@tempa\relax%
        \let\noexpand\svg@tempa\noexpand\relax%
      \else%
        \def\noexpand\svg@tempa{\svg@tempa}%
      \fi%
    }%
  \svg@tempb%
}
%    \end{macrocode}
% \end{macro}^^A \svgx@cnv@get@dpi
% \begin{macro}{\svgx@setformatkey}
% \changes{v2.03}{2019/07/29}{check for empty keys, sets \cs{FamilyKeyState}}^^A
% \begin{macro}{\svgx@useformatkey}
% With \cs{svgx@setformatkey} the---maybe output format depend---keys for the 
% conversion tools are set. First argument contains the value given to a key, 
% second the command sequence name of the macro, to whom the value shall be 
% allocated.
%    \begin{macrocode}
\newcommand*\svgx@setformatkey[2]{%
%    \end{macrocode}
% A key of the form \val{\meta{key}=\{\meta{format}=\meta{value}\}} is given. 
% The desired output format can be accessed with |##1|, the value with |##2|
% within the arguments of \cs{svgx@ifkeyandval}.
%    \begin{macrocode}
  \svgx@ifkeyandval{#1}{%
    \IfArgIsEmpty{##1}{}{%
      \edef\svg@tempa{\trim@spaces{##1}}%
      \svg@ifvalueisrelax{##2}{%
        \expandafter\let\csname #2@\svg@tempa\endcsname\relax%
      }{%
        \@namedef{#2@\svg@tempa}{##2}%
      }%
      \FamilyKeyStateProcessed%
    }%
%    \end{macrocode}
% A key of the form \val{\meta{key}=\{\meta{format}=\meta{value}\}} is given.
% The value can be used with |##1|.
%    \begin{macrocode}
  }{%
    \svg@ifvalueisrelax{##1}{%
      \expandafter\let\csname #2\endcsname\relax%
    }{%
      \@namedef{#2}{##1}%
    }%
    \FamilyKeyStateProcessed%
  }%
}
%    \end{macrocode}
% The command \cs{svgx@useformatkey} checks, if a format specific key was 
% definded with \cs{svgx@setformatkey}, whereas the format is given in the 
% second argument. If this is not the case, the setting for all output formats 
% is used. After that, a specific key appended with a |+| can be used to do 
% some additional stuff.
%    \begin{macrocode}
\newcommand*\svgx@useformatkey[3]{%
  \scr@ifundefinedorrelax{#1@#2}{%
    \scr@ifundefinedorrelax{#1}{}{%
      \expandafter\ifx\csname #1\endcsname\@empty\else%
        #3\@nameuse{#1}\space%
      \fi%
    }%
    \scr@ifundefinedorrelax{#1@#2+}{}{%
      \expandafter\ifx\csname #1@#2+\endcsname\@empty\else%
        #3\@nameuse{#1@#2+}\space%
      \fi%
    }%
  }{%
%    \end{macrocode}
% If this a format specific key was definded, it is used.
%    \begin{macrocode}
    \expandafter\ifx\csname #1@#2\endcsname\@empty\else%
      #3\@nameuse{#1@#2}\space%
    \fi%
  }%
}
%    \end{macrocode}
% \end{macro}^^A \svgx@useformatkey
% \end{macro}^^A \svgx@setformatkey
%
%
%
% \subsection{Commands for the separate auxiliary \LaTeX-file}
%
% For the extraction of independent graphics, an auxiliary \LaTeX~file is 
% needed. Within this file, the following commands are used to include the 
% desired graphic.
%
% \begin{macro}{\svgxsetbox}
% \changes{v2.02}{2018/09/07}{late execution of \cs{svgxsetpapersize}}^^A
% \begin{macro}{\svgx@setbox}
% \changes{v2.02}{2018/09/07}{new}^^A
% \begin{macro}{\if@svgx@standalone}
% \changes{v2.02}{2018/09/07}{new}^^A
% Within the preamble of the auxiliary \LaTeX~file, the desired grahic is used 
% to setup a box, which is used both to define the papersize as well as for the 
% output itself. The macro \cs{svgx@setbox} is executed twice, the first time 
% in the preamble and the second time at the very end of \cs{AtBeginDocument} 
% if package \pkg{etoolbox} was loaded.
%
% The switch \cs{if@svgx@standalone} is defined for enabling classes to 
% implement a different behavoiur for \pkg{svg-extract} in standalone mode.
% for example, \TUDScript-classes are using this switch.
%    \begin{macrocode}
\newif\if@svgx@standalone
\newcommand*\svgxsetbox[4][]{%
  \@svgx@standalonetrue%
  \svgx@setbox{#1}{#2}{#3}{#4}%
  \scr@ifundefinedorrelax{AtEndPreamble}{%
    \let\svg@tempa\@firstofone%
  }{%
    \def\svg@tempa{\AtEndPreamble}%
  }%
  \svg@tempa{\AtBeginDocument{\svgx@setbox{#1}{#2}{#3}{#4}}}%
}
\newcommand*\svgx@setbox[4]{%
  \sbox\svg@box{%
    \svg@local@param@set{#1,draft=false}%
    \Gin@draftfalse%
    \svg@ink@include{#2}{#3}{#4}%
  }%
  \svgxsetpapersize% 
}
%    \end{macrocode}
% \end{macro}^^A \if@svgx@standalone
% \end{macro}^^A \svgx@setbox
% \end{macro}^^A \svgxsetbox
% \begin{macro}{\svgxsetpapersize}
% \changes{v2.00a}{2017/02/28}{Bug fix for checking stock- and mediasizes}^^A
% This macro sets all well known length macros for defining the paper size as 
% well as the type area to the size of \cs{svg@box}.
%    \begin{macrocode}
\newcommand*\svgxsetpapersize{%
  \setlength\paperwidth{\the\wd\svg@box}%
%    \end{macrocode}
% Due to the fact, that the lengths for stock- and mediasizes are maybe set to
% \cs{relax}, these macros are checked with \cs{scr@ifundefinedorrelax}.
%    \begin{macrocode}
  \scr@ifundefinedorrelax{stockwidth}{}{%
    \setlength\stockwidth{\paperwidth}%
  }%
  \scr@ifundefinedorrelax{mediawidth}{}{%
    \setlength\mediawidth{\paperwidth}%
  }%
  \setlength\textwidth{\paperwidth}%
  \setlength\paperheight{\the\dimexpr\ht\svg@box+\dp\svg@box\relax}%
  \scr@ifundefinedorrelax{stockheight}{}{%
    \setlength\stockheight{\paperheight}%
  }%
  \scr@ifundefinedorrelax{mediaheight}{}{%
    \setlength\mediaheight{\paperheight}%
  }%
  \setlength\textheight{\paperheight}%
%    \end{macrocode}
% Any other length regarding the layout is set to have no influence at all. 
% Hence the document has the same size as the graphic.
%    \begin{macrocode}
  \hoffset=-1in%
  \oddsidemargin=\z@%
  \evensidemargin=\z@%
  \voffset=-1in%
  \topmargin=\z@%
  \headheight=\z@%
  \headsep=\z@%
  \topskip=\z@%
  \footskip=\z@%
  \marginparsep=\z@%
  \marginparwidth=\z@%
  \marginparpush=\z@%
}
\@onlypreamble\svgxsetpapersize
%    \end{macrocode}
% \end{macro}^^A \svgxsetpapersize
% \begin{macro}{\svgxoutputbox}
% \begin{macro}{\if@svgx@beamer}
% With \cs{svgxoutputbox} the created box is displayed.
%    \begin{macrocode}
\newif\if@svgx@beamer
\@ifclassloaded{beamer}{\@svgx@beamertrue}{}%
\newcommand*\svgxoutputbox{%
  \begingroup%
    \setlength\parindent{\z@}%
    \setlength\parskip{\z@}%
    \setlength\parfillskip{\z@}%
    \if@svgx@beamer%
      \setbeamertemplate{navigation symbols}{}%
      \begin{frame}[plain]%
      \usebox\svg@box%
      \end{frame}%
    \else%
      \usebox\svg@box%
    \fi%
    \endgraf%
  \endgroup%
}
%    \end{macrocode}
% \end{macro}^^A \if@svgx@beamer
% \end{macro}^^A \svgxoutputbox
%
% \iffalse
%</package&extract&option>
%<*package&option>
% \fi
%
%
%
% \section{Processing Options}
%
% Setting the default options and processing the given ones during when loading 
% the packages.
%    \begin{macrocode}
%<*base>
\FamilyExecuteOptions{SVG}{%
  inkscape=true,inkscapeversion=0,inkscapepath=basesubdir,%
  inkscapelatex=true,inkscapearea=drawing,distort=false,%
  usexcolor=true,usetransparent=true%
}
%</base>
%<*extract>
\FamilyExecuteOptions{SVG}{%
  extract=forced,extractpath=basesubdir,%
  extractruns=2,extractname=namenumbered,extractdistort=inherit,%
  convert=magick,convert=false,%
  gsdevice={png=png16m},gsdevice={jpeg=jpeg},gsdevice={jpg=jpeg},%
  gsdevice={tif=tiff48nc},gsdevice={tiff=tiff48nc},%
  gsdevice={eps=eps2write},gsdevice={ps=ps2write}%
}
%</extract>
\FamilyProcessOptions{SVG}
%    \end{macrocode}
%
% \iffalse
%</package&option>
%<*package&base&body>
% \fi
%
%
%
% \section{Macros for file access}
%
% \begin{macro}{\svg@shell@mkdir}
% \begin{macro}{\svg@shell@@mkdir}
% \begin{macro}{\svg@shell@mv}
% \begin{macro}{\svg@shell@@mv}
% \begin{macro}{\svg@shell@rm}
% \begin{macro}{\svg@shell@@rm}
% Finally, platform macros for creating directories as well as moving and 
% deleting files are provided.
%    \begin{macrocode}
\newcommand*\svg@shell@mkdir[1]{%
  \begingroup%
%    \end{macrocode}
% A directory should only be created, if it isn't the current working directory.
%    \begin{macrocode}
    \svg@quotes@remove[{#1}]{\svg@tempa}%
    \@svg@tempswatrue%
    \ifstr{\svg@tempa}{}{\@svg@tempswafalse}{%
    \ifstr{\svg@tempa}{\@currdir}{\@svg@tempswafalse}{%
    }}%
    \if@svg@tempswa%
      \ShellEscape{\svg@shell@@mkdir{\svg@tempa}}%
    \fi%
  \endgroup%
}
\newcommand*\svg@shell@mv[2]{%
  \begingroup%
%    \end{macrocode}
% Expand in order to prevent overwriting \cs{svg@tempa} or \cs{svg@tempb} if 
% given as argument.
% \ToDo{pass first argument directly to svg@IfFileExists}[v2.03]
%    \begin{macrocode}
    \edef\svg@tempa{%
      \noexpand\svg@quotes@remove[{#1}]{\noexpand\svg@tempa}%
      \noexpand\svg@quotes@remove[{#2}]{\noexpand\svg@tempb}%
    }%
    \svg@tempa%
    \svg@IfFileExists{\svg@tempa}{%
      \ifx\svg@tempa\svg@tempb\else%
        \ShellEscape{\svg@shell@@mv\space"\svg@tempa"\space"\svg@tempb"}%
      \fi%
    }{}%
  \endgroup%
}
\newcommand*\svg@shell@rm[1]{%
  \begingroup%
    \svg@quotes@remove[{#1}]{\svg@tempa}%
    \svg@IfFileExists{\svg@tempa}{%
      \ShellEscape{\svg@shell@@rm\space"\svg@tempa"}%
    }{}%
  \endgroup%
}
%    \end{macrocode}
% The platform dependent commands for file access. In order to do not raise a 
% warning, package \pkg{ifplatform} is only used if \file{-{}-shell-escape} 
% flag is enabled.
%    \begin{macrocode}
\ifnum\pdf@shellescape=\@ne\relax
  \svg@RequirePackage{ifplatform}[2010/10/22]
\fi
\@ifpackageloaded{ifplatform}{
  \ifwindows
    \newcommand*\svg@shell@@mkdir[1]{if not exist "#1" mkdir "#1"}
    \newcommand*\svg@shell@@mv{move}
    \newcommand*\svg@shell@@rm{del}
  \else
    \newcommand*\svg@shell@@mkdir[1]{mkdir -p "#1"}
    \newcommand*\svg@shell@@mv{mv}
    \newcommand*\svg@shell@@rm{rm}
  \fi
}{}
%    \end{macrocode}
% \end{macro}^^A \svg@shell@@rm
% \end{macro}^^A \svg@shell@rm
% \end{macro}^^A \svg@shell@@mv
% \end{macro}^^A \svg@shell@mv
% \end{macro}^^A \svg@shell@@mkdir
% \end{macro}^^A \svg@shell@mkdir
%
% \iffalse
%</package&base&body>
% \fi
%
% \Finale
%
\endinput
%
% \iffalse
%<*package&old>
% \fi
%
%
%
% \section{Old Implementation}
%
%   The package, which requires the packages \pkg{xkeyval} for the options,
%   \pkg{subfig} for automatic labeling within the subfigure command, the 
%   \pkg{import} package for correct handling of paths, \pkg{graphicx} for the
%   PDF~file inclusion commands, \pkg{transparent} for transparency, and 
%   \pkg{xcolor} for color, is initialized.
%    \begin{macrocode}
\ProvidesPackage{svg}[2012/09/05 v1.0 SVG inclusion and extraction]%
\@ifpackageloaded{xkeyval}{}{\RequirePackage{xkeyval}}%
\@ifpackageloaded{subfig}{}{\RequirePackage{subfig}}%
\@ifpackageloaded{import}{}{\RequirePackage{import}}%
\@ifpackageloaded{graphicx}{}{\RequirePackage{graphicx}}%
\@ifpackageloaded{transparent}{}{\RequirePackage{transparent}}%
\@ifpackageloaded{xcolor}{}{\RequirePackage{xcolor}}%
%    \end{macrocode}
%
%
%
% \subsection{Input definition}
%
%   All commands used for input (i.e. for the SVG~file and preamble) are 
%   defined within the \cs{SVG@in} prefix, and set by the key definition of the 
%   line following their definition. The \opt{exclude} boolean, used to stop the
%   inclusion of the SVG~file within the document is also defined.
%    \begin{macrocode}
\def\SVG@in@preamble{\jobname.tex}%
\define@key[SVG]{svg.sty}{preamble}{\def\SVG@in@preamble{#1}}%
\def\SVG@in@path{./}%
\define@key[SVG]{svg.sty}{svgpath}{\def\SVG@in@path{#1}}%
\def\SVG@in@end{\begin{document}}%
\define@key[SVG]{svg.sty}{end}{\def\SVG@in@end{#1}}%
\define@boolkey[SVG]{svg.sty}[SVG@in@]{exclude}[true]{}%
%    \end{macrocode}
%
%
%
% \subsection{Length definition}
%
%   All commands used for output are defined within the \cs{SVG@out} prefix,
%   beginning with the dimensions of the extracted image. If no dimensions are
%   supplied both \cs{ifSVG@out@usewidth} and \cs{ifSVG@out@useheight} are 
%   \val{false}, and so the natural dimensions of the SVG~file are used.
%   If both \cs{ifSVG@out@usewidth} and \cs{ifSVG@out@useheight} are 
%   \val{true}, the width is used.
%    \begin{macrocode}
\newlength\SVG@out@width%
\newif\ifSVG@out@usewidth%
\define@key[SVG]{svg.sty}{width}{%
  \setlength{\SVG@out@width}{#1}\SVG@out@usewidthtrue%
}%
\newlength\SVG@out@height%
\newif\ifSVG@out@useheight%
\define@key[SVG]{svg.sty}{height}{%
  \setlength{\SVG@out@height}{#1}\SVG@out@useheighttrue%
}%
%    \end{macrocode}
%
%
%
% \subsection{Extract booleans}
%
%   The booleans for the extraction formats are defined. Additionally,
%   the global export variable is defined, which is set to \val{true}
%   whenever any extraction is requested.
%    \begin{macrocode}
\define@boolkey[SVG]{svg.sty}[SVG@out@]{pdf}[true]{}%
\define@boolkey[SVG]{svg.sty}[SVG@out@]{eps}[true]{}%
\define@boolkey[SVG]{svg.sty}[SVG@out@]{png}[true]{}%
\newif\ifSVG@out@extract
%    \end{macrocode}
%
%
%
% \subsection{Output definitions}
%
%   The extraction path, extraction root name, clean boolean,
%   pre-\LaTeX~commands, and post-\LaTeX~commands are defined.
%    \begin{macrocode}
\def\SVG@out@path{./}%
\define@key[SVG]{svg.sty}{path}{\def\SVG@out@path{#1}}%
\def\SVG@out@name{Fig.\arabic{svgfigure}\alph{subfigure}}%
\define@key[SVG]{svg.sty}{name}{\def\SVG@out@name{#1}}%
\define@boolkey[SVG]{svg.sty}[SVG@out@]{clean}[true]{}%
\def\SVG@out@pretex{}%
\define@key[SVG]{svg.sty}{pretex}{\def\SVG@out@pretex{#1}}%
\def\SVG@out@postex{}%
\define@key[SVG]{svg.sty}{postex}{\def\SVG@out@postex{#1}}%
%    \end{macrocode}
%
%
%
% \subsection{Command definitions}
%
%   The command options are defined within the prefix \cs{SVG@cmd} and
%   are set by the key definition following each command definition.
%    \begin{macrocode}
\def\SVG@cmd@inkscape{inkscape -z -C}%
\define@key[SVG]{svg.sty}{inkscape}{\def\SVG@cmd@inkscape{#1}}%
\def\SVG@cmd@pdflatex{pdflatex}%
\define@key[SVG]{svg.sty}{pdflatex}{\def\SVG@cmd@pdflatex{#1}}%
\def\SVG@cmd@pdftops{pdftops -eps}%
\define@key[SVG]{svg.sty}{pdftops}{\def\SVG@cmd@pdftops{#1}}%
\def\SVG@cmd@convert{convert -density 300}%
\define@key[SVG]{svg.sty}{convert}{\def\SVG@cmd@convert{#1}}%
%    \end{macrocode}
%
%
%
% \subsection{Process options}
%
%   All the options for the package are processed, and a counter is defined, 
%   which is used to correctly handle the \cnt{subfigure} counting.
%    \begin{macrocode}
\ProcessOptionsX[SVG]%
\newcounter{svgfigure}[figure]%
%    \end{macrocode}
%
%
%
% \subsection{User interface}
%
%   Define the macro to globally set keys.
%    \begin{macrocode}
\def\setsvg#1{\setkeys[SVG]{svg.sty}{#1}}%
%    \end{macrocode}
%
%   Define the macro used to include a SVG~file. Set the keys and
%   determine if extraction should occur.
%    \begin{macrocode}
\newread\SVG@in@file%
\newwrite\SVG@out@file%
\def\includesvg{\@ifnextchar[\@includesvg{\@includesvg[]}}%
\def\@includesvg[#1]#2{%
  \setkeys[SVG]{svg.sty}{#1}%
  \SVG@out@extractfalse%
  \ifSVG@out@pdf \SVG@out@extracttrue \fi%
  \ifSVG@out@eps \SVG@out@extracttrue \fi%
  \ifSVG@out@png \SVG@out@extracttrue \fi%
%    \end{macrocode}
% Run \app{Inkscape} to separate the SVG~file into text and image. Only 
% run \app{Inkscape} if the SVG~file is newer than the generated text 
% and image.
%    \begin{macrocode}
  \ifnum\pdfstrcmp{%
    \pdffilemoddate{\SVG@in@path#2.svg}%
  }{%
    \pdffilemoddate{\SVG@in@path#2.pdf}%
  }>0\relax%
    \immediate\write18{%
      \SVG@cmd@inkscape\space%
        -f \SVG@in@path#2.svg\space%
        -A \SVG@in@path#2.pdf\space%
        --export-latex%
    }%
  \fi%
%    \end{macrocode}
% Determine the image width and height using \cs{includegraphics}.
%    \begin{macrocode}
  \ifSVG@out@usewidth%
    \settoheight\SVG@out@height{%
      \includegraphics[width=\SVG@out@width]{\SVG@in@path#2}%
    }%
  \else%
    \ifSVG@out@useheight%
      \settowidth\SVG@out@width{%
        \includegraphics[height=\SVG@out@height]{\SVG@in@path#2}%
      }%
    \else%
      \settoheight\SVG@out@height{%
        \includegraphics{\SVG@in@path#2}%
      }%
      \settowidth\SVG@out@width{\includegraphics{\SVG@in@path#2}}%
    \fi%
  \fi%
%    \end{macrocode}
% Open the output file for extraction.
%    \begin{macrocode}
% Open the output file.
  \ifSVG@out@extract%
    \setcounter{svgfigure}{\value{figure}}%
    \stepcounter{svgfigure}%
    \def\SVG@out@filename{\SVG@out@name}%
    \immediate\openout\SVG@out@file=\SVG@out@path\SVG@out@filename.tex%
  \fi%
%    \end{macrocode}
% Open and write the preamble. Notice that the catcodes for |#| need
% to be changed to prevent double expansion when reading the line.
%    \begin{macrocode}
  \ifSVG@out@extract%
    \def\SVG@in@line{}%
    \immediate\openin\SVG@in@file=\SVG@in@preamble%
  \fi%
  \newif\ifSVG@in@read%
  \ifSVG@out@extract \SVG@in@readtrue \fi%
  \@whilesw\ifSVG@in@read\fi{%
    \catcode`\#=12\relax\endlinechar=-1%
    \immediate\read\SVG@in@file to \SVG@in@line%
    \ifx\SVG@in@end\SVG@in@line%
      \SVG@in@readfalse%
    \else%
      \immediate\write\SVG@out@file{\unexpanded\expandafter{\SVG@in@line}}%
    \fi%
    \ifeof\SVG@in@file\SVG@in@readfalse\fi%
    \endlinechar=13\catcode`\#=6\relax%
  }%
  \ifSVG@out@extract \immediate\closein\SVG@in@file \fi%
%    \end{macrocode}
% Now write everything needed after the preamble. This includes
% requiring the \pkg{import} package and defining all the dimensions need
% to match the document size with the image size.
%    \begin{macrocode}
  \ifSVG@out@extract%
    \def\SVG@out@defpack{%
      \makeatletter%
      \@ifpackageloaded{import}{}{\RequirePackage{import}}%
      \@ifpackageloaded{graphicx}{}{\RequirePackage{graphicx}}%
      \@ifpackageloaded{transparent}{}{\RequirePackage{transparent}}%
      \@ifpackageloaded{xcolor}{}{\RequirePackage{xcolor}}%
      \makeatother%
    }%
    \def\SVG@out@defwidth{\def\svgwidth{0.99\textwidth}}%
    \def\SVG@out@definput{\import{\SVG@in@path}{#2.pdf_tex}}%
    \immediate\write\SVG@out@file{\unexpanded\expandafter{\SVG@out@defpack}}%
    \immediate\write\SVG@out@file{%
      \noexpand\setlength{\pdfpagewidth}{\the\SVG@out@width}%
    }%
    \immediate\write\SVG@out@file{%
      \noexpand\setlength{\pdfpageheight}{\the\SVG@out@height}%
    }%
    \immediate\write\SVG@out@file{%
      \noexpand\setlength{\paperheight}{\pdfpageheight}%
    }%
    \immediate\write\SVG@out@file{%
      \noexpand\setlength{\paperwidth}{\pdfpagewidth}%
    }%
    \immediate\write\SVG@out@file{%
      \noexpand\setlength{\textheight}{\paperheight}%
    }%
    \immediate\write\SVG@out@file{%
      \noexpand\setlength{\textwidth}{\paperwidth}%
    }%
    \immediate\write\SVG@out@file{%
      \noexpand\setlength{\textheight}{\paperheight}%
    }%
    \immediate\write\SVG@out@file{\noexpand\setlength{\oddsidemargin}{-1in}}%
    \immediate\write\SVG@out@file{\noexpand\setlength{\evensidemargin}{-1in}}%
    \immediate\write\SVG@out@file{\noexpand\setlength{\topmargin}{-1in}}%
    \immediate\write\SVG@out@file{\noexpand\setlength{\headheight}{0in}}%
    \immediate\write\SVG@out@file{\noexpand\setlength{\headsep}{0in}}%
    \immediate\write\SVG@out@file{\noexpand\setlength{\topskip}{0in}}%
    \immediate\write\SVG@out@file{\noexpand\setlength{\footskip}{0in}}%
    \immediate\write\SVG@out@file{\noexpand\setlength{\parindent}{0in}}%
    \immediate\write\SVG@out@file{\noexpand\setlength{\parsep}{0in}}%
    \immediate\write\SVG@out@file{\noexpand\setlength{\parskip}{0in}}%
    \immediate\write\SVG@out@file{%
      \noexpand\begin{document}%
      \noexpand\pagestyle{empty}%
      \noexpand\begin{center}%
        \unexpanded\expandafter{\SVG@out@defwidth}%
        \unexpanded\expandafter{\SVG@out@pretex}%
        \expandafter\noexpand\SVG@out@definput%
        \unexpanded\expandafter{\SVG@out@postex}%
        \noexpand\end{center}%
        \noexpand\end{document}%
      }%
    \immediate\closeout\SVG@out@file%
  \fi%
%    \end{macrocode}
% Run \LaTeX{} on the extracted file and create the PDF~file.
%    \begin{macrocode}
  \ifSVG@out@extract%
    \immediate\write18{%
      \SVG@cmd@pdflatex\space\SVG@out@path\SVG@out@filename.tex%
    }%
  \fi%
%    \end{macrocode}
% Convert the PDF~file to EPS~file if requested.
%    \begin{macrocode}
  \ifSVG@out@eps%
    \immediate\write18{\SVG@cmd@pdftops\space\SVG@out@filename.pdf}%
    \immediate\write18{%
      mv \SVG@out@filename.eps\space\SVG@out@path\SVG@out@filename.eps%
    }%
  \fi%
%    \end{macrocode}
% Convert the PDF~file to PNG~file if requested.
%    \begin{macrocode}
  \ifSVG@out@png%
    \immediate\write18{%
      \SVG@cmd@convert\space%
        \SVG@out@filename.pdf\space\SVG@out@filename.png%
    }%
    \immediate\write18{%
      mv \SVG@out@filename.png\space\SVG@out@path\SVG@out@filename.png%
    }%
  \fi%
%    \end{macrocode}
% Clean up if requested.
%    \begin{macrocode}
  \ifSVG@out@extract%
    \ifSVG@out@pdf%
      \immediate\write18{%
        mv \SVG@out@filename.pdf\space\SVG@out@path\SVG@out@filename.pdf%
      }%
    \else%
      \ifSVG@out@clean \immediate\write18{rm \SVG@out@filename.pdf} \fi%
    \fi%
    \ifSVG@out@clean%
      \immediate\write18{%
        rm\space%
          \SVG@out@path\SVG@out@filename.tex\space%
          \SVG@out@filename.aux\space%
          \SVG@out@filename.log\space%
          \SVG@out@filename.out%
      }%
    \fi%
  \fi%
%    \end{macrocode}
% Finally, include the SVG~file in the current document and end the package.
%    \begin{macrocode}
  \ifSVG@in@exclude\else%
    {%
      \def\svgwidth{\the\SVG@out@width}%
      \SVG@out@pretex%
      \import{\SVG@in@path}{#2.pdf_tex}%
      \SVG@out@postex%
    }%
  \fi%
}%
%    \end{macrocode}
%
% \iffalse
%</package&old>
% \fi
%
% \Finale
%
\endinput
