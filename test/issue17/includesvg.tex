%\RequirePackage[2015/01/01]{latexrelease}
\documentclass[ngerman]{scrartcl}
\iftutex
  \usepackage{fontspec}
\else
  \usepackage[T1]{fontenc}
  \usepackage[ngerman=ngerman-x-latest]{hyphsubst}
\fi
\usepackage{lmodern}
%\usepackage{babel}
\usepackage[]{graphicx}
\usepackage[inkscape=forced,draft=false]{svg}
\usepackage[extract=forced,clean=false,convert=forced]{svg-extract}
%\usepackage[space]{grffile}

%\svgsetup{inkscapeexe="C:/Program Files/Inkscape1/inkscape.exe"}
%\svgsetup{inkscapeversion=detect}
%\svgsetup{inkscapeversion=0.94.2}
%\svgsetup{inkscapepath=}
\svgpath{{"test ink"/}{foo}}
%\graphicspath{{C:/Users/Hanisch/}}

\begin{document}
\makeatletter

% TODO extractformat=eps,convertformat=eps > move?
% TODO extractformat={pdf,eps},convertformat=eps
% TODO extractformat=eps,convertformat=pdf > clean?!
% TODO extractformat=,convertformat=pdf
% \includesvg[%
% %  inkscapename="aaa bbb",
  % extract=false,
  % extractname="foo bar",
% %  inkscapeformat=eps,
% %  extractpreamble="aaa/bbb",
  % extractformat={ps},%pdf,eps,ps},
  % convertformat={png},%
  % latexext=".ltx"
% ]{svg-example}

\includegraphics{svg-example_svg-latex.pdf}

\makeatletter
\meaning\input@path

\meaning\Ginput@path

\IfFileExists{svg-example_svg-latex.pdf}{then}{else}
\meaning\@filef@und

\IfFileExists{svg-example.svg}{then}{else}
\meaning\@filef@und

\openin\@inputcheck"svg-example.svg" %
\ifeof\@inputcheck
nope
\else
jepp
\fi

%\includesvg[]{MWE}
%
%\includesvg[pretex={\catcode`\_11},width=\textwidth]{pipeline-test}

%\meaning\transparent
%
%\includesvg[inkscapelatex=false]{mwe bla/MWE}

%\includeinkscape[]{"svg-example_svg"-tex.PdF}
%
%\includeinkscape[]{MWE_svg-latex}
%
%\includeinkscape[extractpreamble="bbb"]{svg-example_svg-tex.pDf_TEX}

%\def\bla{ aaa " bbb " ccc }
%\svg@quotes@normalize{\bla}
%\meaning\bla
%
%\def\bla{ aaa " bbb " ccc }
%\svg@quotes@normalize[" vvv "]{\bla}
%\meaning\bla

\end{document}


\includesvg[
%  lastpage=1,
  inkscapepath=basesubpath,inkscape=forced,
%  extract=false,
  %extractformat={bar,baar,bla,eps,ps},
  convertformat={png,jpg},%inkscapepath=,
%  inkscapename=  "      aaa ",%vvv,
%  width=6cm,
%  height=5cm,% 
%  scale=1.5,
%%  extractwidth=\relax,
%%  extractheight=8cm ,
%  extractscale=2.5,
%  extract=force
]{svg example .foo.svg}
\end{document}

\includesvg[
  lastpage=1,
  convertformat={png,jpg},
  inkscapepath=svgpath,
%  width=6cm,
%  height=5cm,%
%  scale=1.5,
%%  extractwidth=\relax,
%%  extractheight=8cm ,
%  extractscale=2.5,
%  extract=force
]{svg.example}
%
%\includeinkscape[%
%  extractpreamble=aaa bbb.bla,
%  lastpage=1,extract=force,
%  convertformat={png,jpg},
%  extractname=./bla/testfile,
%%  inkscapepath=,
%%  width=6cm,
%%  height=5cm,%
%%  scale=1.5,
%%%  extractwidth=\relax,
%%%  extractheight=8cm ,
%%  extractscale=2.5,
%%  extract=force
%]{svgexample.foo_svg-tex}

%\includeinkscapex[
%  lastpage=1,
%  extractwidth=7cm ,
%  extractheight=8cm ,
%  extractscale=2.5
%]{"svg-example_svg-tex.PDF_TEX"}
\end{document}

\begin{figure}
%\centering
%\def\svgwidth{12cm}
%\svgsetup{width=5cm}

%\let\svgwidth\relax
%\def\svgwidth{3cm}
%\fbox{%
\includesvg[%
%  clean=true,
%  inkscapepath="aaa bbb",
%  extractpath="hi hh",
%  extractname=,
%  convert,
%  inkscapearea=crop,
%  pretex={\footnote{aaa}},%
%  height=\relax,
%  width=5cm,
%  command=test,
%  width=relax,
%  width=.6\textwidth,
%%  extractangle=2,
%%  extractheight=-3cm,
%  scale=1.5,%
%  inkscapelatex=false,
%][
%  distort, 
%  keepaspectratio,
%%  draft,
%  extractheight=3cm,
%  extractscale=0.5,
%  angle=15,
%  origin=r,
%  lastpage=3,
%%  distort,
%  extractdistort,
%][
%  width=5cm,
%  draft
]{svg-example}%
%}
%\fbox{}%
%\includesvgx[foo=blubb,angle=15]{svg-example}
%\includesvg[scale=0.7,inkscape=true,extract=false,exclude=false,convert=false]{svg-example}
%\fbox{\includesvg[extractformat={,pdf},convertformat={png,}]{svg-example}}
%\includesvg[pretex={\catcode`\_11},width=\textwidth]{pipeline-test}
\end{figure}

%\fbox{\includegraphics[scale=2,height=5cm]{./svg-inkscape/svg-example_svg-tex.pdf}}

\end{document}

\listfiles
\documentclass{scrartcl}
\iftutex
  \usepackage{fontspec}
\else
  \usepackage[T1]{fontenc}
  \usepackage[ngerman=ngerman-x-latest]{hyphsubst}
\fi
\usepackage{graphicx}
\usepackage{xcolor}
\usepackage{svg}
\usepackage{svg-extract}

%\svgpath{{C:/texlive/2019/texmf-dist/doc/latex/svg/}}
\svgsetup{%
%  inkscapeexe="C:/Program Files/Inkscape1/inkscape",
%  gsexe="C:/Program Files/gs/gs9.20/bin/gswin64",
}

\begin{document}
\begin{figure}
\centering
\svgsetup{clean=true}
\fbox{\includesvg[extractformat=,convertformat={png}]{svg-example}}
\caption{text\label{fig:example-tex}}
\end{figure}
\end{document}

\listfiles
\documentclass[parskip=full,ngerman]{scrreprt}% Test
\usepackage[T1]{fontenc}
\usepackage{babel}
\usepackage{etoolbox}
\usepackage{svg}
%\usepackage{svg-extract}
%\svgsetup{clean=true}
\providecommand{\transparent}[1]{}

%\pdfsuppresswarningpagegroup=1

\svgpath{{../source/examples/}}

\begin{document}
\includesvg[%
  width=20cm,height=8cm,%inkscapelatex=false
%  inkscapeformat=pdf,
%  inkscapelatex=false,
%  distort=true,
  angle=-12.5,
%  extractdistort=false,
%  extractangle=inherit,
]{svg-example}%


\includeinkscape[%
  width=20cm,height=8cm,%inkscapelatex=false
%  inkscapeformat=pdf,
%  distort=true,
  angle=12.5,
%  extractdistort=false,
%  extractangle=inherit,
]{svg-example_svg-tex.pdf}%
\end{document}
