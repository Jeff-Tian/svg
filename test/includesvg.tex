\listfiles
\documentclass{scrartcl}
\ifpdftex{
  \usepackage[T1]{fontenc}
}{
  \usepackage{fontspec}
}

\usepackage{graphicx}
\usepackage{xcolor}
\usepackage{svg}
\usepackage{svg-extract}

\svgpath{{C:/texlive/2019/texmf-dist/doc/latex/svg/}}
\svgsetup{%
%  inkscapeexe="C:/Program Files/Inkscape1/inkscape",
}

\begin{document}
\begin{figure}
\centering
\svgsetup{clean=true}
\fbox{\includesvg[extractformat=,convertformat={png}]{svg-example}}
\caption{text}\label{fig:example-tex}
\end{figure}
\end{document}

\listfiles
\documentclass[parskip=full,ngerman]{scrreprt}% Test
\usepackage[T1]{fontenc}
\usepackage{babel}
\usepackage{etoolbox}
\usepackage{svg}
%\usepackage{svg-extract}
%\svgsetup{clean=true}
\providecommand{\transparent}[1]{}

%\pdfsuppresswarningpagegroup=1

\svgpath{{../source/examples/}}

\begin{document}
\includesvg[%
  width=20cm,height=8cm,%inkscapelatex=false
%  inkscapeformat=pdf,
%  inkscapelatex=false,
%  distort=true,
  angle=-12.5,
%  extractdistort=false,
%  extractangle=inherit,
]{svg-example}%


\includeinkscape[%
  width=20cm,height=8cm,%inkscapelatex=false
%  inkscapeformat=pdf,
%  distort=true,
  angle=12.5,
%  extractdistort=false,
%  extractangle=inherit,
]{svg-example_svg-tex.pdf}%
\end{document}
