\listfiles
\documentclass[ngerman]{scrartcl}
\ifpdftex{
  \usepackage[T1]{fontenc}
}{
  \usepackage{fontspec}
}
\usepackage{babel}
%\usepackage[draft]{graphicx}
\usepackage[auto]{svg}
\usepackage[auto,clean]{svg-extract}

%\RequirePackage{ifplatform}[2010/10/22]

\svgsetup{inkscapeexe="C:/Program Files/Inkscape1/inkscape.exe"}
\svgsetup{inkscapeversion=detect}
%\svgsetup{inkscapeversion=0.94.2}

\begin{document}
\edef\bla{\csname svg@ink@ver\endcsname}
\meaning\bla
\includesvg[
  lastpage=1,
  extract=forced,convert=forced,extractformat={bar,baar,bla,pdf,eps}
%  width=6cm,
%  height=5cm,%
%  scale=1.5,
%%  extractwidth=\relax,
%%  extractheight=8cm ,
%  extractscale=2.5,
%  extract=force
]{"svg-example"}
%\includeinkscape[
%  lastpage=1,
%  extractwidth=7cm ,
%  extractheight=8cm ,
%  extractscale=2.5
%]{"svg-example_svg-tex.PDF_TEX"}
blubb
foo
\end{document}

\begin{figure}
%\centering
%\def\svgwidth{12cm}
%\svgsetup{width=5cm}

%\let\svgwidth\relax
%\def\svgwidth{3cm}
%\fbox{%
\includesvg[%
%  clean=true,
%  inkscapepath="aaa bbb",
%  extractpath="hi hh",
%  extractname=,
%  convert,
%  inkscapearea=crop,
%  pretex={\footnote{aaa}},%
%  height=\relax,
%  width=5cm,
%  command=test,
%  width=relax,
%  width=.6\textwidth,
%%  extractangle=2,
%%  extractheight=-3cm,
%  scale=1.5,%
%  inkscapelatex=false,
%][
%  distort, 
%  keepaspectratio,
%%  draft,
%  extractheight=3cm,
%  extractscale=0.5,
%  angle=15,
%  origin=r,
%  lastpage=3,
%%  distort,
%  extractdistort,
%][
%  width=5cm,
%  draft
]{svg-example}%
%}
%\fbox{}%
%\includesvgx[foo=blubb,angle=15]{svg-example}
%\includesvg[scale=0.7,inkscape=true,extract=false,exclude=false,convert=false]{svg-example}
%\fbox{\includesvg[extractformat={,pdf},convertformat={png,}]{svg-example}}
%\includesvg[pretex={\catcode`\_11},width=\textwidth]{pipeline-test}
\end{figure}

%\fbox{\includegraphics[scale=2,height=5cm]{./svg-inkscape/svg-example_svg-tex.pdf}}

\end{document}

\listfiles
\documentclass{scrartcl}
\ifpdftex{
  \usepackage[T1]{fontenc}
}{
  \usepackage{fontspec}
}

\usepackage{graphicx}
\usepackage{xcolor}
\usepackage{svg}
\usepackage{svg-extract}

%\svgpath{{C:/texlive/2019/texmf-dist/doc/latex/svg/}}
\svgsetup{%
%  inkscapeexe="C:/Program Files/Inkscape1/inkscape",
%  gsexe="C:/Program Files/gs/gs9.20/bin/gswin64",
}

\begin{document}
\begin{figure}
\centering
\svgsetup{clean=true}
\fbox{\includesvg[extractformat=,convertformat={png}]{svg-example}}
\caption{text}\label{fig:example-tex}
\end{figure}
\end{document}

\listfiles
\documentclass[parskip=full,ngerman]{scrreprt}% Test
\usepackage[T1]{fontenc}
\usepackage{babel}
\usepackage{etoolbox}
\usepackage{svg}
%\usepackage{svg-extract}
%\svgsetup{clean=true}
\providecommand{\transparent}[1]{}

%\pdfsuppresswarningpagegroup=1

\svgpath{{../source/examples/}}

\begin{document}
\includesvg[%
  width=20cm,height=8cm,%inkscapelatex=false
%  inkscapeformat=pdf,
%  inkscapelatex=false,
%  distort=true,
  angle=-12.5,
%  extractdistort=false,
%  extractangle=inherit,
]{svg-example}%


\includeinkscape[%
  width=20cm,height=8cm,%inkscapelatex=false
%  inkscapeformat=pdf,
%  distort=true,
  angle=12.5,
%  extractdistort=false,
%  extractangle=inherit,
]{svg-example_svg-tex.pdf}%
\end{document}
